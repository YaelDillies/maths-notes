\documentclass{article}

% preamble
\def\npart{III}
\def\nyear{2023}
\def\nterm{Michaelmas}
\def\nlecturer{Dr Julian Sahasrabudhe}
\def\ncourse{Ramsey Theory on Graphs}
\def\draft{Incomplete}

\ifx \nauthor\undefined
  \def\nauthor{Ya\"el Dillies}
\else
\fi

\author{Based on lectures by \nlecturer \\\small Notes taken by \nauthor}
\date{\nterm\ \nyear}
\ifdefined\draft
\title{Part \npart\ -- \ncourse\ (\draft)}
\else
\title{Part \npart\ -- \ncourse}
\fi

\usepackage[utf8]{inputenc}
\usepackage{amsmath}
\usepackage{amsthm}
\usepackage{amssymb}
\usepackage{cancel}
\usepackage{enumerate}
\usepackage{mathtools}
\usepackage{fancyhdr}
\usepackage{graphicx}
\usepackage[dvipsnames]{xcolor}
\usepackage{tikz}
\usepackage{wrapfig}
\usepackage{centernot}
\usepackage{float}
\usepackage{braket}
\usepackage{marginnote}
\usepackage{mathdots}
\usepackage{mathrsfs}
\usepackage{ifthen}
\usepackage{imakeidx}
\usepackage{parskip}
\usepackage{relsize}
\usepackage{tabularx}
\usepackage[hypcap=true]{caption}
\usepackage[shortlabels]{enumitem}
\usepackage[pdftex,
  colorlinks=true,
  linkcolor=lblue,
  pdfauthor={\nauthor},
  pdfsubject={Cambridge Maths Notes: Part \npart\ - \ncourse},
  pdftitle={\ncourse - Part \npart},
pdfkeywords={Cambridge Mathematics Maths Math \npart\ \nterm\ \nyear\ \ncourse}]{hyperref}

\usepackage[capitalise,nameinlink,noabbrev]{cleveref}
\usepackage{nameref}
\usepackage[margin=1.5in,a4paper]{geometry}

\reversemarginpar
\newcommand{\lecnum}[1]{\leavevmode\marginnote{\emph{Lecture #1}}\ignorespaces}
\newcounter{lecturenumber}
\newcommand{\newlec}{\stepcounter{lecturenumber}\lecnum{\arabic{lecturenumber}}}

% Theorems
\theoremstyle{definition}
\newtheorem*{aim}{Aim}
\newtheorem*{assumption}{Assumption}
\newtheorem*{axiom}{Axiom}
\newtheorem*{claim}{Claim}
\newtheorem*{cor}{Corollary}
\newtheorem*{conjecture}{Conjecture}
\newtheorem*{defi}{Definition}
\newtheorem*{eg}{Example}
\newtheorem*{egs}{Examples}
\newtheorem*{ex}{Exercise}
\newtheorem*{fact}{Fact}
\newtheorem*{goal}{Goal}
\newtheorem*{idea}{Idea}
\newtheorem*{law}{Law}
\newtheorem*{lemma}{Lemma}
\newtheorem*{notation}{Notation}
\newtheorem*{note}{Note}
\newtheorem*{obs}{Observation}
\newtheorem*{prop}{Proposition}
\newtheorem*{properties}{Properties}
\newtheorem*{question}{Question}
\newtheorem*{rrule}{Rule}
\newtheorem*{steps}{Steps}
\newtheorem*{thm}{Theorem}

\newtheorem*{rmk}{Remark}
\newtheorem*{rmks}{Remarks}
\newtheorem*{warning}{Warning}
\newtheorem*{exercise}{Exercise}

\newtheorem{nthm}{Theorem}[section]
\newtheorem{ndef}[nthm]{Definition}
\newtheorem{nprop}[nthm]{Proposition}
\newtheorem{nconjecture}{Conjecture}
\newtheorem{ncor}[nthm]{Corollary}
\newtheorem{nex}[nthm]{Example}
\newtheorem{nlemma}[nthm]{Lemma}
\newtheorem{problem}[nthm]{Problem}

% Command redirections
\let\P\oldP
\let\oldemptyset\emptyset
\let\emptyset\varnothing

% Letter shorthands
\newcommand{\C}{\mathbb C}
\newcommand{\bbE}{\mathbb E}
\newcommand{\F}{\mathbb F}
\newcommand{\K}{\mathbb K}
\newcommand{\N}{\mathbb N}
\newcommand{\P}{\mathbb P}
\newcommand{\Q}{\mathbb Q}
\newcommand{\R}{\mathbb R}
\newcommand{\Z}{\mathbb Z}
\newcommand{\mcA}{\mathcal A}
\newcommand{\mcB}{\mathcal B}
\newcommand{\mcC}{\mathcal C}
\newcommand{\mcD}{\mathcal D}
\newcommand{\mcE}{\mathcal E}
\newcommand{\mcF}{\mathcal F}
\newcommand{\mcG}{\mathcal G}
\newcommand{\mcH}{\mathcal H}
\newcommand{\mcM}{\mathcal M}
\newcommand{\mcN}{\mathcal N}
\newcommand{\mcO}{\mathcal O}
\newcommand{\mcP}{\mathcal P}
\newcommand{\mcQ}{\mathcal Q}
\newcommand{\mcR}{\mathcal R}
\newcommand{\mcS}{\mathcal S}
\newcommand{\mcT}{\mathcal T}
\newcommand{\mcU}{\mathcal U}
\newcommand{\mcV}{\mathcal V}
\newcommand{\eps}{\varepsilon}
\newcommand{\Eps}{\mathcal E}

\newcommand{\curlybrack}[1]{\left\{ #1\right\}}
\newcommand{\abs}[1]{\left\lvert #1\right\rvert}
\newcommand{\norm}[1]{\left\lVert #1\right\rVert}
\newcommand{\inn}[2]{\left\langle #1, #2\right\rangle}
\newcommand{\floor}[1]{\left\lfloor #1\right\rfloor}
\newcommand{\ceil}[1]{\left\lceil #1\right\rceil}
\newcommand{\doublesqbrack}[1]{[\![#1]\!]}

\newcommand{\imp}{\implies}
\newcommand{\for}{\forall}
\newcommand{\mor}{\rightarrow}
\newcommand{\nin}{\notin}
\newcommand{\comp}{\circ}
\newcommand{\union}{\cup}
\newcommand{\inter}{\cap}
\newcommand{\Union}{\bigcup}
\newcommand{\Inter}{\bigcap}
\newcommand{\hatplus}{\mathbin{\widehat{+}}}
\newcommand{\symdif}{\mathbin\varbigtriangleup}
\newcommand{\aeeq}{\overset{\text{ae}}=}
\newcommand{\lexlt}{\overset{\text{lex}}<}
\newcommand{\colexlt}{\overset{\text{colex}}<}
\newcommand{\wtendsto}{\overset w\mor}
\newcommand{\wstartendsto}{\overset{w*}\mor}
\renewcommand{\vec}[1]{\boldsymbol{\mathbf{#1}}}
\renewcommand{\bar}[1]{\overline{#1}}

\newcommand*{\E}{
  \mathop{
    \mathchoice{\vcenter{\hbox{\larger[4]$\mathbb{E}$}}}
               {\kern0pt\mathbb{E}}
               {\kern0pt\mathbb{E}}
               {\kern0pt\mathbb{E}}
  }\displaylimits
}

\newcommand{\named}[1]{\textbf{#1}\index{#1}}
\newcommand{\bonusnamed}[1]{\textbf{#1}\index{#1@*#1}}

\let\Im\relax
\let\Re\relax

\DeclareMathOperator{\Ber}{Ber}
\DeclareMathOperator{\conv}{conv}
\DeclareMathOperator{\diam}{diam}
\DeclareMathOperator{\codim}{codim}
\DeclareMathOperator{\esssup}{ess sup}
\DeclareMathOperator{\Ext}{Ext}
\DeclareMathOperator{\id}{id}
\DeclareMathOperator{\im}{im}
\DeclareMathOperator{\Im}{Im}
\DeclareMathOperator{\interior}{int}
\DeclareMathOperator{\lhs}{LHS}
\DeclareMathOperator{\rank}{rank}
\DeclareMathOperator{\Re}{Re}
\DeclareMathOperator{\rhs}{RHS}
\DeclareMathOperator{\Span}{Span}
\DeclareMathOperator{\Spec}{Spec}
\DeclareMathOperator{\supp}{supp}
\DeclareMathOperator{\Var}{Var}

\definecolor{lblue}{rgb}{0., 0.05, 0.6}
\definecolor{mblue}{rgb}{0.2, 0.3, 0.8}
\definecolor{morange}{rgb}{1, 0.5, 0}
\definecolor{mgreen}{rgb}{0.1, 0.4, 0.2}
\definecolor{mred}{rgb}{0.5, 0, 0}

\colorlet{bred}{red}
\colorlet{bblue}{Cyan!50!blue}
\colorlet{byellow}{yellow}
\colorlet{bgreen}{YellowGreen!50!Green}
\colorlet{borange}{red!20!yellow}
\colorlet{bpurple}{violet}

\newcommand{\red}[1]{\textcolor{bred}{#1}}
\newcommand{\green}[1]{\textcolor{bgreen}{#1}}
\newcommand{\blue}[1]{\textcolor{bblue}{#1}}
\newcommand{\yellow}[1]{\textcolor{byellow}{#1}}
\newcommand{\orange}[1]{\textcolor{borange}{#1}}
\newcommand{\purple}[1]{\textcolor{bpurple}{#1}}

\pagestyle{fancy}
\fancyhf{}
\fancyfoot[R]{\href{yaeldillies.github.io/maths-notes}{\color{lblue}{Updated online}}}
\fancyfoot[C]{\thepage}
\ifdefined\draft
\fancyfoot[L]{\emph{\draft}}
\else
\fi
\renewcommand{\headrulewidth}{0pt}
\renewcommand{\footrulewidth}{0.2pt}

% Counters and table of content

\swapnumbers
\reversemarginpar

\usetikzlibrary{positioning, decorations.pathmorphing, decorations.text, calc, backgrounds, fadings}
\tikzset{node/.style = {circle,draw,inner sep=0.8mm}}

\makeindex[intoc]

\setcounter{section}{-1}

% and here we go!
\begin{document}
\maketitle

\tableofcontents

\clearpage
\section{Introduction}

\newlec

\begin{notation}
  We write
  \begin{itemize}
    \item $[n] = \{1, \dots, n\}$
    \item $K_n$ for the complete graph on $n$ vertices.
    \item For $X$ a set, $r \in \N$, $X^{(r)} = \{S \subseteq X | |S| = r\}$
    \item $\chi$ for a $k$-coloring of the edges of $K_n$
      \begin{align*}
        \chi : E(K_n) & \to [k] \\
        \chi : E(K_n) & \to \{\text{red}, \text{blue}\} & (\text{if } k = 2)
      \end{align*}
  \end{itemize}
\end{notation}

Ramsey theory is usually concerned with the following question:

\begin{quotation}
  \textit{Can we find some order in enough disorder?}
\end{quotation}

In this course, we will specialise this question to graphs. We are thus interested in the following:

\begin{quotation}
  \textit{What can we say about the structure of an arbitrary $2$-coloring of the edges of $K_n$?}
\end{quotation}

\begin{dfn}
  Define the {\bf Ramsey number} $R(\ell, k)$ to be the least $n$ for which every $2$-edge coloring contains either a blue $K_\ell$ or a red $K_k$, and the {\bf diagonal Ramsey number} $R(k) = R(k, k)$ to be the least $n$ for which every $2$-edge coloring contains a monochromatic $K_k$.
\end{dfn}

It is unclear that such a $n$ even exists! We shall prove it in due course.

$R(\ell, k) = R(k, \ell)$. By convention, we will usually assume $\ell \le k$.

\begin{eg}
  $R(3) = 6$ because
  \begin{itemize}
    \item The following coloring shows that $R(3) > 5$. TODO: add picture
    \item If we have $6$ vertices, we can pick a vertex $v$. By pigeonhole, three of the neighbors of $v$ are connected to $v$ via the same color, say red. Now either two of those neighbors are connected with a red edge, in which case they form a red triangle with $v$, or they are connected with blue edges to each other, in which case they form a blue triangle. As a way to remember this proof, we encourage you to watch the following music video: \href{https://youtu.be/vE7MW2lk55E}{Everybody's looking for Ramsey}
  \end{itemize}
\end{eg}

\section{Old bounds on \texorpdfstring{$R(\ell, k)$}{R(l, k)}}

\begin{thm}[Erd\H os-Szekeres, 1935]
  $$R(\ell, k) \le \binom{k + \ell - 2}{\ell - 1}$$
  In particular, $R(\ell, k)$ is well-defined.
\end{thm}

\begin{lem}
  For all $k, \ell \ge 3$,
  $$R(\ell, k) \le \underbrace{R(\ell - 1, k)}_a + \underbrace{R(\ell, k - 1)}_b$$
\end{lem}
\begin{proof}
  Let $n = a + b$. Pick a vertex $v$. By pigeonhole, either
  \begin{itemize}
    \item $v$ has at least $a$ red neighbors. Either these neighbors contain a red $K_{\ell - 1}$ (in which case we chuck $v$ in), or contain a blue $K_k$ (in which case we already won).
    \item $v$ has at least $b$ blue neighbors. Either these neighbors contain a blue $K_{k - 1}$ (in which case we chuck $v$ in), or contain a red $K_\ell$ (in which case we already won).
  \end{itemize}
\end{proof}

\begin{proof}[Proof of Erd\H os-Szekeres]
  Use that $R(\ell, 2) = \ell$ and induct on $k$ and $\ell$.
\end{proof}

\begin{cor}
  $$R(k) \le \binom{2k} k \le C\frac{4^k}{\sqrt k}$$
  for some constant $C$.
\end{cor}

\subsection{Lower bounds}

Can we find edge colorings on many vertices without a monochromatic $K_k$?
Certainly, we can at least do so on $(k - 1)^2$ vertices.

TODO: Insert figure

This polynomial lower bound is eons away from our exponential upper bound. For quite some time (in the 1930s), people thought that the lower bound was closer to the truth than the upper bound. Surprisingly, it is possible to show an exponential lower bound without actually exhibiting such a coloring!

\begin{thm}[Erd\H os, 1948]
  $$R(k) \ge \frac{k - 1}{e\sqrt 2}2^{\frac k 2}$$
\end{thm}
\begin{fact}
  $$\left(\frac n k\right)^k \le \binom n k \le \left(\frac{en}k\right)^k$$
\end{fact}
\begin{proof}
  Let $n = \ceil{\frac{k - 1}{e\sqrt 2}2^{\frac k 2}}$ and $\chi$ be a random red/blue edge coloring of $K_n$ (each edge is independently colored red or blue with probability $\frac 1 2$). We see that
  \begin{align*}
    \P(\chi \text{ contains a monochromatic } K_k)
    & = \P\left(\bigcup_{S \in [n]^{(k)}} \{S \text{ monochromatic}\}\right) \\
    & \le \binom n k \P([k] \text{ monochromatic}) \\
    & = \binom n k 2^{-\binom k 2 + 1} \\
    & \le 2 \left(\frac{en}k\right)^k 2^{-\frac{k(k - 1)}2} \\
    & = 2 \left(\frac{en}k 2^{-\frac{k - 1}2}\right)^k \\
    & \le 2\left(1 - \frac 1 k\right)^k \\
    & < 1
  \end{align*}
  Hence
  $$2^{\frac k 2} \le R(k) \le 4^k$$
\end{proof}

This proof is remarkable by the fact that it proves that the probability of some object existing is high, without actually constructing such an object. In fact it is still an important open problem to explicitly construct a $K_k$-free edge-coloring of $K_n$ with $n$ exponential in $k$. In other words, {\it even though $K_k$-free edge-colorings are abundant, we don't know how to write down a single one}.

\begin{rmk}
  The use of ``constructive'' here is quite different to that in other areas of mathematics. We do not mean that the proof requires the Law of Excluded Middle or the Axiom of Choice, nor that we do not provide an algorithm to find a graph without monochromatic $K_k$. \\
  Since there are only finitely many red/blue edge-colorings of $K_n$ for a fixed $n$, there trivially is an algorithm to find such a coloring: enumerate them all and try them one by one. Less obviously, there is a procedure to systematically remove any use of the axiom of choice from the proofs of most of the results in this course. Excluded Middle is also redundant since the case splits we consider can be decided in finite time (again, everything is finite). \\
  A more careful definition of ``constructive'' here is about complexity of the description of the object: Erd\H os' lower bound does not provide any better {\it deterministic} algorithm than "Try all edge-colorings", and this has complexity $\Omega\left(2^{\binom n 2}\right)$ (without even accounting for the time it takes to check whether a coloring contains a monochromatic $K_k$). In contrast, we would expect a constructive lower bound to yield an edge-coloring in a polynomial number of operations in $n$.
\end{rmk}

\begin{question}
  What's the base of the exponent here? Is there even such a base?
\end{question}

\newlec

We know
$$R(3, k) \le \binom{k + 1}2 \le (k + 1)^2$$

\begin{dfn}
  An {\bf independent set} in a graph is a set of vertices that does not contain an edge. The {\bf independence number} $\alpha(G)$ is the maximum size of an independent set of $G$.
\end{dfn}

\begin{dfn}[Binomial Random Graph]
  For $n \in \N, 0 \le p \le 1$, we define $G(n, p)$ the probability space of graphs where each edge is independently present with probability $p$.
\end{dfn}

\begin{thm}[Erd\H os]
  $$R(3, k) \ge c\left(\frac k{\log k}\right)^{\frac 32}$$
  for some constant $c > 0$.
\end{thm}
\begin{idea}
  We will look at a binomial random graph and choose the parameters so that there are very few red $K_k$ and the number of blue $K_3$ is at most some fixed proportion of $n$. Then we will remove one vertex from each red $K_k$ and one vertex from each blue $K_3$. The resulting graph will have neither and most likely will still contain a fixed proportion of the vertices we started with.
\end{idea}
\begin{proof}
  Change the language. Discuss the blue graph. We are now looking for the maximum number of edges of a graph with no triangles and no independent set of size $k$. \\
  Take $n = \left(\frac k{\log k}\right)^{\frac 32}, p = n^{-\frac 23} = \frac{\log k}k$. Now sample $G \sim G(n, p)$ and define $\tilde G$ to be $G$ with one vertex removed from each triangle and independent set of size $k$. By construction, $K_3 \not\subseteq \tilde G$ and $\alpha(\tilde G) < k$. We will show $\bbE |\tilde G| \ge \frac n2$ using
  $$|\tilde G| \ge n - \#\text{triangles in } G - \#\text{independent sets of size $k$ in } G$$
  First,
  $$\bbE \#\text{triangles in } G
  = \sum_{T \in [n]^(3)} \P(T \text { triangle in } G)
  = \binom n3 p^3 \le \frac{(np)^3}6 = \frac n6$$
  Second,
  \begin{align*}
    \bbE \#\text{independent sets of size $k$ in } G
    & = \binom nk (1 - p)^{\binom k2} \\
    & \le \left(\frac{en}k\right)^k e^{-p\binom k2} \\
    & \sim \left(\frac{en}k e^{-\frac{pk}2}\right)^k \\
    & = \left(\frac{ek^{\frac 32}}{k\log^{\frac 32} k} e^{-\frac{\log k}2}\right)^k \\
    & = \left(\frac e{\log^{\frac 32} k}\right)^k \longrightarrow 0
  \end{align*}
  Hence, for large enough $k$,
  $$\bbE |\tilde G| \ge n - \frac n6 - 1 \ge \frac n2 = \frac 12 \left(\frac k{\log k}\right)^{\frac 32}$$
  By adjusting $c > 0$, we have proved the theorem.
\end{proof}
\begin{rmk}
  The values of $n$ and $p$ come from the constraints
  $$n^3p^3 \ll n, \quad \frac{\log n}k \ll p$$
\end{rmk}

We are being wasteful here. Why throw an entire vertex away when we could get away with removing a single edge? Because we might accidentally create an independent set of size $k$. But we can be smarter...

\begin{idea}
  Take a maximal collection of edge-disjoint triangles in $G \sim G(n, p)$ and remove all edges from these triangles.
\end{idea}

\begin{thm}[Erd\H os]
  $$R(3, k) \ge c\left(\frac k{\log k}\right)^2$$
  for some constant $c > 0$.
\end{thm}

\begin{lem}
  Let $\mathcal F = \{A_1, \dots, A_m\}$ be a family of events in a probability space. Let $\Eps_t$ be the event that $t$ {\it independent} events from $\mathcal F$ occur. then
  $$\P(\Eps_t) \le \frac 1{t!}\left(\sum_{i = 1}^m \P(A_i)\right)^t$$
\end{lem}
\begin{proof}
  Note that
  $$1_{\Eps_t} \le \frac 1{t!} \sum_{\substack{i \in [m]^t \\ A_{i_1}, \dots, A_{i_t}\text{ independent}}} 1_{A_{i_1}} \dots 1_{A_{i_t}}$$
  So
  \begin{align*}
    \P(\Eps_t)
    & \le \frac 1{t!} \sum_{\substack{i \in [m]^t \\ A_{i_1}, \dots, A_{i_t}\text{ independent}}} \P(A_{i_1}) \dots \P(A_{i_t}) \\
    & \le \frac 1{t!} \sum_{i \in [m]^t} \P(A_{i_1}) \dots \P(A_{i_t}) \\
    & = \frac 1{t!} \left(\sum_{i = 1}^m \P(A_i)\right)^t
  \end{align*}
\end{proof}

\newlec

\begin{idea}
  Instead of killing vertices, we will kill edges. We only need to kill edges from a maximal set of edge-disjoint triangles. For this killing to create an independent set $I$ of size $k$, we must have killed all edges in $I$. But with high probability all sets of size $k$ contain a fixed fraction of the expectation $p\binom k2$ of the number of edges, and having so many edge-disjoint triangles each with two vertices among $k$ fixed vertices is unlikely.
\end{idea}

\begin{lem}
  Let $n, k \in \N, p \in [0, 1]$ be such that $pk \ge 16\log n$. Then with high probability every subset of size $k$ of $G \sim G(n, p)$ contains at least $\frac{pk^2}8$ edges.
\end{lem}

\begin{proof}[Proof of Erd\H os' bound]
  We fix $n = \left(\frac{c_1 k}{\log k}\right)^2, p = c_2n^{-\frac 12} = \frac{c_2}{c_1}\frac{\log k}k$. Let $G \sim G(n, p), \mathcal T$ a maximal collection of edge-disjoint triangles in $G$, $\tilde G$ be $G$ with all edges of $\mathcal T$ removed. Note, $\tilde G$ contains no triangle. We show
  $$\P(\alpha(\tilde G) \ge k) < 1$$
  Let $Q$ be the event that every set of $k$ vertices of $G$ contains $\ge \frac{pk^2}8$ edges. Setting $\frac{c_2}{c_1} = 48$, we get
  $$pk = \frac{c_2}{c_1}\frac{\log k}k k = 48\log k > 16\log n$$
  so that $\P(Q) = 1 - o(1)$ by the lemma. Now note that
  $$\P(\alpha(\tilde G) \ge k) \le \P(\alpha(\tilde G) \ge k, Q) + \cancelto 0{\P(Q^c)}$$
  So we focus on $\P(\alpha(\tilde G) \ge k, Q)$. Observe that if $\tilde G$ contains an independent set $I$ of size $k$ and $Q$ holds, then $I$ contains $\ge \frac{pk^2}8$ edges of $G$ by assumption. But $I$ is an independent set in $\tilde G$, so all $\frac{pk^2}8$ edges must have belonged to some triangle $T \in \mathcal T$ and been removed. Therefore
  \begin{align*}
    \P(\alpha(\tilde G) \ge k, Q)
    & \le \P\left(\exists S \in [n]^{(k)}, \mathcal T \text{ meets $S$ in $\ge \frac{pk^2}8$ edges}\right) \\
    & \le \binom n k \P\left(\underbrace{\substack{\text{at least $t$ triangles of $\mathcal T$} \\ \text{meet $[k]$ in at least two vertices}}}_B\right)
  \end{align*}
  where $t = \frac{pk^2}{24}$. Let $\{T_i\}$ be the collection of triangles in $K_n$ that meet $[k]$ in at least two vertices. Let $A_i = \{T_i \subseteq G\}$. Note that if $T_{i_1}, \dots, T_{i_k}$ are edge-disjoint, then $A_{i_1}, \dots, A_{i_k}$ are independent. So
  \begin{align*}
    \P(B)
    & \le \P(\Eps_t) \\
    & \le \frac 1{t!}\left(\sum_{\substack{T_i \subseteq K_n \text{ intersects } [k] \\ \text{ in at least two vertices}}} \P(T_i \subseteq G)\right)^t \\
    & \le \frac 1{t!} (k^2np^3)^t \\
    & \le \left(\frac{ek^2np^3}t\right)^t \\
    & = (24enp^2)^t = (24ec_2^2)^t = e^{-t}
  \end{align*}
  by choosing $c_2 = \frac 1{\sqrt{24} e}$. To finish, observe that
  $$t = \frac{pk^2}{24} = 2k\log k \ge k\log n$$
  Hence
  $$\binom n k \P(B) \le \binom n k e^{-t} \le \left(\frac{en}k e^{-\log n}\right)^k = \left(\frac ek\right)^k \to 0$$
\end{proof}


\subsection{Large deviation inequalities}

Let $Z$ be a gaussian random variable.
$$\P(Z - \bbE Z \ge t) \le e^{\frac{-t}{2\Var Z}}$$
Let $X_1, \dots, X_n$ be iid Bernoulli random variables. We denote this $X_i \sim \Ber(p)$. Write $S_n = X_1 + \dots + X_n$. Note $\bbE S_n = np, \Var(S_n) = np(1 - p)$.

\begin{idea}
  Often, the tail of $S_n$ looks like a gaussian tail.
\end{idea}

\begin{thm}[Chernoff inequality]
  Let $X_1, \dots, X_n \sim \Ber(p)$. Then
  $$\P\left(\abs{S_n - pn} \ge t\right) \le 2\exp\left(\underbrace{-\frac{t^2}{2pn}}_{\text{meat}} + \underbrace{\frac{t^3}{(pn)^2}}_{\text{error term}}\right)$$
\end{thm}

\newlec

\begin{proof}[Proof of the $\frac{pk^2}8$ lemma]
  Using Chernoff on $e(G[[k]])$, namely with $p := p, n := \binom k2, t := \frac{pk^2}4$, we get
  \begin{align*}
    \P(G \text{ fails the statement})
    & = \P\left(\exists S \in [n]^{(k)}, e(G[s]) < \frac{pk^2}8\right) \\
    & \le \binom nk \P\left(e(G[[k]]) < \frac{pk^2}8\right) \\
    & \le \binom nk \P\left(\frac{pk^2}4 < \abs{e(G[[k]]) - p\binom k2}\right) \\
    & \le 2\left(\frac{en}k\right)^k \exp\left(-\frac{pk^2}{16} + \frac 18\right) \\
    & \ll \left(\frac{en}k\right)^k \exp\left(-k\log n\right) \\
    & = \left(\frac ek\right)^k
  \end{align*}
  which tends to 0 as $k$ tends to infinity.
\end{proof}

\subsection{The Local Lemma}

The probabilistic method is like finding the hay in the hay stack. What if we want to find the needle?

\begin{dfn}
  Let $\mathcal F = \{A_1, \dots, A_n\}$ be a family of events in a probability space. A {\bf dependency graph} $\Gamma$ is a graph with vertices $\mathcal F$ such that the event $A_i$ is independent of $\sigma(A_j \mid j \not\sim i)$ for all $i \in [n]$.
\end{dfn}

\begin{rmks}~
  \begin{itemize}
    \item A dependency graph is not unique.
    \item The complete graph is always a dependency graph.
    \item The empty graph is a dependency graph iff the $A_i$ are globally independent.
  \end{itemize}
\end{rmks}

\begin{thm}[The Local Lemma, symmetric version]
  Let $\mathcal F = \{A_1, \dots, A_n\}$ be a family of events in a probability space, let $\Gamma$ be a dependency graph for $\mathcal F$ with maximum degree $\Delta$. If $\P(A_i) \le \frac 1{e(\Delta + 1)}$ for all $i$, then
  $$\P\left(\Inter_i A_i^c\right) > 0$$
\end{thm}

\begin{thm}[Spencer]
  $$R(k) \ge (1 - o(1)) \frac{\sqrt 2k}e 2^{\frac k2}$$
\end{thm}
\begin{proof}
  Let $n = (1 - \eps)\frac{\sqrt 2k}e2^{\frac k2}$ for some $\eps > 0$. Let $\chi$ be a random edge-coloring of $K_n$ uniformly over all colorings. Define, for $S \in [n]^{(k)}$, the event
  $$A_S = \{S \text{ is monochromatic in } \Z\}$$
  Note we want $\P\left(\Inter_{S \in [n]^{(k)}} A_S^c\right) > 0$. Define the dependency graph $\Gamma$ by
  $$S \sim T \iff 1 < \abs{S \inter T} < k$$
  The maximum degree of $\Gamma$ Is
  $$\Delta = \sum_{t = 2}^{k - 1}\binom kt\binom{n - k}{k - t} = \binom nk - k\binom{n - k}{k - 1} - \binom{n - k}k - 1$$
  To apply the Local Lemma, we just check
  $$\P(A_S) = 2^{-\binom k2 + 1} \le \frac 1{e(\Delta + 1)}$$
\end{proof}

\begin{thm}[Lopsided Local Lemma]
  Let $\mathcal F = \{A_1, \dots, A_n\}$ be a family of events on a probability space, $\Gamma$ a dependency graph for $\mathcal F$, $0 \le x_1, \dots, x_n < 1$ satisfying
  $$\P(A_i) \le x_i \prod_{j \sim i}(1 - x_j)$$
  Then
  $$\P\left(\Inter_i A_i^c\right) \ge \prod_i (1 - x_i) > 0$$
\end{thm}

\begin{thm}[Erd\H os, 1961]
  $$R(3, k) \ge c\left(\frac k{\log k}\right)^2$$
  for some $c > 0$.
\end{thm}
\begin{proof}
  Let $n = \eps^4\left(\frac k{\log k}\right)^2, p = \frac\eps{\sqrt n} = \frac{\log k}{\eps k}, G \sim G(n, p)$. For all $T \in [n]^{(3)}$ and $I \in [n]^{(k)}$, define $A_T = \{T \subseteq G\}$ and $B_I = \{I \subseteq G^c\}$. We want
  $$\P\left(\Inter_{T \in [n]^{(3)}} A_T^c \inter \Inter_{I \in [n]^{(k)}} B_I^c\right) > 0$$
  Define the dependency graph $\Gamma$ by
  \begin{align*}
    T \sim T' & \iff \abs{T \inter T'} = 2 \\
    T \sim I & \iff 2 \le \abs{I \inter T} \\
    I \sim I' & \iff 2 \le \abs{I \inter I'} < k
  \end{align*}
  We see that
  \begin{align*}
    \#\{T' \in [n]^{(3)} \mid T' \sim T\} \le 3n & , \#\{I \in [n]^{(k)} \mid I \sim T\} \le 3n^{k - 2} \\
    \#\{T \in [n]^{(3)} \mid T \sim I\} \le k^2n & , \#\{I' \in [n]^{(k)} \mid I' \sim I\} \le k^2n^{k - 2}
  \end{align*}
  We therefore check that
  \begin{enumerate}
    \item $\P(A_T) \le x_T \prod_{T' \sim T} (1 - x_{T'}) \prod_{I \sim T} (1 - x_I)$. Indeed,
      \begin{align*}
        \lhs & = p^3 \\
        \rhs & \ge 3p^3 (1 - 3p^3)^{3n}(1 - n^{-k})^{3n^{k - 2}}
      \end{align*}
      and, using $1 - x \ge e^{-2x}$ for small enough $x$,
      $$(1 - 3p^3)^{3n}(1 - n^{-k})^{3n^{k - 2}} \ge \exp(-18p^3n - 6n^{-2}) = \exp(-18\eps^2 p - 6n^{-2}) \to 1$$

\newlec

    \item $\P(B_I) \le x_I \prod_{T \sim I}(1 - x_T) \prod_{I' \sim I} (1 - x_{I'})$. Indeed,
      \begin{align*}
        \lhs & = (1 - p)^{\binom k2} \\
        \rhs & = n^{-k}(1 - n^{-k})^{k^2n^{k - 2}}(1 - 3p^3)^{k^2n}
      \end{align*}
      and, using $1 - x \ge e^{-2x}$ for small enough $x$,
      \begin{align*}
        \log\rhs
        & \ge -k\log n - 2k^2n^{-2} - 6k^2p^3n \\
        & \ge -2k\log k - 6k^2p\eps^2 + o(1) \\
        & \ge -\frac{k\log k}{4\eps} + o(1) \\
        & \ge -p\binom k2 \\
        & \ge \log\lhs
      \end{align*}
  \end{enumerate}
\end{proof}

\begin{proof}[Proof of the Local Lemma]
  Applying $\P(A \inter B) = \P(A) \P(B \mid A)$ repeatedly, write
  \begin{align*}
    \P\left(\Inter_{i = 1}^n A_i^c\right)
    & = \prod_{i = 1}^n \P(A_i^c \mid A_1^c \inter \dots \inter A_{i - 1}^c) \\
    & = \prod_{i = 1}^n (1 - \P(A_i \mid A_1^c \inter \dots \inter A_{i - 1}^c))
  \end{align*}
  It is enough to show that $\P(A_i \mid A_1^c \inter \dots \inter A_{i - 1}^c) \le x_i$. We prove
  $$\P(A_i \mid \Inter_{j \in S} A_j^c) \le x_i$$
  for all $S \subseteq [n]$ by induction:
  \begin{itemize}
    \item $S = \emptyset$. Done by assumption.
    \item Write
    $$I = \Inter_{\substack{j \in S \\ j \not\sim i}} A_j^c, D = \Inter_{\substack{j \in S \\ j \sim i}} A_j^c$$
    So
    $$\P(A_i \mid I \inter D) = \frac{\P(A_i \inter I \inter D)}{\P(I \inter D)} \le \frac{\P(A_i \inter I)}{\P(I \inter D)} = \frac{\P(A_i)\P(I)}{\P(I \inter D)} = \frac{\P(A_i)}{\P(D \mid I)}$$
    Now, write $D = A_{i_1}^c \inter \dots \inter A_{i_m}^c$ and
    $$\P(D \mid I) = \prod_{j = 1}^m (1 - \P(A_{i_j} \mid I \inter A_{i_1}^c \inter \dots \inter A_{i_{j - 1}}^c)) \ge \prod_{j = 1}^m (1 - x_{i_j}) \ge \prod_{j \sim i} (1 - x_j)$$
  \end{itemize}
\end{proof}

\begin{thm}[Lovasz Local Lemma, symmetric version]
  For $\mathcal F = \{A_1, \dots, A_n\}$, $\Gamma$ a dependency graph with maximum degree $\Delta$, if $\P(A_i) \le \frac 1{e(\Delta + 1)}$ for all $i$, then
  $$\P\left(\Inter_i A_i^c\right) \ge \left(1 - \frac 1{e(\Delta + 1)}\right)^n > 0$$
\end{thm}
\begin{proof}
  We use the Lopsided Local Lemma with $x_i = \frac 1{e(\Delta + 1)}$. Note
  $$x_i \prod_{j \sim i} (1 - x_j) \ge \frac 1{\Delta + 1}\left(1 - \frac 1{\Delta + 1}\right)^\Delta \ge \frac 1{e(\Delta + 1)} \ge \P(A_i)$$
  So Lopsided Local Lemma applies.
\end{proof}

We now know
$$\frac{ck^2}{(\log k)^2} \le R(3, k) \le (k + 1)^2$$

\begin{thm}[State of the art on $R(3, k)$]
  $$\underbrace{\left(\frac 14 + o(1)\right)\frac{k^2}{\log k}}_{\substack{\text{Fiz Pontiveros} \\ \text{Griffiths} \\ \text{Morris}} + \substack{\text{Bohman} \\ \text{Keevash}}} \le R(3, k) \le \underbrace{(1 + o(1))\frac{k^2}{\log k}}_{\substack{\text{Ajtai} \\ \text{Komlós} \\ \text{Szemerédi}} + \text{Shearer}}$$
\end{thm}

\clearpage

\subsection{Upper bounds on \texorpdfstring{$R(3, k)$}{R(3, k)}}

\newlec

\begin{thm}[Ajtai, Komlós, Szemerédi]
  $$R(3, k) \le c \frac{k^2}{\log k}$$
  for some $c > 0$. In fact, we will see $c = 1 + o(1)$.
\end{thm}

\begin{thm}[Ajtai, Komlós, Szemerédi]
  Let $G$ be a triangle-free graph on $n$ vertices with maximum degree $\Delta$. Then
  $$\alpha(G) \ge c \frac n\Delta \log\Delta$$
  for some absolute constant $c > 0$.
\end{thm}

\begin{rmk}
  For general graphs, we know $\alpha(G) \ge \frac n{\chi(G)} \ge \frac n{\Delta + 1}$ by the naïve greedy algorithm and this is basically best possible. The extra $\log d$ factor will come from tracking how sparse our graph is becoming as we remove vertices from it.
\end{rmk}

We apply a random greedy algorithm to prove the following theorem due to Shearer.

Define
$$f(x) = \frac{x \log x - x + 1}{(x - 1)^2}$$
extending continuously to $[0, 1]$ by $f(0) = 1, f(1) = \frac 12$. We remark that
\begin{itemize}
  \item $f$ is continuous and differentiable.
  \item $f$ is antitone and convex.
  \item $0 < f(x) < 1$
  \item $(x + 1)f(x) = 1 + (x - x^2)f(x)$
\end{itemize}


\begin{thm}[Shearer]
  Let $G$ be a triangle-free graph on $n$ vertices with average degree $d > 0$.
  Then
  $$\alpha(G) \ge nf(d) \ge (1 - o(1)) \frac n\Delta \log\Delta$$
\end{thm}
\begin{proof}
  Induction on $n$.

  Let $x$ a vertex to be chosen later and $G' := G - x - N(x)$. Writing $d'$ the average degree of $G'$ and applying induction to $G'$, we get
  \begin{align*}
    \alpha(G)
    & \ge 1 + \alpha(G') \\
    & \ge 1 + (n - \deg x - 1)f(d') \\
    & \ge 1 + (n - \deg x - 1)(f(d) + (d' - d)f'(d)) \\
    & = nf(d) + 1 - (\deg x + 1)f(d) + (d\deg x + d + n(d' - d) - d'\deg x - d')f'(d) \\
    & = nf(d) + 1 - (\deg x + 1)f(d) + \left(d\deg x + d - 2\sum_{y \sim x} \deg y\right)f'(d)
  \end{align*}
  where we used that $N(x)$ is an independent set to get that
  \begin{align*}
    n(d - d')
    & = n \left(\frac{2e(G)}n - \frac{2e(G')}{n - \deg x - 1}\right) \\
    & = 2(e(G) - e(G')) - (\deg x + 1)d' \\
    & = 2\sum_{y \sim x} \deg y - (\deg x + 1)d'
  \end{align*}
  So we want to choose $x$ such that
  $$(\deg x + 1)f(d) \le 1 + (d\deg x + d - 2\sum_{y \sim x} \deg y)f'(d)$$
  We average over $x$:
  \begin{align*}
    \E_x \lhs & = (d + 1)f(d) \\
    \E_x \rhs & = 1 + (d^2 + d - 2\E_x\sum_{y \sim x} \deg y)f'(d)
  \end{align*}
  Notice that
  $$\E_x\sum_{y \sim x} \deg y = \frac 1n \sum_x\sum_{y \sim x} \deg y = \frac 1n \sum_x \deg x^2 \ge \left(\frac 1n\sum_x \deg x\right)^2$$
  Since $f'(d) \le 0$, we get
  $$\E_x \rhs \ge 1 + (d^2 + d - 2d^2)f'(d) = 1 + (d - d^2)f'(d) = (d + 1)f(d) = \E_x \lhs$$
  So such $x$ exists.
\end{proof}

\begin{proof}[Proof of the AKS bound using Shearer]
  Let $G$ be a graph on $n$ vertices with neither a triangle nor an independent set of size $k$. We have
  $$d \le \Delta \le \alpha(G) < k$$
  where the middle inequality holds by triangle freeness. Hence Shearer says
  $$k > \alpha(G) \ge \frac nd \log d \ge \frac nk \log k$$
  And $n \le \frac{k^2}{\log k}$ as wanted.
\end{proof}

\newlec

\begin{proof}[Second proof of AKS]
  Let $I$ be an independent set in $G$ sampled uniformly among all independent sets of $G$. We will show
  $$\bbE\abs I \ge c\frac nd\log n$$
  Let $v$ be a vertex. We define the random variable
  $$X_v = d1_{v \in I} + \abs{N(v) \inter I}$$
  For any independent set $I$,
  $$\sum_v X_v \le 2d\abs I$$
  So
  $$\sum_v \E_I X_v \le 2d \E_I \abs I$$
  So we want to show that $\E_I X_v \ge c \log d$ for all $v$. \\
  Let $G' = G - v - N(v)$, find $J$ an independent set in $G'$ minimising $\E_I[X_v \mid I \setminus (N(v) \union \{v\}) = J]$ and let $F = \{w \in N(v) \mid N(w) \inter J = \emptyset\}$ and $t = \abs F$. Note carefully that $I \inter (F \union \{v\})$ is uniform over all independent sets in $G[F \union {v}]$, and that the independent sets of $G[F \union {v}]$ are exactly $\{v\}$ and the $2^t$ subsets of $F$. Hence
  $$\E_I X_v \ge \E_{I \setminus (N(v) \union \{v\}) = J} X_v = \E_{I \subseteq F \union \{v\}} X_v = \frac 1{2^t + 1}d + \frac{2^t}{2^t + 1}\frac t2 \ge c \log d$$
  for some $c > 0$ by optimising over $t$.
\end{proof}

\begin{thm}[Ajtai-Komlós-Szemerédi]
  Let $\ell \in \N$. Then for sufficiently large $k \in \N$ we have
  $$R(\ell, k) \le \left(\frac 4{\log k}\right)^{\ell - 2}k^{\ell - 1}$$
\end{thm}

We have seen that the power of $k$ is correct for $\ell = 3$. As of recently, we also know that it is correct for $\ell = 4$.

\clearpage

\section{3-uniform hypergraph Ramsey numbers}

We define $K_n^{(r)}$ to be the {\bf complete $r$-uniform hypergraph} on $n$ vertices. The {\bf $r$-uniform hypergraph Ramsey number} $R^{(r)}(\ell, k)$ is the minimal $n$ such that every edge coloring of $K_n^{(r)}$ contains either a blue $K_\ell^{(r)}$ or a red $K_k^{(r)}$. As before, the {\bf hypergraph diagonal Ramsey number} is $R^{(r)}(k) = R^{(r)}(k, k)$.

\begin{thm}[Erd\H os-Rado]
  $$R^{(3)}(\ell, k) \le 2^{\binom{R(\ell - 1, k - 1)}2}$$
  Eg $R^{(3)}(k) \le 2^{16^k}$.
\end{thm}
\begin{proof}
  Let $t = R(\ell - 1, k - 1)$ and $n = 2^{\binom t2}$. Let $\chi$ be a red/blue edge coloring of $K_n^{(3)}$. Let $v_1, v_2 \in [n]$. Define
  $$A_{1, 2} = \{w \in [n] \mid \chi({v_1, v_2, w}) = c_{1, 2}\}$$
  where $c_{1, 2}$ is the {\bf majority color}, chosen so that
  $$\abs{A_{1, 2}} \ge \frac n2 - 1$$
  Let $v_3 \in A_{1, 2}$. Define
  $$A_{1, 3} = \{w \in A_{1, 2} \mid \chi({v_1, v_3, w}) = c_{1, 3}\}$$
  where $c_{1, 3}$ is the majority color. Now define
  $$A_{2, 3} = \{w \in A_{1, 3} \mid \chi({v_2, v_3, w}) = c_{2, 3}\}$$
  where $c_{2, 3}$ is the majority color, and so on... \\
  After $t$ steps, our world has size $\abs{A_{t - 1, t}} \ge n2^{-\binom t2} \ge 1$. We thus have $\{v_1, \dots, v_t\}$ such that $\chi(\{v_i, v_j, v_k\}) = c_{i, j}$ if $k > i, j$. $c$ is a coloring of $\{v_1, \dots, v_t\}^{(2)}$. By definition of Ramsey, we're done.
\end{proof}

\clearpage

\subsection{Off-diagonal}

\newlec

Erd\H os-Rado gives
$$R^{(3)}(4, k) \le 2^{ck^4}$$

\begin{thm}[Conlon, Fox, Sudakov, 2010]
  $$R^{(3)}(4, k) \le k^{ck^2}$$
\end{thm}

Erd\H os-Rado makes us shrink our world by a factor of $2$ at every query. Can we ask fewer questions? This suggests the following game.

\begin{dfn}[The Ramsey game]
  At each question, we expose a new vertex $v_i$ and get to ask our adversary to expose the color of a collection of edges $v_j v_i$ where $j < i$. Our goal is to force a blue $K_3$ or a red $K_k$ with as few queries as possible.
\end{dfn}

\begin{lem}
  In the Ramsey game, we can force a blue $K_3$ or a red $K_k$ in at most $2k^3$ queries whose answer is ``red'' and $k^2$ queries whose answer is ``blue''.
\end{lem}
\begin{proof}
  As we expose vertices, we sort them into {\bf levels} $1, 2, 3, \dots$ The first vertex at level $i$ is called the {\bf root} of level $i$ and denoted $r_i$.

  We start by putting $v_1$ into level $1$ and setting $r_1 := v_1$. When we expose vertex $v_i$, we ask for the color of $v_ir_1, \dots, v_ir_r$ until we get replied ``blue''.
  \begin{itemize}
    \item If we get a blue response to $v_ir_j$ for some $j$, stick $v_i$ in level $j$ and expose all edges to previous vertices of level $j$.
    \item If all $v_ir_j$ get replied ``red'', make $v_i$ a new root.
  \end{itemize}
  TODO: Insert picture

  Assuming we have not encountered a blue $K_3$ or red $K_k$, every level contains at most $k$ vertices and there are at most $k$ levels. We have exposed at most $k^2$ red edges and $k$ blue edges in each level, and $k^3$ red edges between levels, so in total at most $2k^3$ red edges and $k^2$ blue edges.
\end{proof}

The idea now is that our adversary wants to reply ``red'' most of the time, so we're willing to shrink our world much more when they reply ``blue''.

\begin{proof}[Proof that $R^{(3)}(4, k) \le k^{ck^2}$]
  The proof follows the proof of Erd\H os-Rado but we now only refine our world based on the pairs that we query in the Ramsey game. We also have a different rule about when to refine our world to be blue vs red.

  Start with $A_0 = [n]$ where $n = k^{ck^2}$ and define
  $$A_0 \supseteq A_1 \supseteq A_2 \supseteq \dots \supseteq A_n$$
  where $e_j$ is the edge coming from the Ramsey game and
  \begin{align*}
    A_j^B & = \{x \in A_j \mid \chi(e_{j + 1} \union \{x\}) = \text{blue}\} \\
    A_j^R & = \{x \in A_j \mid \chi(e_{j + 1} \union \{x\}) = \text{red}\} \\
    A_{j + 1} & =
    \begin{cases}
      A_j^B & \text{ if } \abs{A_j^B} \ge \frac 1k \abs{A_j} \\
      A_j^R & \text{ if } \abs{A_j^R} \ge \left(1 - \frac 1k\right) \abs{A_j}
    \end{cases}
  \end{align*}
  At the end of time,
  $$\abs{A_m} \ge n\left(\frac 1k\right)^{k^2}\left(1 - \frac 1k\right)^{2k^3} \ge k^{(c - 1)k^2}e^{-4k^2} \ge 1$$
  if we pick $c$ large enough.
\end{proof}

What about lower bounds? Let's try the probabilistic method.

Color triples blue with probability $p$.
$$\bbE\left[\#\text{red }K_k^{(3)}\right] = \binom nk (1 - p)^{\binom k3} \le \left(\frac{en}k\right)^k e^{-p\binom k3} = \left(\frac{en}k e^{-\frac{pk^2}6}\right)^k$$
This is nontrivial if $p \gg \frac 1{k^2}$. Then
$$\bbE\left[\#\text{blue }K_4^{(3)}\right] = \binom n4 p^4 \ge \left(\frac n4\right)^4 \gg \left(\frac n{k^2}\right)^4$$
So the (naïve) probabilistic approach looks useless for anything better than polynomial in $k$.

\newlec

\begin{thm}
  $$R^{(3)}(4, k) \ge 2^{\frac{k - 1}2}$$
\end{thm}
\begin{proof}
  Let $n = 2^{\frac{k - 1}2} - 1$ and $T$ a random tournament on $[n]$. For $x, y \in [n]$ distinct, define
  $$\chi(\{x, y, z\}) =
  \begin{cases}
    \text{blue} & \text{if $x, y, z$ oriented} \\
    \text{red} & \text{if $x, y, z$ acyclic}
  \end{cases}$$
  TODO: Insert oriented and acyclic pictures
  \begin{obs}
    A tournament on $4$ points has at least one transitive triple.
  \end{obs}
  This immediately implies there is no blue $K_4^{(3)}$.
  \begin{obs}
    If $K$ is a tournament where every triple is transitive, then $K$ itself is transitive.
  \end{obs}
  Hence
  $$\bbE \#\text{red } K_k^{(3)} = \bbE \#\text{transitive } K_k = \binom nk k! 2^{-\binom k2} \le n^k 2^{-\binom k2} = \left(n2^{-\frac{k - 1}2}\right)^k < 1$$
  Hence there exists a tournament with no red $K_k^{(3)}$.
\end{proof}

\begin{thm}[Conlon, Fox, Sudakov, 2010]
  $$R^{(3)}(4, k) \ge k^{\frac k5}$$
  for sufficiently large $k$.
\end{thm}
\begin{proof}[Proof (Stepping up)]
  Set $n = k^{\frac k5}, r = R(3, \frac k4) - 1$. Let $\theta$ be an edge coloring of $K_r$ with no blue $K_3$ or red $K_{\frac k4}$. Now let $\sigma$ be a random edge coloring in $r$ colors. For $x < y < z$, define
  $$\chi(\{x, y, z\}) =
  \begin{cases}
    \theta(\{\sigma(xy), \sigma(xz)\}) & \text{ if } \sigma(xy) \ne \sigma(xz) \\
    \text{ red} & \text{ if } \sigma(xy) = \sigma(xz)
  \end{cases}$$
  \begin{idea}
    The fact that $\theta$ has no blue triangle will imply that $\chi$ has no blue $K_4^{(3)}$. The fact that $\theta$ has no red $K_{\frac k4}$ will imply that $\chi$ has no red $K_k^{(3)}$ with high probability.
  \end{idea}
  Assume that $x < y_1 < y_2 < y_3$ form a blue $K_4^{(3)}$. Note that $\sigma(xy_1)$, $\sigma(xy_2)$, $\sigma(xy_3)$ are distinct. So $\sigma(xy_1)\sigma(xy_2), \sigma(xy_2)\sigma(xy_3), \sigma(xy_3)\sigma(xy_1)$ are the edges of a triangle in $K_r$ which is blue in $\theta$. Contradiction.
  
  TODO: Insert figure

  Now assume that $K = \{x_1, \dots, x_k\}$ is a red $K_k^{(3)}$ in $\chi$. For each $i \in [k]$,
  $$\abs{\{\sigma(x_ix_j) \mid j > i\}} < \frac k4$$
  as otherwise one can find $i < j_1 < \dots < j_{\frac k4}$ such that $\sigma(x_ix_{j_1}), \dots, \sigma(x_ix_{j_{\frac k4}})$ are all distinct, meaning that they are vertices of a red $K_{\frac k4}$ in $\theta$.
  
  TODO: Insert figure

  Call such a set $K \in [n]^{[k]}$ {\bf sad}. We consider
  \begin{align*}
    \bbE \#\text{ sad sets}
    & \le n^k \prod_{i = 1}^k \binom r{\frac k4} \left(\frac k{4r}\right)^{k - i} \\
    & = n^k \binom r{\frac k4}^k \left(\frac k{4r}\right)^{\sum_{i = 1}^k k - i} \\
    & \le n^k \left(\frac{4er}k\right)^{\frac{k^2}4} \left(\frac k{4r}\right)^{\frac{k^2}4} \\
    & = \left(n\left(\frac e4 \frac kr\right)^{\frac k4}\right)^k \\
    & \le \left(nk^{-\frac k4 + o(k)}\right)^k \text{ since } r > k^{2 - o(1)} \\
    & < \left(nk^{-\frac k5}\right)^k \\
    & = 1 \text{ since } n = k^{\frac k5}
  \end{align*}
  Hence find some $\sigma$ such that no set is sad. We're done.
\end{proof}


\subsection{Diagonal}

\newlec

The state of the art for $R^{(3)}(k)$ is
$$2^{ck^2} \le R^{(3)}(k) \le 2^{2^{2k + 1}}$$
for some $c > 0$.

\begin{conj}[Erd\H os, Hajnal, Rado, 1965]
  $$R^{(3)}(k) \ge 2^{2^{ck}}$$
  for some $c > 0$.
\end{conj}

\begin{thm}[Erd\H os, Hajnal, 1980]
  $$\underbrace{R_4^{(3)}(k)}_{4\text{ colors}} \ge 2^{2^{ck}}$$
  for some $c > 0$.
\end{thm}

This is an example of ``stepping up''.

\begin{lem}
  For all $k$,
  $$R_4^{(3)}(k) \ge 2^{R(k - 1) - 1}$$
\end{lem}
\begin{proof}
  Let $r = R(k - 1) - 1$ and let $\theta$ be a red-blue edge coloring with no monochromatic $K_{k - 1}$. We now color triples on the ground set $\{0, 1\}^r$. For $x, y \in \{0, 1\}^r, x \ne y$, define
  $$f(x, y) = \max \{i \mid x_i \ne y_i\}$$
  so that
  $$x < y \iff x_{f(x, y)} < y_{f(x, y)}$$
  This is the reverse lexicographic order. Note that $f(x, y) \ne f(y, z)$ if $x < y < z$. We now define our coloring $\xi$ of $\{x, y, z\}, x < y < z$ to be one of four colors depending on which of the following hold:
  $$f(x, y) < f(y, z), \quad \theta(f(x, y), f(y, z)) = \text{red}$$
  Let $x_1 < \dots < x_k$ be the vertices of a monochromatic $K_k^{(3)}$, WLOG of the color where both conditions hold. We claim that $f(x_1, x_2) < f(x_2, x_3) < \dots f(x_{k - 1}, x_k)$ are the vertices of a monochromatic, in fact red, $K_{k - 1}$ in $\theta$. Indeed, since the $x_i$ are increasing,
  $$f(x_{i + 1}, x_{j + 1}) = \max_{i < \ell \le j} f(x_\ell, x_{\ell + 1}) = f(x_j, x_{j + 1})$$
  So
  $$\theta(f(x_i, x_{i + 1}), f(x_j, x_{j + 1})) = \theta(f(x_i, x_{i + 1}), f(x_{i + 1}, x_{j + 1})) = \text{red}$$
\end{proof}

\begin{rmk}
  We know a version of this stepping up (due to Erd\H os-Hajnal) for 2-colorings when the uniformity is $\ge 4$.
\end{rmk}

\begin{thm}[Conlon, Fox, Sudakov, 2010]
  $$R^{4}(2k + 1) \ge 2^{R^{(3)}(k - 1) - 1}$$
  and for $s \ge 4$ we have
  $$R^{(s + 1)}(k + 1) \ge 2^{R^{(s)}(k) - 1}$$
\end{thm}

The state of the art for $R^{(3)}(k)$ is
$$\underbrace{2^{\dots^{2^{c_0 k}}}}_{s - 2} \le R^{(s)}(k) \le \underbrace{2^{\dots^{2^{c_1 k}}}}_{s - 1}$$


\section{The Szemerédi Regularity Lemma}

{\bf Informal Statement} \\
For all $\eps > 0$, there exist numbers $\ell(\eps), L(\eps)$ such that every graph can be partitioned into $V_1, \dots, V_k$, where $\ell(\eps) \le k \le L(\eps)$, $\abs{\abs{V_i} - \abs{V_j}} \le 1$ such that, for all but $\eps\binom k2$ pairs $(i, j)$, the graph between $V_i$ and $V_j$ is ``random-like'' up to some coarseness $\eps$.

\begin{dfn}
  In a graph $G$, let $X, Y$ be disjoint sets of vertices. We say that $(X, Y)$ is {\bf $\eps$-uniform} (aka {\bf $\eps$-regular}) if for all $X' \subseteq X, Y' \subseteq Y$ such that $\abs{X'} \ge \eps\abs X, \abs{Y'} \ge \eps\abs Y$ we have
  $$\abs{d(X', Y') - d(X, Y)} < \eps$$
  where
  $$d(X', Y') = \frac{e(X, Y)}{\abs X\abs Y}$$
  is the {\bf density} of edges.
\end{dfn}

\begin{rmk}
  We need some lower bound on $\abs{X'}, \abs{Y'}$ in the definition, else uniformity will basically never hold.
\end{rmk}

\newlec

\begin{prop}
  For $\eps > 0$, let $(X, Y)$ be an $\eps$-regular pair in a graph $G$. Let $p = d(X, Y)$. Then
  \begin{align*}
    \#\{x \in X \mid \abs{N(x) \inter Y} < (p - \eps)\abs Y\} & < \eps\abs X \\
    \#\{x \in X \mid \abs{N(x) \inter Y} > (p + \eps)\abs Y\} & < \eps\abs X
  \end{align*}
\end{prop}
\begin{proof}
  Let $X' = \{x \in X \mid \abs{N(x) \inter Y} < (p - \eps)\abs Y\}$. By definition of $X'$,
  $$d(X', Y) = \frac{e(X', Y)}{\abs X\abs Y} < \frac{(p - \eps)\abs Y\abs{X'}}{\abs{X'}\abs Y} = p - \eps$$
  So $\abs{X'} < \eps\abs X$ by definition of $\eps$-uniformity.
\end{proof}

\begin{lem}[Embedding lemma for triangles]
  Let $\eps > 0$ and $p \ge 2\eps$. Let $G$ be a graph on $V = V_1 \union V_2 \union V_3$ where $V_1, V_2, V_3$ are disjoint of size $m \ge 1$, $(V_i, V_j)$ are $\eps$-uniform for $i \ne j$ and $d(V_i, V_j) \ge p \ge 2\eps$. Then there are at least
  $$(1 - 2\eps)(p - \eps)^3m^3$$
  triangles in $G$.
\end{lem}
\begin{proof}
  We look at triangles with one vertex in each $V_i$. The number of $x \in V_1$ such that
  $$\abs{N(x) \inter V_2} \ge (p - \eps)m, \quad \abs{N(x) \inter V_3} \ge (p - \eps)m$$
  is at least $(1 - 2\eps)m$. For each such $x$, the number of $V_1, V_2, V_3$ triangles containing $x$ is at least
  $$e(N(x) \inter V_2, N(x) \inter V_3) \ge (p - \eps)\abs{N(x) \inter V_2} \abs{N(x) \inter V_3} \ge (p - \eps)^3m^2$$
  since $\abs{N(x) \inter V_2}, \abs{N(x) \inter V_3} \ge (p - \eps)m \ge \eps m$. So we get $(1 - 2\eps)(p - \eps)^3m^3$ triangles in total.

  TODO: Insert figure
\end{proof}

\begin{dfn}
  We say a partition $V = V_1 \union \dots \union V_k$ is an {\bf equipartition} if $\abs{\abs{V_i} - \abs{V_j}} \le 1$ for all $i, j$. Such a partition is {\bf $\eps$-uniform} if $(V_i, V_j)$ is $\eps$-uniform for all but $\eps\binom k2$ pairs $\{i, j\}$.
\end{dfn}

\begin{thm}[Szemerédi Regularity Lemma]
  For all $\eps > 0, \ell \in \N$, there exists $L \in \N$ such that every graph has an $\eps$-uniform equipartition in $k \in [\ell, L]$ parts.
\end{thm}
\begin{rmks}~
  \begin{itemize}
    \item It is really important that $L$ does not depend on $\abs G$.
    \item This does not directly say anything about graphs with $e(G) = o(n^2)$
    \item The proof of the regularity lemma gives
    $$L \le \underbrace{2^{\dots^2}}_{\eps^{-5}}$$
    \item Gowers showed that
    $$L \ge \underbrace{2^{\dots^2}}_{\eps^{-1/16}}$$
    is needed.
  \end{itemize}
\end{rmks}

\begin{lem}[Triangle removal lemma]
  For all $\eps > 0$, there exists $\delta > 0$ such that every graph with at most $\delta n^3$ triangles contains $\eps n^2$ edges that together kill all triangles.
\end{lem}
\begin{proof}
  Let $\eps' = \eps/4, \ell = 10\eps^{-1}, L = L(\eps, \ell), \delta = 2^{-16}\eps^4L^{-3}$. By Szemerédi Regularity Lemma, find a $\eps'$-uniform equipartition
  $$V = V_1 \union \dots \union V_k$$
  where $\ell \le k \le L$. We say $(i, j)$ is {\bf bad} if either of the following holds:
  \begin{enumerate}
    \item $d(V_i, V_j) \le \eps/2$
    \item $(V_i, V_j)$ is not $\eps'$-uniform
    \item $i = j$
  \end{enumerate}
  Now define
  $$T = \{xy \in E(G) \mid x \in V_i, y \in V_j, (i, j) \text{ bad}\}$$
  Then
  \begin{align*}
    \abs T \le
    & \#\text{edges between pairs with } d(V_i, V_j) \le \eps/2 \\
    & + \#\text{edges between $\eps'$-non-uniform pairs} \\
    & + \sum_i \#\text{edges inside } V_i \\
    \le{} & \frac\eps 2 \binom k2 \left(\frac nk\right)^2 + \frac\eps 4 \left(\frac nk\right)^2 \binom k2 + k\left(\frac nk\right)^2 \\
    \le{} & \frac\eps 4 n^2 + \frac\eps 8 n^2 + \frac{n^2}k \\
    \le{} & \left(\frac 14 + \frac 18 + \frac 1{10}\right)\eps n^2 \\
    \le{} & \eps n^2
  \end{align*}
  Now, $G - T$ is triangle-free. Indeed if $x, y, z$ is a triangle then there must be $a, b, c$ distinct such that $(V_a, V_b), (V_b, V_c), (V_c, V_a)$ are $\eps/4$-uniform and
  $$d(V_a, V_b), d(V_b, V_c), d(V_c, V_a) \ge \frac\eps 2$$
  So the triangle embedding lemma tells us that there are at least
  $$\left(1 - 2 \frac\eps 4\right) \left(\frac\eps 4\right)^2 \left(\frac nk\right)^3 \ge \left(1 - \frac\eps 2\right)2^{-6}L^{-3}\eps^3n^3 > \delta n^3$$
  triangles. Contradiction.
\end{proof}

\newlec

\begin{thm}[Roth's theorem]
  For $\eps > 0$, there exists $N$ such that for all $n \ge N$ any $A \subseteq [n]$ of density at least $\eps$ contains a non-trivial arithmetic progression of length three.
\end{thm}
\begin{proof}
  Assume $\eps > 0$. Let $\delta$ be the $\delta$ from the triangle removal lemma applied to $3\eps$, and set $N = 3\delta^{-1}$. Define a tripartite graph on $[3n], [3n], [3n]$ by declaring that $(x, x + a, x + 2a)$ are the edges of a triangle for each $x \in [3n]$ and $a \in A$. We call such a triangle {\bf explicit}. There are $3n\abs A \ge 3\eps n^2$ explicit triangles in our graph and they are edge-disjoint. Therefore triangle removal tells us that there are at least $\delta n^3 = 3N^{-1}n^3 > 3n^2 \ge 3n\abs A$ triangles in our graph. Hence there must be some triangle that's not explicit. But a non-explicit triangle $(x, x + a, x + a + b) = (x, x + a, x + 2c)$ exactly corresponds to a non-trivial arithmetic progression.
\end{proof}

\begin{thm}[Tur\' an]
  Let $G$ be a $K_{r + 1}$-free graph. Then
  $$e(G) \le \left(1 - \frac 1r\right)\frac{n^2}2$$
  and this bound is sharp for the Tur\' an graph.
\end{thm}

But the Tur\' an graph is a bit pathological in the sense that it has huge independence number $\alpha = \frac nr$. What if we require $\alpha(G) = o(n)$?

For $r = 2$, the neighborhood of any vertex is an independent set, so $\deg v = o(n)$ and $e(G) = o(n^2)$. For $r = 3$, we get an interesting problem.

\begin{thm}[Szemerédi]\label{thm:k4-free-small-indep}
  For $\eps > 0$, there exists $\delta > 0$ such that any $K_4$-free graph $G$ on $n$ vertices with $\alpha(G) < \delta n$ has
  $$e(G) \le \frac{n^2}8 + \eps n^2$$
  In other words, if $G$ is $K_4$-free and $\alpha(G) = o(n)$, then $e(G) \le \frac{n^2}8 + o(n^2)$.
\end{thm}
\begin{rmks}
  The $\frac 18$ is sharp!
\end{rmks}

\begin{dfn}
  For $0 < \alpha < 1, \eps > 0$ and a partition
  $$V(G) = V_1 \union \dots \union V_k$$
  the {\bf reduced graph} $R_{\alpha, \eps}$ is the graph whose vertices are $[k]$ and
  $$i \sim j \iff (V_i, V_j) \eps-\text{uniform and } d(V_i, V_j) \ge \alpha$$
\end{dfn}

\begin{proof}[Proof of Theorem \ref{thm:k4-free-small-indep}]
  Let $\eps > 0$. We may assume $\eps < 1/4$. Let $L = L(10\eps^{-1}, \eps/4), \delta = \eps^2L^{-1}/10$. Assume $G$ is $K_4$-free and $\alpha(G) < \delta n$. We want
  $$e(G) \le \frac{n^2}8 + \eps n^2$$
  Regularity with $\eps/4$ and $\ell = 10/\eps$ gives a partition $V = V_1 \union \dots \union V_k$ where $\ell \le k \le L$. Let $R = R_{\eps, \eps/4}(V_1, \dots, V_k)$ be the corresponding reduced graph and $m = n/k$.
  \begin{claim}
    $R$ is triangle-free.
  \end{claim}
  \begin{proof}
    Assume $V_a, V_b, V_c$ form a triangle in $R$. Since the pairs are $\eps/4$-uniform and have density at least $\eps$, the triangle embedding lemma tells us that there are at least
    $$(1 - 2\eps)\eps^3 m^3$$
    triangles on $V_a, V_b, V_c$, where $m = \frac nk$. Find therefore $x \in V_a, y \in V_b$ such that $x \sim y$ and
    $$\#\{z \in V_c \mid x, y, z\text{ is a triangle}\} \ge (1 - 2\eps)\eps^3m \ge \frac 12 \eps^3 \frac nL > \delta n$$
    Hence $Z$ can't be an independent set and there are some $u, v \in Z$ such that $u \sim v$. But then $x, y, u, v$ is a $K_4$. Contradiction.
  \end{proof}

  \newlec

  \begin{claim}
    All $\eps/4$-uniform pairs $(V_i, V_j)$ have $d(V_i, V_j) \le 1/2 + \eps$.
  \end{claim}
  \begin{proof}
    Assume $(V_i, V_j)$ contradicts the claim. Then let
    $$U = \{v \in V_i \mid \abs{N(v) \inter V_j} \ge \frac 12 + \frac\eps 2\}$$
    so that $\abs U \ge (1 - \eps/4)m > \delta n$ by uniformity. By assumption, there therefore exist $x, y \in U$ such that $x \sim y$. Now note that
    $$\abs{N(x) \inter N(y)} \ge \eps\abs{V_j} = \eps m \ge \delta n$$
    So there exist $u, v \in N(x) \inter N(y)$ such that $u \sim v$. But now $x, y, u, v$ is a $K_4$. Contradiction.
  \end{proof}
  The first claim gives us a bound on the number of edges in the reduced graph, the second claim gives us a bound on the density of those edges, and all other edges have negligible density. So we are morally done. Formally, by splitting the edges of $G$ according to which kind of pair of the partition it belongs to,
  \begin{align*}
    e(G)
    & \le \underbrace{e(R_{\eps, \eps/4})\left(\frac 12 + \eps\right)m}_{\text{in the reduced graph}}
    + \underbrace{\frac\eps 4 \binom k2 m^2}_{\text{not $\eps/4$-uniform}}
    + \underbrace{\eps \binom k2 m^2}_{\text{low density}}
    + \underbrace{k m^2}_{\text{within a part}} \\
    & \le \frac 12 e(R_{\eps, \eps/4}) m^2 + C\eps n^2 \\
    & \le \frac{n^2}8 + C'\eps n^2 \text{ by Tur\' an}
  \end{align*}
\end{proof}

\begin{thm}[Chv\' atal-Rödl-Szemerédi-Trotter, 1983]
  $$r(H) \le C_d \abs H$$
  for some constant $C_d$ where $H$ has max degree $d$.
\end{thm}
\begin{rmk}
  The naïve Ramsey bound gives
  $$r(H) \le R(\abs H) \le 4^{\abs H}$$
\end{rmk}

\begin{lem}[Embedding lemma]
  Let $H$ be a $r$-partite graph with max degree $d$ and $r$-partition
  $$V(H) = W_1 \union \dots \union W_r$$
  where $\abs{W_i} \le s$. Now let $m \in \N, \eps, \lambda \in ]0, 1[$ be such that
  $$\eps(d + 1) < (\lambda - \eps)^d, \quad s \le \eps m$$
  where $\abs{V_i} = m$, $(V_i, V_j)$ is $\eps$-uniform and $d(V_i, V_j) \ge \lambda$. Then $H \subseteq G$.
\end{lem}
\begin{proof}
  We define an algorithm to embed $H$ into $G$. In each step of the algorithm, we will choose a new vertex in $V(H)$ that has not been embedded yet and find some vertex in $V(G)$ to map it to.

  At step $t$, for each vertex $u \in V(H)$, define
  $$E_t = \text{ already embedded vertices}, \quad \mathcal C_t(u) = V_\ell \inter \Inter_{\substack{w \in E_t \\ w \sim_H u}} N_G(w) \text{ where } u \in V_\ell$$
  At each step of the algorithm, ensure the following:
  \begin{enumerate}
    \item If $x, y \in E_t$ and $x \sim_H y$, then $x \sim_G y$.
    \item For all $u \nin E_t$,
    $$\mathcal C_t(u) \ge m(\lambda - \eps)^{\abs{N(u) \inter E_t}}$$
  \end{enumerate}
  At step $t + 1$, let $v \nin E_t$. We want to find a vertex in $\mathcal C_t(v)$ to map $v$ to. This ensure condition 1. Now, for $u \nin E_t \union \{v\}$,
  $$\mathcal C_{t + 1}(u) = \begin{cases}
    \mathcal C_t(u) \inter N_G(v) & \text{ if } u \sim_H v \\
    \mathcal C_t(u) & \text{ if } u \not\sim_H v
  \end{cases}$$
  So we just need to exclude vertices from $\mathcal C_t(u)$ that have
  \begin{enumerate}
    \item $\abs{\mathcal C_t(u) \inter N_G(v)} < (\lambda - \eps)\abs{\mathcal C_t(u)}$
    \item are already identified with vertices in $E_t$.
  \end{enumerate}
  By uniformity, there are at most $d\eps m$ vertices of the first type. Since $\abs{W_\ell} \le s \le \eps m$, there are at most $\eps m$ vertices of the second type. Thus there are at least
  $$\abs{\mathcal C_t(u)} - \eps(d + 1)m \ge (\lambda - \eps)^d m - \eps(d + 1)m > 0$$
  good choices for $v$. Pick one. Done.
\end{proof}

\newlec

\begin{lem}
  Let $H$ be a $n$-vertex graph with max degree $d$. Then there exists a partition
  $$V(H) = V_1 \union \dots \union V_{10d^2}$$
  so that all edges are between parts and
  $$\abs{V_i} \le \frac{100n}d$$
\end{lem}
\begin{proof}
  $H$ is $d + 1$-colorable, say with parts $W_1, \dots, W_{d + 1}$. Split each $W_i$ into at most $\ceil{d/100}$ parts of size at most $100n/d$. This gives at most
  $$(d + 1)\ceil{\frac d{100}} \le \frac{d^2}{25} \le 10d^2$$
  parts for big enough $d$.
\end{proof}

\begin{proof}[Proof of the theorem]
  Let $r = 10d^2, t = R(10d^2)$. Let $\eps < \frac 1{t + 1}$ be such that
  $$(d + 1)\eps < \left(\frac 12 - \eps\right)^d$$
  Let $\ell > t + 1$ and $L(\ell, \eps)$ be the Regularity Lemma constant. Writing $\abs H = n$. We will show that
  $$r(H) \le \frac{Ln}\eps$$
  Let $N > \max(100/d, 1) Ln/\eps$ and $\chi$ be a red/blue edge coloring of $K_N$. Let $G$ be the red graph. Apply the Regularity Lemma with parameters $\eps, \ell$ to get a partition $V_1, \dots, V_k$ with $\ell \le k \le L$.

  {\bf Step 1} \\
  There are at least $(1 - \eps)\binom k2 > (1 - \frac 1{t + 1})\binom k2$ $\eps$-uniform pairs, so we can find a $K_t$ in the graph of $\eps$-uniform pairs. WLOG $V_1, \dots, V_t$ is that $K_t$.

  {\bf Step 2} \\
  Color each pair $(V_i, V_j)$ with the majority color between $V_i$ and $V_j$ in $G$. Apply Ramsey to find a monochromatic $K_{10d^2}$. WLOG $V_1, \dots, V_{10d^2}$ are all majority red. Partition $H$ into parts $W_1, \dots, W_{10d^2}$ of size at most $100n/d$. Since
  $$\abs{W_i} \le \frac{100n}d \le \frac{\eps N}L \le \frac{\eps N}k$$
  we can apply the Embedding Lemma with $\lambda = 1/2, m = N/L, s = \eps N/L$ to finish.
\end{proof}


\section{Dependent Random Choice}

\begin{dfn}
  Let $G$ be a graph. We say $R \subseteq V(G)$ is {\bf $(s, k)$-rich} if, for all $x_1, \dots, x_k \in R$,
  $$k \le \abs{N(x_1) \inter \dots \inter N(x_k)}$$
\end{dfn}

TODO: Insert picture

\begin{thm}
  Let $G$ be a graph on $n$ vertices with $m$ edges. Let $t, s, r, k > 0$ satisfy
  $$\frac{(2m)^t}{n^{2t - 1}} - \binom ns \left(\frac kn\right)^t \ge r$$
  Then $G$ contains a $(s, k)$-rich set of size at least $r$.
\end{thm}
\begin{rmk}
  $t$ is a free parameter in the statement, so we get to optimise over $t$ in applications.
\end{rmk}
\begin{idea}
  Choose $v_1, \dots, v_t \in V(G)$ uniformly at random and consider $R = N(v_1) \inter \dots \inter N(v_t)$. neighborhoods of high degree often are in $R$, so $R$ is likely to be rich.
\end{idea}
\begin{proof}
  Let $v_1, \dots, v_t \in V(G)$ be chosen uniformly at random. First consider
  \begin{align*}
    \E_{v_1, \dots, v_t} \abs{N(v_1) \inter \dots \inter N(v_t)}
    & = \E_{v_1, \dots, v_t} \sum_y 1_{y \sim v_1, \dots, v_t} \\
    & = \sum_y \P(y \sim v_1, \dots, v_t) \\
    & = \sum_y \P(y \sim v_1) \dots P(y \sim v_t) \\
    & = \sum_y \left(\frac{d(y)}n\right)^t \\
    & \ge n \left(\frac{2m}{n^2}\right)^t \\
    & = \frac{(2m)^t}{n^{2t - 1}}
  \end{align*}
  Let
  $$Y = \{(y_1, \dots, y_s) \in (N(v_1) \inter \dots \inter N(v_t))^s \mid \abs{N(y_1) \inter \dots \inter N(y_s)} < k\}$$
  We compute
  \begin{align*}
    \E_{v_1, \dots, v_t} \abs Y
    & = \E_{v_1, \dots, v_t} \sum_{\abs{N(y_1) \inter \dots \inter N(y_s)} < k} 1_{y_1, \dots, y_s \in N(v_1) \inter \dots \inter N(v_t)} \\
    & = \sum_{\abs{N(y_1) \inter \dots \inter N(y_s)} < k} \P(y_1, \dots, y_s \in N(v_1) \inter \dots \inter N(v_t)) \\
    & \le \binom ns \left(\frac kn\right)^t
  \end{align*}
  Define $R$ to be $N(v_1) \inter \dots \inter N(v_t)$ with a vertex removed from each tuple $(y_1, \dots, y_s)$ such that $\abs{N(y_1) \inter \dots N(y_s)} < k$. Then
  $$\E \abs R \ge \E[\abs{N(v_1) \inter \dots \inter N(v_t)} - Y] \ge \frac{(2m)^t}{n^{2t - 1}} - \binom ns \left(\frac kn\right)^t \ge r$$
  So $\abs R \ge r$ for some $y_1, \dots, y_s$, and we have found a $(s, k)$-rich set of size at least $r$.
\end{proof}

In Part II Graph Theory, we saw
$$\mathrm{ex}(n, K_{t, t}) \le C_t n^{2 - 1/t}$$

\begin{thm}[Füredi, 1991]
  Let $H$ be a bipartite graph with bipartition $A \union B$ where $\deg y \le s$ for all $y \in B$. Then
  $$\mathrm{ex}(n, H) \le C_H n^{2 - 1/s}$$
  for some constant $C_H$.
\end{thm}

\newlec

\begin{lem}[Rich Set Embedding Lemma]
  Let $H$ be a bipartite graph on parts $A$ and $B$ such that $\deg y \le s$ for all $y \in B$. Let $G$ be a graph and $R \subseteq V(G)$ be a $(s, \abs B)$-rich set with $\abs R \ge \abs A$. Then $H \subseteq G$.
\end{lem}
\begin{proof}
  Let $A = \{x_1, \dots, x_a\}, B = \{y_1, \dots, y_b\}$. Find $\tilde x_1, \dots, \tilde x_a \in R$ distinct. Start by embedding $x_i \mapsto \tilde x_i$. We choose $\tilde y_i$ such that $y_i \mapsto \tilde y_i$ inductively as follows. Assume we have already determined $\tilde y_1, \dots, \tilde y_i$. Consider $N(y_{i + 1}) = \{x_{i_1}, \dots, x_{i_\ell}\}$ and note that
  $$\abs{N(\tilde x_{i_1} \inter \dots \inter N(\tilde x_{i_\ell}))} \ge \abs H$$
  by richness. Thus simply choose any
  $$\tilde y_{i + 1} \in N(\tilde x_{i_1}) \inter \dots \inter N(\tilde x_{i_\ell}) \setminus \{\tilde y_1, \dots, \tilde y_i\}$$
\end{proof}

\begin{proof}[Proof of Füredi]
  We want to apply our Rich Set Embedding Lemma to find a $(s, \abs H)$-rich set $R \subseteq V(G)$ such that $\abs R \ge \abs H$. We know $m \ge C_H n^{2 - 1/s}$ for some constant $C_H$ to be chosen and we want to find $t$ such that
  $$\frac{(2m)^t}{n^{2t - 1}} - \binom ns \left(\frac{\abs H}N\right)^t \ge \abs H$$
  Take $C_H = 2\abs H, t = s$ so that
  $$\lhs \ge \frac{(2C_H)^t n^{2t - t/s}}{n^{2t - 1}} - \left(\frac{en}s\right)^s \left(\frac{\abs H}n\right)^t = \left(4^s - \left(\frac es\right)^s\right)\abs H^s \ge \abs H$$
\end{proof}

Let $Q_d$ be the graph on $\{0, 1\}^d$ such that $x \sim y$ iff they differ in one coordinate.

\begin{thm}
  For all large enough $d$,
  $$r(Q_d) \le 2^{3d}$$
\end{thm}
\begin{proof}
  We want to apply DRC + Rich Set Embedding Lemma. Let $N = 2^{3d}$ and $\chi$ be a red-blue coloring of $K_N$. Let $G$ be the graph of the majority color, so that
  $$e(G) \ge \frac 12 \binom N2 =: m$$
  By the Rich Set Embedding Lemma, we are done if we find a $(d, 2^d)$-rich set $R \subseteq V(G)$ with $\abs R \ge 2^d$. By DRC, there exists such a rich set if we can find $t$ such that
  $$\frac{(2m)^t}{N^{2t - 1}} - \binom Nd \left(\frac{2^d}N\right)^t \ge 2^d$$
  Pick $t = \frac 32 d$. Then
  \begin{align*}
    \lhs
    & \ge \frac 1{N^{2t - 1}} \left(\frac{N(N - 1)}2\right)^t - \left(\frac{eN}D\right)^d \left(\frac{2^d}N\right)^t \\
    & = \left(1 - \frac 1N\right)^t \frac N{2^t} - \left(\frac ed\right)^d N^{d - \frac 23 t} \\
    & = (1 - o(1))2^{\frac 32 d} - \left(\frac ed\right)^d \\
    & \gg 2^d
  \end{align*}
\end{proof}
\begin{rmk}
  The same argument shows
  $$r(Q_d) \le 2^{(\varphi + 1 + o(1))d}$$
  where $\varphi + 1 = \frac{3 + \sqrt 5}2 = 2.618\dots$.
\end{rmk}

How does $r(Q_d)$ grow?

\begin{conj}[Erd\H os]
  $$r(Q_d) \le C 2^d$$
\end{conj}

\begin{thm}[Conlon-Fox-Sudakov]
  $$r(Q_d) \le 2^{2d + o(d)}$$
\end{thm}

\begin{thm}[Tikharimov]
  $$r(Q_d) \le 2^{(2 - c)d}$$
  for some $c > 0$ and all large enough $d$.
\end{thm}


\subsection{Ramsey-Tur\' an}

\begin{dfn}
  For $n, k \in \N$ and a graph $H$, we define
  $$\mathrm{RT}(n, H, t) = \max\{e(G) \mid \abs G = n, G \not\supseteq H, \alpha(G) \le k\}$$
\end{dfn}

We showed that
$$\mathrm{RT}(n, K_4, o(n)) \le \frac{n^2}8 + o(n^2)$$

\begin{thm}[Sudakov]
  Let $\delta(n) = \exp(-\omega(n)\sqrt{\log n})$ where $\omega(n) \to \infty$. Then
  $$\mathrm{RT}(n, K_4, \delta(n) n) = o(n^2)$$
\end{thm}
\begin{proof}
  Let $G$ be a $n$-vertex graph with $G \not\supseteq K_4$ and $\alpha(G) \le \delta(n)n$. Let's first find a $(2, \delta(n)n)$-rich set $R \subseteq V(G)$ such that $\abs R > \delta(n)n$. By DRC, we are looking for $t$ such that
  $$\frac{(2\eps n^2)^t}{n^{2t - 1}} - \binom n2 \left(\frac{\delta(n)n}n\right)^t \ge \delta(n)n$$
  Choose $t = 2\sqrt{\log n}/\omega(n)$ so that $\delta(n)^t = n^{-2}$. Since $\omega(n)^2 \gg 2\log\eps^{-1}$, we have
  \begin{align*}
    \lhs
    & \ge \eps^t n - n^2 \delta(n)^t \\
    & = \exp\left(-\frac{3\log \eps^{-1}}{\omega(n)}\sqrt{\log n}\right)n - 1 \\
    & \ge \exp\left(-\omega(n)\sqrt{\log n}\right)n \quad \text{ for large enough }n \\
    & = \delta(n)n
  \end{align*}
\end{proof}


\section{Exponential improvement on Ramsey numbers}

\newlec

\printindex
\end{document}