\documentclass{article}

% preamble
\def\npart{III}
\def\nyear{2023}
\def\nterm{Michaelmas}
\def\nlecturer{Dr Andr\'as Zs\'ak}
\def\ncourse{Functional Analysis}
\def\draft{Incomplete}

\ifx \nauthor\undefined
  \def\nauthor{Ya\"el Dillies}
\else
\fi

\author{Based on lectures by \nlecturer \\\small Notes taken by \nauthor}
\date{\nterm\ \nyear}
\ifdefined\draft
\title{Part \npart\ -- \ncourse\ (\draft)}
\else
\title{Part \npart\ -- \ncourse}
\fi

\usepackage[utf8]{inputenc}
\usepackage{amsmath}
\usepackage{amsthm}
\usepackage{amssymb}
\usepackage{cancel}
\usepackage{enumerate}
\usepackage{mathtools}
\usepackage{fancyhdr}
\usepackage{graphicx}
\usepackage[dvipsnames]{xcolor}
\usepackage{tikz}
\usepackage{wrapfig}
\usepackage{centernot}
\usepackage{float}
\usepackage{braket}
\usepackage{marginnote}
\usepackage{mathdots}
\usepackage{mathrsfs}
\usepackage{ifthen}
\usepackage{imakeidx}
\usepackage{parskip}
\usepackage{relsize}
\usepackage{tabularx}
\usepackage[hypcap=true]{caption}
\usepackage[shortlabels]{enumitem}
\usepackage[pdftex,
  colorlinks=true,
  linkcolor=lblue,
  pdfauthor={\nauthor},
  pdfsubject={Cambridge Maths Notes: Part \npart\ - \ncourse},
  pdftitle={\ncourse - Part \npart},
pdfkeywords={Cambridge Mathematics Maths Math \npart\ \nterm\ \nyear\ \ncourse}]{hyperref}

\usepackage[capitalise,nameinlink,noabbrev]{cleveref}
\usepackage{nameref}
\usepackage[margin=1.5in,a4paper]{geometry}

\reversemarginpar
\newcommand{\lecnum}[1]{\leavevmode\marginnote{\emph{Lecture #1}}\ignorespaces}
\newcounter{lecturenumber}
\newcommand{\newlec}{\stepcounter{lecturenumber}\lecnum{\arabic{lecturenumber}}}

% Theorems
\theoremstyle{definition}
\newtheorem*{aim}{Aim}
\newtheorem*{assumption}{Assumption}
\newtheorem*{axiom}{Axiom}
\newtheorem*{claim}{Claim}
\newtheorem*{cor}{Corollary}
\newtheorem*{conjecture}{Conjecture}
\newtheorem*{defi}{Definition}
\newtheorem*{eg}{Example}
\newtheorem*{egs}{Examples}
\newtheorem*{ex}{Exercise}
\newtheorem*{fact}{Fact}
\newtheorem*{goal}{Goal}
\newtheorem*{idea}{Idea}
\newtheorem*{law}{Law}
\newtheorem*{lemma}{Lemma}
\newtheorem*{notation}{Notation}
\newtheorem*{note}{Note}
\newtheorem*{obs}{Observation}
\newtheorem*{prop}{Proposition}
\newtheorem*{properties}{Properties}
\newtheorem*{question}{Question}
\newtheorem*{rrule}{Rule}
\newtheorem*{steps}{Steps}
\newtheorem*{thm}{Theorem}

\newtheorem*{rmk}{Remark}
\newtheorem*{rmks}{Remarks}
\newtheorem*{warning}{Warning}
\newtheorem*{exercise}{Exercise}

\newtheorem{nthm}{Theorem}[section]
\newtheorem{ndef}[nthm]{Definition}
\newtheorem{nprop}[nthm]{Proposition}
\newtheorem{nconjecture}{Conjecture}
\newtheorem{ncor}[nthm]{Corollary}
\newtheorem{nex}[nthm]{Example}
\newtheorem{nlemma}[nthm]{Lemma}
\newtheorem{problem}[nthm]{Problem}

% Command redirections
\let\P\oldP
\let\oldemptyset\emptyset
\let\emptyset\varnothing

% Letter shorthands
\newcommand{\C}{\mathbb C}
\newcommand{\bbE}{\mathbb E}
\newcommand{\F}{\mathbb F}
\newcommand{\K}{\mathbb K}
\newcommand{\N}{\mathbb N}
\newcommand{\P}{\mathbb P}
\newcommand{\Q}{\mathbb Q}
\newcommand{\R}{\mathbb R}
\newcommand{\Z}{\mathbb Z}
\newcommand{\mcA}{\mathcal A}
\newcommand{\mcB}{\mathcal B}
\newcommand{\mcC}{\mathcal C}
\newcommand{\mcD}{\mathcal D}
\newcommand{\mcE}{\mathcal E}
\newcommand{\mcF}{\mathcal F}
\newcommand{\mcG}{\mathcal G}
\newcommand{\mcH}{\mathcal H}
\newcommand{\mcM}{\mathcal M}
\newcommand{\mcN}{\mathcal N}
\newcommand{\mcO}{\mathcal O}
\newcommand{\mcP}{\mathcal P}
\newcommand{\mcQ}{\mathcal Q}
\newcommand{\mcR}{\mathcal R}
\newcommand{\mcS}{\mathcal S}
\newcommand{\mcT}{\mathcal T}
\newcommand{\mcU}{\mathcal U}
\newcommand{\mcV}{\mathcal V}
\newcommand{\eps}{\varepsilon}
\newcommand{\Eps}{\mathcal E}

\newcommand{\curlybrack}[1]{\left\{ #1\right\}}
\newcommand{\abs}[1]{\left\lvert #1\right\rvert}
\newcommand{\norm}[1]{\left\lVert #1\right\rVert}
\newcommand{\inn}[2]{\left\langle #1, #2\right\rangle}
\newcommand{\floor}[1]{\left\lfloor #1\right\rfloor}
\newcommand{\ceil}[1]{\left\lceil #1\right\rceil}
\newcommand{\doublesqbrack}[1]{[\![#1]\!]}

\newcommand{\imp}{\implies}
\newcommand{\for}{\forall}
\newcommand{\mor}{\rightarrow}
\newcommand{\nin}{\notin}
\newcommand{\comp}{\circ}
\newcommand{\union}{\cup}
\newcommand{\inter}{\cap}
\newcommand{\Union}{\bigcup}
\newcommand{\Inter}{\bigcap}
\newcommand{\hatplus}{\mathbin{\widehat{+}}}
\newcommand{\symdif}{\mathbin\varbigtriangleup}
\newcommand{\aeeq}{\overset{\text{ae}}=}
\newcommand{\lexlt}{\overset{\text{lex}}<}
\newcommand{\colexlt}{\overset{\text{colex}}<}
\newcommand{\wtendsto}{\overset w\mor}
\newcommand{\wstartendsto}{\overset{w*}\mor}
\renewcommand{\vec}[1]{\boldsymbol{\mathbf{#1}}}
\renewcommand{\bar}[1]{\overline{#1}}

\newcommand*{\E}{
  \mathop{
    \mathchoice{\vcenter{\hbox{\larger[4]$\mathbb{E}$}}}
               {\kern0pt\mathbb{E}}
               {\kern0pt\mathbb{E}}
               {\kern0pt\mathbb{E}}
  }\displaylimits
}

\newcommand{\named}[1]{\textbf{#1}\index{#1}}
\newcommand{\bonusnamed}[1]{\textbf{#1}\index{#1@*#1}}

\let\Im\relax
\let\Re\relax

\DeclareMathOperator{\Ber}{Ber}
\DeclareMathOperator{\conv}{conv}
\DeclareMathOperator{\diam}{diam}
\DeclareMathOperator{\codim}{codim}
\DeclareMathOperator{\esssup}{ess sup}
\DeclareMathOperator{\Ext}{Ext}
\DeclareMathOperator{\id}{id}
\DeclareMathOperator{\im}{im}
\DeclareMathOperator{\Im}{Im}
\DeclareMathOperator{\interior}{int}
\DeclareMathOperator{\lhs}{LHS}
\DeclareMathOperator{\rank}{rank}
\DeclareMathOperator{\Re}{Re}
\DeclareMathOperator{\rhs}{RHS}
\DeclareMathOperator{\Span}{Span}
\DeclareMathOperator{\Spec}{Spec}
\DeclareMathOperator{\supp}{supp}
\DeclareMathOperator{\Var}{Var}

\definecolor{lblue}{rgb}{0., 0.05, 0.6}
\definecolor{mblue}{rgb}{0.2, 0.3, 0.8}
\definecolor{morange}{rgb}{1, 0.5, 0}
\definecolor{mgreen}{rgb}{0.1, 0.4, 0.2}
\definecolor{mred}{rgb}{0.5, 0, 0}

\colorlet{bred}{red}
\colorlet{bblue}{Cyan!50!blue}
\colorlet{byellow}{yellow}
\colorlet{bgreen}{YellowGreen!50!Green}
\colorlet{borange}{red!20!yellow}
\colorlet{bpurple}{violet}

\newcommand{\red}[1]{\textcolor{bred}{#1}}
\newcommand{\green}[1]{\textcolor{bgreen}{#1}}
\newcommand{\blue}[1]{\textcolor{bblue}{#1}}
\newcommand{\yellow}[1]{\textcolor{byellow}{#1}}
\newcommand{\orange}[1]{\textcolor{borange}{#1}}
\newcommand{\purple}[1]{\textcolor{bpurple}{#1}}

\pagestyle{fancy}
\fancyhf{}
\fancyfoot[R]{\href{yaeldillies.github.io/maths-notes}{\color{lblue}{Updated online}}}
\fancyfoot[C]{\thepage}
\ifdefined\draft
\fancyfoot[L]{\emph{\draft}}
\else
\fi
\renewcommand{\headrulewidth}{0pt}
\renewcommand{\footrulewidth}{0.2pt}

% Counters and table of content

\swapnumbers
\reversemarginpar

\usetikzlibrary{positioning, decorations.pathmorphing, decorations.text, calc, backgrounds, fadings}
\tikzset{node/.style = {circle,draw,inner sep=0.8mm}}

\makeindex[intoc]

\setcounter{section}{-1}

% and here we go!
\begin{document}
\maketitle

\tableofcontents

\clearpage

\section{Introduction}

\subsection*{Prerequisites}

\begin{itemize}
  \item some basic functional analysis
  \item a bit of measure theory
  \item a bit of complex analysis
\end{itemize}

\subsection*{Books}

Books relevant to the course are:
\begin{itemize}
  \item Bollob\'as, \textit{Linear Analysis}
  \item Murphy, \textit{$C^*$-algebras}
  \item Rudin
  \item Graham-Allan
\end{itemize}

\subsection*{Notation}

We will use $\K$ to mean "either $\R$ or $\C$".

For $X$ a normed space, we define
\begin{align*}
  B_X & = \{x \in X \mid \norm x \le 1\} \\
  S_X & = \{x \in X \mid \norm x = 1\} \\
  D_X & = \{x \in X \mid \norm x < 1\}
\end{align*}

For $X, Y$ normed spaces, we write $X \sim Y$ if $X, Y$ are isomorphic, ie there
exists a linear bijection $T : X \to Y$ such that $T$ and $T^{-1}$ are continuous. We write $X \cong Y$ if $X, Y$ are isometrically isomorphic, ie there exists a surjective linear map $T : X \to Y$ such that $\norm{T x} = \norm x$ for all $x$.

\clearpage

\section{Hahn-Banach extension theorems}

\newlec

Let $X$ be a normed space. The {\bf dual space} of $X$ is the space $X^*$ of
bounded linear functionals on $X$.
$X^*$ is always a Banach space in the operator norm: for $f \in X^*$,
$$\norm f = \sup_{x \in B_X} |f(x)|$$

\begin{egs*}~
  \begin{itemize}
    \item For $1 < p, q < \infty$, $p^{-1} + q^{-1} = 1$, $\ell_p^* \cong \ell_q$
    \item $\ell_1^* \cong \ell_\infty$, $c_0^* \cong \ell_1$
    \item If $H$ is a Hilbert space, then $H^* \cong H$ (the isomorphism is conjugate-linear in the complex case).
  \end{itemize}
\end{egs*}

For $x \in X, f \in X^*$, we write $\inn x f = f(x)$. Note that
$$\inn x f = |f(x)| \le \norm f\norm x$$

\begin{dfn*}
  Let $X$ be a {\it real} vector space. A functional $p : X \to \R$ is
  \begin{itemize}
    \item {\bf positive homogeneous} if $p(tx) = tp(x)$ for all $x \in X$, $t \ge 0$
    \item {\bf subadditive} if $p(x + y) \le p(x) + p(y)$ for all $x, y \in X$
  \end{itemize}
\end{dfn*}

\begin{dfn*}
  Let $P$ be a preorder, $A \subseteq P, x \in P$. We say
  \begin{itemize}
    \item $x$ is an {\bf upper bound} for $A$ if $\for a \in A, a \le x$.
    \item $A$ is a {\bf chain} if $\for a, b \in A, a \le b \lor b \le a$.
    \item $x$ is a {\bf maximal element} if $\for y \in P, x \not < y$
  \end{itemize}
\end{dfn*}

\begin{fact}[Zorn's lemma]
  A nonempty preorder in which all nonempty chains have an upper bound has a maximal element.
\end{fact}

\begin{thm}[Hahn-Banach, positive homogeneous version]\label{thm:hb-positive}
  Let $X$ be a real vector space and $p : X \to \R$ be positive homogeneous and subadditive. Let $Y$ be a subspace of $X$ and $g : Y \to \R$ be linear such that $\for y \in Y, g(y) \le p(y)$. Then there exists $f : X \to \R$ linear such that $f\restriction_Y = g$ and $\for x \in X, f(x) \le p(x)$.
\end{thm}
\begin{proof}
  Let $P$ be the set of pairs $(Z, h)$ where $Z$ is a subspace of $X$ with $Y \subseteq Z$ and $h : Z \to \R$ linear, $h\restriction_Y = g$ and $\for z \in Z, h(z) \le p(z)$. $P$ is nonempty since $(Y, g) \in P$, and is partially ordered by
  $$(Z_1, h_1) \le (Z_2, h_2) \iff Z_1 \subseteq Z_2 \land h_2\restriction_{Z_1} = h_1$$
  If $\{(Z_i, h_i) \mid i \in I\}$ is a chain with $I$ nonempty, then we can define
  $$Z := \Union_{i \in I} Z_i, h\restriction_{Z_i} = h_i$$
  The definition of $h$ makes sense thanks to the chain assumption. $(Z, h) \in P$ is therefore an upper bound for the chain. \\
  Hence find by Zorn a maximal element $(Z, h)$ of $P$. If $Z = X$, we won. So assume there is some $x \in X \ Z$. Let $W = \Span(Z \cup \{x\})$ and define $f : W \to \R$ by
  $$f(z + \lambda x) = h(z) + \lambda\alpha$$
  for some $\alpha \in \R$. Then $f$ is linear and $f\restriction_Z = h$. We now look for $\alpha$ such that $\for w \in W, f(w) \le p(w)$. We would then have $(W, f) \in P$ and $(Z, h) < (W, f)$, contradicting maximality of $(Z, h)$. \\
  We need
  $$h(z) + \lambda\alpha \le p(z + \lambda x) \for z \in Z, \lambda \in \R$$
  Since $p$ is positive homogeneous, this becomes
  \begin{align}
    h(z) + \alpha \le p(z + x)
    h(z) - \alpha \le p(z - x)
  \end{align}
  ie
  $$h(z) - p(z - x) \le \alpha \le p(z + x) - h(z) \for z \in Z$$
  The existence of $\alpha$ now amounts to
  $$h(z_1) - p(z_1 - x) \le \alpha \le p(z_2 + x) - h(z_2) \for z_1, z_2 \in Z$$
  But indeed
  $$h(z_1) + h(z_2) = h(z_1 + z_2) \le p(z_1 + z_2) \le p(z_1 - x) + p(z_2 + x)$$
\end{proof}

\begin{dfn*}
  Let $X$ be a $\K$-vector space. A {\bf seminorm} on $X$ is a functional $p : X \to \R$ such that
  \begin{itemize}
    \item $\for x \in X, p(x) \ge 0$
    \item $\for x \in X, \lambda \in \K, p(\lambda x) = |\lambda| p(x)$
    \item $\for x, y \in X, p(x + y) \le p(x) + p(y)$
  \end{itemize}
\end{dfn*}

\begin{rmk}
  $$\text{norm} \imp \text{seminorm} \imp \text{positive homogeneous}$$
\end{rmk}

\newlec

\begin{thm}[Hahn-Banach, absolute homogeneous version]\label{thm:hb-absolute}
  Let $X$ be a real of complex vector space and $p$ a seminorm on $X$. Let $Y$ be a subspace of $X$, $g$ a linear functional on $Y$ such that $\for y \in Y, |g(y)| \le p(y)$. Then there exists a linear functional $f$ on $X$ such that $f\restriction_Y = g$ and $\for x \in X, |f(x)| \le p(x)$.
\end{thm}
\begin{proof}~\\
  {\bf Real case}
  $$\for y \in Y, g(y) \le |g(y)| \le p(y)$$
  By Theorem \ref{thm:hb-positive}, there exists $f : X \to \R$ such that $f\restriction_Y = g$ and $\for x \in X, f(x) \le p(x)$. We also have
  $$\for x \in X, -f(x) = f(-x) \le p(-x) = p(x)$$
  Hence $|f(x)| \le p(x)$ \\
  {\bf Complex case} \\
  $\Re g : Y \to \R$ is real-linear.
  $$\for y \in Y, |\Re g(y)| \le |g(y)| \le p(y)$$
  By the real case, find $h : X \to \R$ real-linear such that $h\restriction_Y = \Re g$
  \begin{claim}
    There exists a unique complex-linear $f : X \to \C$ such that $h = \Re f$.
  \end{claim}
  \begin{proof}~\\
    {\bf Uniqueness} \\
    If we have such $f$, then
    \begin{align*}
      f(x)
      & = \Re f(x) + i\Im f(x) \\
      & = \Re f(x) - i\Re f(ix) \\
      & = h(x) - ih(ix)
    \end{align*}
    {\bf Existence} \\
    Define $f(x) = h(x) - ih(ix)$. Then $f$ is real-linear and $f(ix) = if(x)$, so $f$ is complex-linear with $\Re f = h$.
  \end{proof}
  We now have $f : X \to \C$ such that $\Re f = h$.
  $$\Re f\restriction_Y = h\restriction_Y = \Re g$$
  So, by uniqueness, $f\restriction_Y = g$. \\
  Given $x \in X$, find $\lambda$ with $|\lambda| = 1$ such that
  \begin{align*}
    |f(x)|
    & = \lambda f(x) \\
    & = f(\lambda x) \\
    & = \Re f(\lambda x) \\
    & = h(\lambda x) \\
    & \le p(\lambda x) \\
    & = p(x)
  \end{align*}
\end{proof}

\begin{rmk}
  For a complex vector space $X$, if we write $X_\R$ for $X$ considered as a real vector space, the above proof shows that
  $$\Re : (X^*)_\R \to X_\R^*$$
  is an isometric isomorphism.
\end{rmk}

\begin{cor}\label{cor:hb-point}
  Let $X$ be a $\K$-vector space, $p$ a seminorm on $X$, $x_0 \in X$. Then there exists a linear functional $f$ on $X$ such that $f(x_0) = p(x_0)$ and $\for x \in X, |f(x)| \le p(x)$.
\end{cor}
\begin{proof}
  Let $Y = \Span(x_0)$,
  \begin{align*}
    g : Y & \to \K \\
    \lambda x_0 & \mapsto \lambda p(x_0)
  \end{align*}
  We see that $\for y \in Y, g(y) \le p(y)$. Hence find by Theorem \ref{thm:hb-absolute} a linear functional $f$ on $X$ such that $f\restriction_Y = g$ and $\for x \in X, |f(x)| \le p(x)$. We check that $f(x_0) = g(x_0) = p(x_0)$.
\end{proof}

\begin{thm}[Hahn-Banach, existence of support functionals]\label{thm:hb-support}
  Let $X$ be a real or complex normed space. Then
  \begin{enumerate}
    \item If $Y$ is a subspace of $X$ and $g \in Y^*$, then there exists $f \in X^*$ such that $f\restriction_Y = g$ and $\norm f = \norm g$.
    \item Given $x_0 \ne 0$, there exists $f \in S_{X^*}$ such that $f(x_0) = \norm{x_0}$.
  \end{enumerate}
\end{thm}
\begin{proof}~
  \begin{enumerate}
    \item Let $p(x) = \norm g \norm x$. Then $p$ is a seminorm on $X$ and
      $$\for y \in Y, |g(y)| \le \norm g \norm y = p(y)$$
      Find by Theorem \ref{thm:hb-positive} a linear functional $f$ on $X$ such that $f\restriction_Y = g$ and $\for x \in X, |f(x)| \le p(x) = \norm g \norm x$. So $\norm f \le \norm g$. Since $f\restriction_Y = g$, we also have $\norm g \le \norm f$. Hence $\norm f = \norm g$.
    \item Apply Corollary \ref{cor:hb-point} with $p(x) = \norm x$ to get $f \in X^*$ such that
      $$\for x \in X, |f(x)| \le \norm x \text{ and } f(x_0) = \norm{x_0}$$
      It follows that $\norm f = 1$.
  \end{enumerate}
\end{proof}

\begin{rmks}~
  \begin{itemize}
    \item Part 1 is a sort of linear version of Tietze's extension theorem: Given $K$ compact Hausdorff, $L \subseteq K$ closed, $g : L \to \K$ continuous, there exists $f : K \to \K$ such that $f\restriction_L = g$ and $\norm f_\infty = \norm g_\infty$.
    \item Part 2 shows that for all $x \ne y$ in $X$ there exists $f \in X^*$ such that $f(x) \ne f(y)$, namely $X^*$ {\bf separates points} of $X$. This is a sort of linear version of Urysohn: $C(K)$ separates points of $K$.
    \item The $f$ in part 2 is called a {\bf norming functional}, aka {\bf support functional}, for $x_0$. The existence of support functionals shows that
      $$x_0 = \max_{g \in B_{X^*}} \inn{x_0}g$$
      Assuming $X$ is a real normed space and $\norm{x_0} = 1$, we have $B_X \subseteq \{x \in X| f(x) \le 1\}$. Visually, TODO: insert tangency diagram
  \end{itemize}
\end{rmks}

\subsection{Bidual}

Let $X$ be a normed space. Then $X^{**}$ is called the {\bf bidual} or {\bf second dual} of $X$.

For $x \in X$, define $\hat x : X^* \to \K$, the {\bf evaluation at $x$}, by $\hat x(f) = f(x)$. $\hat x$ is linear and $|\hat x(f)| = |f(x)| \le \norm f \norm x$, so $\hat x \in X^{**}$ and $\norm{\hat x} \le \norm x$.

The map $x \mapsto \hat x : X \to X^{**}$ is called the {\bf canonical embedding} of $X$ into $X^{**}$.

\begin{thm}\label{thm:can-emb}
  The canonical embedding is an isometric embedding.
\end{thm}
\begin{proof}~\\
  {\bf Linearity}
  \begin{align*}
    \widehat{x + y}(f) & = f(x + y) = f(x) + f(y) = \hat x(f) + \hat y(f) \\
    \widehat{\lambda x}(f) & = f(\lambda x) = \lambda f(x) = \lambda \hat x(f)
  \end{align*}
  {\bf Isometry} \\
  If $x \ne 0$, there exists a support functional $f$ for $x$. Then
  $$ \norm{\hat x} \ge |\hat x(f)| = |f(x)| = \norm x$$
\end{proof}

\begin{rmks}~
  \begin{itemize}
    \item In bracket notation, $\inn f {\hat x} = \inn x f$
    \item Let $\hat X$ be the image of $X$ in $X^{**}$. Theorem \ref{thm:can-emb} says
      $$X \cong \hat X \subseteq X^{**}$$
      We often identify $\hat X$ with $X$ and think of $X$ as living isometrically inside $X^{**}$. Note that
      $$X \text{ complete } \iff \hat X \text{ closed in } X^{**}$$
    \item More generally, $\bar{\hat X}$ is a Banach space containing an isometric copy of $X$ as a dense subspace. We proved that normed spaces have completions!
  \end{itemize}
\end{rmks}

\begin{dfn*}
  A normed space $X$ is {\bf reflexive} if the canonical embedding $X \to X^{**}$ is surjective.
\end{dfn*}

\begin{egs*}~
  \begin{itemize}
    \item Some reflexive spaces are Hilbert spaces, finite-dimensional spaces, $\ell_p$ and $L_p(\mu)$ for $1 < p < \infty$.
    \item Some non-reflexive spaces are $c_0, \ell_1, \ell_\infty, L_1[0, 1]$.
  \end{itemize}
\end{egs*}

\begin{rmks}~
  \begin{itemize}
  \item If $X$ is reflexive, then $X \cong X^{**}$, so $X$ is complete.
  \item There are Banach spaces $X$ such that $X \cong X^{**}$ but $X$ is not reflexive, eg {\bf James' space}. Any isomorphism to the bidual is then necessarily not the canonical embedding.
  \end{itemize}
\end{rmks}

\subsection{Dual operators}

\newlec

Let $X, Y$ be normed spaces. Recall
$$\mcB(X, Y) = \{T : X \to Y \mid T \text{ linear, bounded}\}$$
This is a normed space in the operator norm:
$$\norm T = \sup_{x \in B_X} \norm{Tx}$$
If $Y$ is complete, then so is $\mcB(X, Y)$. For $T \in \mcB(X, Y)$, the {\bf dual operator} of $T$ is the map $T^* : Y^* \to X^*$ given by $T^*g = g \comp T$. In bracket notation $\inn x{T^*g} = \inn{Tx} g$ for $x \in X, g \in Y^*$.

{\bf $T^*$ is linear}
\begin{align*}
  \inn x {T^*(g + h)}
  & = \inn{Tx}{g + h} \\
  & = \inn{Tx}g + \inn{Tx}h \\
  & = \inn x{T^*g} + \and x{T^* h} \\
  & = \inn x{T^*g + T^*h}
\end{align*}
\begin{align*}
  \inn x {T^*(\lambda g)}
  & = \inn{Tx}{\lambda g} \\
  & = \lambda \inn{Tx}g \\
  & = \lambda \inn x{T^*g} \\
  & = \inn x{\lambda T^*g}
\end{align*}

{\bf $T^*$ is bounded}
\begin{align*}
  \norm{T^*}
  & = \sup_{g \in B_{Y^*}} \norm{T^*g} \\
  & = \sup_{g \in B_{Y^*}} \sup_{x \in B_X} \abs{\inn x{T^*g}} \\
  & = \sup_{x \in B_X} \sup_{g \in B_{Y^*}} \abs{\inn{Tx}g} \\
  & = \sup_{x \in B_X} \norm{Tx} \text{ by Theorem \ref{thm:hb-support} (ii)} \\
  & = \norm T
\end{align*}

\begin{rmks}~
  \begin{itemize}
    \item Hahn-Banach is crucial here. Without it, the dual could be $0$.
    \item $\inn\cdot\cdot$ is linear in both arguments. This contrasts with the Hilbert space case where $\inn\cdot\cdot$ is conjugate-linear in one of the arguments. This comes from the conjugate-linearity of the identification $H^* \cong H$.
    \item If $X, Y$ are Hilbert spaces and we identify $X, Y$ with $X^*, Y^*$, respectively, then $T^*$ is the adjoint of $T$.
  \end{itemize}
\end{rmks}

\begin{eg*}
  Let $1 < p, q < \infty$, $p^{-1} + q^{-1} = 1$ and define $R : \ell_p \to \ell_p$ to be the {\bf right shift operator} $(x_0, x_1, \dots) \mapsto (0, x_0, \dots)$. Then $R^* : \ell_q \to \ell_q$ is the {\bf left shift operator} $(x_0, x_1, \dots) \mapsto (x_1, x_2, \dots)$.
\end{eg*}

Some properties of the dual operator are
\begin{enumerate}
  \item $\id_X^* = \id_{X^*}$
  \item $(S + T)^* + S^* + T^*$, $(\lambda T)^* = \lambda T^*$
  \item $(ST)^* = T^*S^*$
  \item $T \mapsto T^* : \mcB(X, Y) \to \mcB(Y^*, X^*)$ is an {\it into} isomorphism.
  \item The double dual of an operator commutes with the double dual embedding. \\
    TODO: Insert commutative diagram
    For all $x$,
    $$\inn g{T^{**}\hat x} = \inn{T^*g}{\hat x} = \inn x{T^*g} = \inn{Tx}g = \inn g{\hat{Tx}}$$
    So $T^{**}\hat x = \widehat{Tx}$.
\end{enumerate}

\begin{rmk}
  From the above properties, if $X \sim Y$, then $X^* \sim Y^*$. Interestingly, if $X$ and $Y$ are reflexive, then we can deduce $X \sim Y$ from $X^* \sim Y^*$.
\end{rmk}

\subsection{Quotient spaces}

Let $X$ be a normed space and $Y$ be a {\it closed} subspace. Then the quotient space $X / Y$ becomes a normed space in the quotient norm:
$$\norm{x + Y} = d(x, Y) = \inf_{y \in Y} \norm{x + y}$$
The quotient map $q : X \to X/Y$ is linear and bounded: $\norm{q(x)} \le \norm x$, so $\norm q \le 1$.

$q$ maps the open unit ball $D_X$ onto $D_{X/Y}$. Indeed, if $x \in D_X$, then $\norm{q(x)} \le \norm x < 1$. Reciprocally, if $q(x) \in D_{X/Y}$, then there exists $y \in Y$ such that $\norm{x + y} < 1$. So $x + y \in D_X$ and $q(x + y) = q(x)$. It follows that $q$ is an open map and $\norm q = 1$.

If $Z$ is another normed space, $T \in \mcB(X, Z)$ and $Y \subseteq \ker T$, then there exists a unique map $\tilde T$ is linear and $\tilde T(D_{X/Y}) = \tilde T(q(D_X)) = T(D_X)$. It follows that $\norm{\tilde T} = \norm T$.

\begin{thm}
  Let $X$ be a normed space. If $X^*$ is separable, then so is $X$.
\end{thm}
\begin{rmk}
  The converse is false, as $X = \ell_1, X^* = \ell_\infty$ shows.
\end{rmk}
\begin{proof}
  Since $X^*$ is separable, so is $S_{X^*}$. Let $f_n$ be a dense subset of $S_{X^*}$. For every $n$, find $x_n \in B_X$ such that $f_n(x_n) > \frac 12$. Let
  $$Y = \bar{\Span\{x_n \mid n \in \N\}}$$
  \begin{claim}
    $Y = X$
  \end{claim}
  Then we're done since $Y$ is separable via $Y = \bar{\Span_\Q\{x_n \mid n \in \N\}}$.
  \begin{proof}
    Assume not. Then we can pick $g \in (X/Y)^*$, $\norm g = 1$ (by Theorem \ref{thm:hb-support} (ii)). Let $f = g \comp q$. Then $\norm f = \norm g = 1$, ie $f \in S_{X^*}$. Thus find $n$ such that $\norm{f - f_n} < \frac 14$, so that
    $$\frac 14 > \norm{f - f_n}\norm{x_n} \ge \abs{(f - f_n)(x_n)} = \abs{f_n(x_n)} > \frac 12$$
    contradiction.
  \end{proof}
\end{proof}

\begin{thm}
  Let $X$ be a separable normed space. Then $X$ embeds isometrically into $\ell_\infty$.
\end{thm}
\begin{proof}
  Let $\{x_n \mid n \in \N\}$ be dense in $X$. For every $n$, find $f_n \in S_{X^*}$, $f_n(x_n) = \norm{x_n}$ (assuming $X \ne \{0\}$). Define $T : X \to \ell_\infty$ by $(Tx)_n = f_n(x)$.

  {\bf Well definition}
  $$ \abs{(Tx)_n} = \abs{f_n(x)} \le \norm{f_n}\norm x = \norm x$$
  Hence $\norm{Tx}_\infty \le \norm x < \infty$.

  {\bf Linearity}
  $$ (T(x + y))_n = f_n(x + y) = f_n(x) + f_n(y) = (Tx + Ty)_n$$
  $$ (T(\lambda x))_n = f_n(\lambda x) = \lambda f_n(x)) = (\lambda Tx)_n$$
  so $T(x + y) = Tx + Ty, T(\lambda x) = \lambda Tx$.

  {\bf Isometry} \\
  We already know $\norm{Tx}_\infty \le \norm x$. On the other hand, find $f$ a supporting functional for $x$ and $f_{n_k}$ a subsequence converging to $f$. Then
  $$\norm{Tx}_\infty \ge \sup_k (Tx)_{n_k} = \sup_k \abs{f_{n_k}(x)} \ge \abs{f(x)} = \norm x$$
\end{proof}

\newlec

\begin{rmks}~
  \begin{itemize}
    \item The result says that $\ell_\infty$ is isometrically universal for the class $\mathcal{SB}$ of separable Banach spaces.
    \item There is a dual result: Every separable Banach space is a quotient of $\ell_1$.
  \end{itemize}
\end{rmks}

\begin{thm}[Vector-valued Liouville]\label{thm:vector-liouville}
  Lex $X$ be a complex Banach space, $f : \C \to X$ holomorphic and bounded. Then $f$ is constant.
\end{thm}
\begin{proof}
  Find $M \ge 0$ such that $\for z \in \C, \abs{f(z)} \le M$. Fix $\phi \in X^*$. $\phi \comp f : \C \to \C$ is

  {\bf bounded}
  $$\abs{\phi(f(z))} \le \norm\phi \norm{f(z)} \le M\norm\phi$$

  {\bf holomorphic}
  $$\frac{\phi(f(z)) - \phi(f(w))}{z - w} = \phi\left(\frac{f(z) - f(w)}{z - w}\right) \to \phi(f'(z))$$

  By scalar Liouville, $\phi \comp f$ is constant. For every $z \in \C, \phi \in X^*$, $\phi(f(z)) = \phi(f(0))$. Since $X^*$ separates points of $X$, $f(z) = f(0)$.
\end{proof}

\begin{rmk}
  This is a typical example of how to transfer a scalar result to a vector-valued one: Prove the result once composed with any functional, then go back using Hahn-Banach.
\end{rmk}

\subsection{Locally convex spaces}

\begin{dfn*}
  A {\bf locally convex space} is a $\K$-vector space such that there exists a family $\mcP$ of seminorms on $X$ that separate points of $X$ in the sense that for all $x \ne 0$ there exists $p \in \mcP$ such that $p(x) \ne 0$.
\end{dfn*}

The family $\mcP$ defines a topology on $X$:
$$U \subseteq X \text{ open} \iff \for x \in U, \exists s \subseteq \mcP \text{ finite}, \eps > 0, \{y \in X \mid \for p \in s, p(x) < \eps\} \subseteq U$$

\begin{rmks}~

  \begin{enumerate}
    \item Addition and scalar multiplication are continuous.
    \item The topology is Hausdorff as $\mcP$ separates points.
    \item $x_n \to x \iff \for p \in \mcP, p(x_n - x) \to 0$
    \item Let $Y$ be a subspace of $X$ and $\mcP_Y = \{p\restriction_Y \mid p \in \mcP\}$. Then $(Y, \mcP_Y)$ is a LCS and its topology is the subspace topology.
    \item Let $\mcP, \mcQ$ be two families of seminorms on $X$ both separating points of $X$. We say $\mcP, \mcQ$ are {\bf equivalent}, write $\mcP \sim \mcQ$, if they induce the same topology on $X$. One interesting result is that
    $$ (X, \mcP) \text{ metrisable} \iff \mcP \text{ equivalent to some countable family}$$
    \item We make $\mcP$ part of the data here out of simplicity, but in grown up mathematics we instead assume that $X$ already comes with a topology and that this topology coincides with the one induced by $\mcP$.
  \end{enumerate}
\end{rmks}

\begin{dfn*}
  A {\bf Fréchet space} is a complete metrisable LCS.
\end{dfn*}

\begin{egs*}~

  \begin{enumerate}
    \item A normed space is a LCS with $\mcP = \{\norm\cdot\}$.
    \item Let $U \subseteq \C$ nonempty open. Let $\mcO(U) = \{f : U \to \C \mid f \text{ holomorphic}\}$. For compact $K \subseteq U$, define $p_K(f) = \sup_{z \in K} \abs{f(z)}$. Let
    $\mcP = \{p_K \mid K \subseteq U \text{ compact}\}$
    Then $(\mcO(U), \mcP)$ is a LCS. If we replace $\{K \subseteq U \text{ compact}\}$ by a compact exhaustion of $U$, then we get a countable separating family equivalent to $\mcP$. So $(\mcO(U), \mcP)$ is metrisable. However it is not normable: no norm on $\mcO(U)$ induces the topology of $(\mcO(U), \mcP)$, which is the topology of uniform convergence. This is a consequence of Montel's theorem.
    \item Fix $d \in \N, \Omega \subseteq \R^d$ a nonempty open set. Let
    $$C^\infty(\Omega) = \{f : \Omega \to \R \mid f \text{ infinitely differentiable}\}$$
    Given a multi-index $\alpha \in \Z^d$, $\alpha$ defines a differential operator
    $$D^\alpha = \left(\frac\partial{\partial x_1}\right)^{\alpha_1} \dots
    \left(\frac\partial{\partial x_d}\right)^{\alpha_d}$$
    For a compact $K \subseteq \Omega, \alpha \in \Z^d$, define
    $$p_{K, \alpha}(f) = \sup_{z \in K}\abs{D^\alpha f(z)}$$
    Let
    $$\mcP = \{p_{K, \alpha} \mid K \subseteq U \text{ compact}, \alpha \in \Z^d\}$$
    Then $(C^\infty, \mcP)$ is a LCS. It is in fact a non-normable Fréchet space.
  \end{enumerate}
\end{egs*}

\begin{lem}\label{lem:lcs-continuity}
  Let $(X, \mcP), (Y, \mcQ)$ be LCS, $T : X \to Y$ linear. TFAE
  \begin{enumerate}
    \item $T$ is continuous
    \item $T$ is continuous at $0$
    \item $\for q \in \mcQ, \exists s \subseteq \mcP \text{ finite}, C \ge 0, \for x \in X, q(Tx) \le C\max_{p \in s} p(x)$
  \end{enumerate}
\end{lem}
\begin{proof}~\\
  {\bf (i) $\iff$ (ii)} \\
  Translation is continuous.

  {\bf (ii) $\imp$ (iii)} \\
  Given $q \in \mcQ$, let $V = \{y \in Y \mid q(y) \le 1\}$. Then $V$ is a neighborhood of $0$ in $Y$. So there exists $U$ neighborhood of $0$ in $X$ such that $T(U) \subseteq V$. WLOG
  $$U = \{x \in X \mid \for p_K \in s, p_K(x) \le \eps\}$$
  Let $p = \max_{p_K \in s} p_K(x)$. If $p(x) = 1$, then $p(\eps x) = \eps$, so $\eps x \in U$ and
  $$q(T(\eps x)) \le 1$$
  By homogeneity, $q(Tx) \le \frac 1\eps p(x)$ for all $x$ such that $p(x) > 0$. If $p(x) = 0$, then $p(\lambda x) = 0$ for all scalar $\lambda$. So $q(T(\lambda x)) \le 1$ for all $\lambda$. Hence $q(Tx) = 0 \le \frac 1\eps p(x)$.

  {\bf (iii) $\imp$ (ii)} \\
  Assume $t \subseteq \mcQ$ is finite, $\eps > 0$, and let $V = \{y \in Y \mid \for q \in t, q(y) \le \eps$ the corresponding neighborhood of $0$. For each $q \in t$, find $s_q \subseteq \mcP$ finite and $C_q$ so that $\for x \in X, q(Tx) \le C_q\max_{p \in s_q}p(x)$. Let
  $$U = \left\{x \in X \mid \for q \in \mcQ, p \in s_q, p(x) \le \frac\eps{C_q}\right\}$$
  Then $U$ is a neighborhood of $0$ and $T(U) \subseteq V$.
\end{proof}

\begin{dfn*}
  Let $(X, \mcP)$ be a LCS. The {\bf dual space} of $X$ is the space of continuous linear functionals $X \to \K$.
\end{dfn*}

\newlec

\begin{lem}\label{lem:lcs-dual}
  Let $f$ be a linear functional on a LCS $(X, \mcP)$. Then
  $$f \in X^* \iff \ker f \text{ closed}$$
\end{lem}
\begin{proof}~\\
  $\imp$ \\
  $\ker f = f^{-1}(0)$ is closed since $f$ is continuous.
  
  $\impliedby$ \\
  If $\ker f = 0$, then $f = 0$ is continuous. Else fix some $x_0 \nin \ker f$. Since $(\ker f)^c$ is open, find $s \subseteq \mcP$ finite, $\eps > 0$ such that
  $$\underbrace{\{x \in X \mid \for p \in s, p(x - x_0) < \eps\}}_U \subseteq (\ker f)^c$$
  Then $U$ is a neighborhood of $0$ and $(x_0 + U) \cap \ker f = \empty$. Note that $U$ is convex and {\bf balanced} ($x \in U, \abs\lambda \le 1 \imp \lambda x \in U$), hence so is $f(U)$ as $f$ is linear. \\
  If $f(U)$ is unbounded, then it is the whole scalar field, hence so is $f(x_0 + U) = f(x_0) + f(U)$. But $0 \in \ker f$, contradicting disjointness. \\
  So find $M$ such that $\abs{f(x)} < M$ for all $x \in U$. For all $\delta > 0$, $\frac\delta M U$ is a neighborhood of $0$ and $f\left(\frac\delta M U\right) \subseteq \{\lambda \in \K \mid \abs\lambda < \delta\}$. Thus $f$ is continuous.
\end{proof}

\begin{thm}[Hahn-Banach]\label{thm:hb-lcs}
  Let $(X, \mcP)$ be a LCS.
  \begin{enumerate}
    \item Given a subspace $Y$ of $X$ and $g \in Y^*$, there exists $f \in X^*$ such that $f\restriction_Y = g$.
    \item Given a closed subspace $Y$ of $X$ and $x_0 \nin Y$, there exists $f \in X^*$ such that $f\restriction_Y = 0$, $f(x_0) \ne 0$.
  \end{enumerate}
\end{thm}
\begin{rmk}
  This means that $X^*$ separates points of $X$.
\end{rmk}
\begin{proof}~
  \begin{enumerate}
    \item By Lemma \ref{lem:lcs-continuity}, find $s \subseteq \mcP$ finite, $C \ge 0$ such that
    $$\for y \in Y, \abs{g(y)} \le C\max_{p \in s}p(y)$$
    Let $p(x) = C\max_{p \in s}p(x)$. Then $p$ is a seminorm on $X$ and $\for y \in Y, \abs{g(y)} \le p(y)$. By Theorem \ref{thm:hb-absolute}, find a linear functional $f$ on $X$ such that $f\restriction_Y = g, \for x \in X, \abs{f(x)} \le p(x)$. By Lemma \ref{lem:lcs-continuity}, $f \in X^*$.
    \item Let $Z = \Span(Y \cup \{x_0\})$ and define a linear functional $g$ on $Z$ by
    $$g(y + \lambda x_0) = \lambda, y \in Y, \lambda \in \K$$
    Then $g\restriction_Y = 0, g(x_0) = 1 \ne 0$ and $\ker g = Y$ is closed, so $g \in Z^*$ by Lemma \ref{lem:lcs-dual}. By part (i), find $f \in X^*$ such that $f\restriction_Z = g$. This works.
  \end{enumerate}
\end{proof}

\clearpage

\section{The dual of \texorpdfstring{$L_p(\mu)$ and $C(K)$}{Lp(mu) or C(K)}}

Let $(\Omega, \mcF, \mu)$ be measure space.

$1 \le p < \infty$
$$L_p(\mu) = \{f : \Omega \to \K \mid f \text{ measurable and } \int_\Omega \abs f^p d\mu < \infty\}$$
This is a normed space in the $L_p$-norm:
$$\norm f_p = \left(\int_\Omega \abs f^p d\mu\right)^{\frac 1p}$$

$p = \infty$ \\
A measurable function $f : \Omega \to \K$ is {\bf essentially bounded} if there exists $N \in \mcF$ such that $\mu(N) = 0$ and $f\restriction_{N^c}$ is bounded.
$$L_p(\mu) = \{f : \Omega \to \K \mid f \text{ measurable and essentially bounded}\}$$
This is a normed space in the $L_\infty$-norm:
$$\norm f_\infty = \esssup \abs f = \inf_{\abs f \le k \text{ ae}} k$$
The inf is attained: there exists some $N \in \mcF, \mu(N) = 0$ such that $\norm f_\infty = \sup_{N^c} \abs f$.

In all cases, we identify functions up to almost everywhere equality.

\begin{thm}
  $L_p(\mu)$ is complete for $1 \le p \le infty$.
\end{thm}

\begin{dfn*}[Complex measures]
  A {\bf complex measure} on $\mcF$ is a countably additive set function $\nu : \mcF \to \C$. \\
  The {\bf total variation measure} $\abs\nu$ is defined by
  $$\abs\nu(A) = \sup_{\substack{A_1, \dots, A_n \text{ measurable} \\ \text{partition of } A}} \sum_k \abs{\nu(A_k)}$$
  $\abs\nu : \mcF \to [0, \infty]$ is a positive measure. Later we'll see that $\abs\nu$ is a finite measure. \\
  The {\bf total variation} of $\nu$ is $\norm\nu_1 = \abs\nu(\Omega)$.
\end{dfn*}

\begin{prop}
  If $\nu$ is a complex measure on $\mcF$ and $A_n \in \mcF$ for all $n$, then
  \begin{itemize}
    \item If $A$ is monotone, then $\nu(\Union_n A_n) = \lim_{n \to \infty} \nu(A_n)$.
    \item If $A$ is antitone, then $\nu(\bigcap_n A_n) = \lim_{n \to \infty} \nu(A_n)$.
  \end{itemize}
\end{prop}

\begin{dfn*}[Signed measures]
  A {\bf signed measure} on $\mcF$ is a countably additive set function $\nu : \mcF \to \R$.
\end{dfn*}

\begin{thm}\label{thm:hahn-decomp}
  If $\nu$ is a signed measure, then there exists a measurable partition $\Omega = P \cup N$ such that for all $A \in \mcF$
  \begin{align*}
    A \subseteq P & \imp \nu(A) \ge 0 \\
    A \subseteq N & \imp \nu(A) \le 0
  \end{align*}
\end{thm}
\begin{rmks}~
  
  \begin{enumerate}
    \item This decomposition is called the {\bf Hahn decomposition} of $\nu$.
    \item Define $\nu^+(A) = \nu(A \cap P), \nu^-(A) = -\nu(A \cap N)$. Then $\nu^+, \nu^-$ are finite positive measures such that $\nu = \nu^+ - \nu^-$. This determines $\nu^+, \nu^-$ uniquely and the decomposition composition $\nu = \nu^+ - \nu^-$ is called the {\bf Jordan decomposition} of $\nu$.
    \item If $\nu$ is a complex measure on $\mcF$, then $\Re\nu, \Im\nu$ are signed measures with Jordan decomposition $\nu_1 - \nu_2, \nu_3 - \nu_4$ respectively. Hence $\nu = \nu_1 - \nu_2 + i\nu_3 - i\nu_4$ is the Jordan decomposition of $\nu$.
    $$\nu_1, \nu_2, \nu_3, \nu_4 \le \abs\nu \le \nu_1 + \nu_2 + \nu_3 + \nu_4$$
    So $\abs\nu$ is a finite measure.
  \end{enumerate}
\end{rmks}
\begin{proof}[Sketch]
  Define $\nu^+(A) = \sup_{\substack{B \in \mcF \\ B \subseteq A}} \nu(B)$. $\nu^+$ is nonnegative and finitely additive.

  {\bf Key step: $\nu^+(\Omega) < \infty$} \\
  By contradiction, construct inductively sequences $A_n, B_n$ such that
  $$\nu^+(A_n) = \infty, \nu(B_n) > n$$
  by taking $A_0 = \Omega, B_{n + 1} \subseteq A_n$ such that $\nu(B_n) > n$ (exists by continuity) and $A_{n + 1} = B_{n + 1}$ or $A_n \setminus B_{n + 1}$. This contradicts countable additivity.

  Now find a sequence $A_n$ such that $\nu(A_n) > \nu^+(\Omega) - 2^{-n}$ and set $P = \liminf_n A_n, N = P^c$. Check that this works.
\end{proof}

\newlec

\begin{dfn*}[Absolute continuity]
  Let $(\Omega, \mcF, \mu)$ be a measure space and $\nu : \mcF \to \C$ a complex measure. $\nu$ is {\bf absolutely continuous} with respect to $\mu$, written $\nu \ll \mu$, if $\for A \in \mcF, \mu(A) = 0 \imp \nu(A) = 0$.
\end{dfn*}

\begin{rmks}~
  \begin{itemize}
    \item $\nu \ll \mu \imp \abs\nu \ll \mu$, so if $\nu$ has Jordan decomposition $\nu = \nu_1 - \nu_2 + i\nu_3 - i\nu_4$ then $\nu_1, \nu_2, \nu_3, \nu_4 \ll \mu$.
    \item If $\nu \ll \mu$, then $\for \eps > 0, \exists \delta > 0, \for A \in \mcF, \mu(A) < \delta \imp \abs{\nu(A)} < \eps$.
  \end{itemize}
\end{rmks}

\begin{eg*}
  Let $f \in L_1(\mu)$. Define $\nu(A) = \int_A f d\mu$ for $A \in \mcF$. By Dominated Convergence, $\nu$ is a complex measure and $\mu(A) = 0 \imp \nu(A) = 0$. So $\nu \ll \mu$.
\end{eg*}

\begin{dfn*}
  $A \in \mcF$ is {\bf $\sigma$-finite} if there exists $A_n$ with $\mu(A_n) < \infty$ such that $A = \Union_n A_n$. Say $\mu$ is {\bf $\sigma$-finite} if $\Omega$ is $\sigma$-finite.
\end{dfn*}

\begin{thm}[Radon-Nikodym]\label{thm:rn}
  Let $\mu$ be a $\sigma$-finite measure and $\nu$ a complex measure such that $\nu \ll \mu$. Then there exists a unique $f \in L_1(\mu)$ such that, for all $A \in \mcF$, $\nu(A) = \int_A f d\mu$. Moreover, $f$ takes values in $\C$/$\R$/$\R^+$ depending on where $\nu$ is valued.
\end{thm}
\begin{proof}~\\
  {\bf Uniqueness} \\
  standard

  {\bf Existence} \\
  $\nu$ is a finite measure (by the Jordan decomposition). WLOG $\mu$ is a finite measure (by $\sigma$-finiteness). Let
  $$\mcH = \curlybrack{h : \Omega \to \R^+ \,\middle|\, h \text{ integrable}, \for A \in \mcF, \int_A h d\mu \ll \nu(A)}$$
  $\mcH \ne \emptyset$ (eg $0 \in \mcH$). Let $\alpha = \sup_{h \in \mcH} \int_\Omega h d\mu$. We see $0 \le \alpha \le \nu(\Omega)$.

  {\bf Claim} \\
  There exists $f \in \mcH$ such that $\alpha = \int_\Omega f d\mu$.

  {\bf Idea} \\
  If $\int_A fd\mu < \nu(A)$, then $f + \frac 1 n 1_A \in \mcH$ (morally, not literally), contradicting the definition of $\alpha$.

  Pick that $f$. Define $\nu_n(A) = \nu(A) - \int_A fd\mu - \frac 1n \mu(A)$. $\nu_n$ has Hahn decomposition $\Omega = P_n \cup N_n$. Then $f + \frac 1n P_n \in \mcH$. By definition of $\alpha$, $\mu(P_n) = 0$. Since $\nu \ll \mu$, $\nu(P_n) = 0$. Let $P = \Union_n P_n, N = \Inter_n N_n$. Then $\Omega = P \cup N, \mu(P) = \nu(P) = 0$. For $A \in \mcF$,
  \begin{align*}
    A \subseteq P & \imp \int_A fd\mu = \nu(A) = 0 \\
    A \subseteq N & \imp \for n, \nu_n \le 0 \imp \int_A fd\mu \ge \nu(A)
  \end{align*}
\end{proof}

\begin{rmks}~
  \begin{itemize}
    \item Without assuming $\nu \ll \mu$, the proof shows there is a decomposition $\nu = \nu_1 + \nu_2$ where $\nu_1(A) = \int_A fd\mu$ and $\nu_2 \perp \mu$ ({\bf orthogonal}, ie there exists a measurable decomposition $\Omega = P \cup N$ such that $\mu(P) = 0, \abs{\nu_2}(N) = 0$). $\nu = \nu_1 + \nu_2$ is the {\bf Lebesgue decomposition} of $\nu$.
    \item The unique $f$ in Theorem \ref{thm:rn} is the {\bf Radon-Nikodym derivative} of $\nu$ with respect to $\mu$, denoted $\frac{d\nu}{d\mu}$. The result says
    $$\nu(A) = \int_\Omega 1_A d\nu = \int_A \frac{d\nu}{d\mu} d\mu = \int_\Omega 1_A \frac{d\nu}{d\mu} d\mu$$
    Hence a measurable function $g$ is $\nu$-integrable iff $g\frac{d\nu}{d\mu}$ is $\mu$-integrable and then
    $$\int_\Omega f d\nu = \int_\Omega g \frac{d\nu}{d\mu} d\mu$$
  \end{itemize}
\end{rmks}

\subsection{Dual space of \texorpdfstring{$L_p(\mu)$}{Lp(mu)}}

Let $(\Omega, \mcF, \mu)$ be a measure space, $1 \le p < \infty, 1 < q \le \infty$ such that $p^{-1} + q^{-1} = 1$. For $g \in L_q$, define $\phi_g : L_p \to \K$ by $\phi_g(f) = \int_\Omega fg d\mu$. By Hölder, $fg \in L_1$, and $\abs{\phi_g(f)} \le \norm f_p \norm g_q$. So $\phi_g$ is well-defined, linear and bounded with $\norm{\phi_g} \le \norm g_q$. Hence $\phi_g \in L_p^*$ and $\phi : L_q \to L_p^*$ is linear and bounded with $\norm\phi \le 1$.

\begin{thm}\label{thm:lp-dual}~
  \begin{enumerate}
    \item If $1 < p < \infty$, then $\phi$ is an isometric isomorphism. So $L_p^* \cong L_q$.
    \item If $p = 1$ and $\mu$ is $\sigma$-finite, then $\phi$ is an isometric isomorphism. So $L_1^* \cong L_\infty$.
  \end{enumerate}
\end{thm}
\begin{proof}~
  \begin{enumerate}
    \item {\bf $\phi$ is isometric} \\
    Let $g \in L_1$. We know $\norm{\phi_g} \le \norm g_q$. Let $\lambda$ be a measurable function with $\abs\lambda = 1, \lambda g = \abs g$. let $f = \lambda\abs g^{q - 1}$. Then
    $$\norm f_p^p = \int_\Omega \abs g^{pq - p} d\mu = \int_\Omega \abs g^q d\mu = \norm g_q^q$$
    So $f \in L_p$ and $\norm f_p = \norm g_q^{\frac qp}$. Then
    $$\norm q_q^{\frac qp}\norm{\phi_g} \ge \abs{\phi_g(f)} = \int_\Omega \abs g^q d\mu = \norm g_q^q$$
    So $\norm{\phi_g} \ge \norm g_q^{q - \frac qp} = \norm g_q$.

    {\bf $\phi$ is onto} \\
    Fix $\psi \in L_p^*$. We seek $g \in L_q$ such that $\psi = \phi_g$. Idea: We want $\psi(1_A) = \int_A g d\mu$.

    {\bf Case 1: $\mu$ is finite} \\
    For $A \in \mcF$, $1_A \in L_p$, so define $\nu(A) = \psi(1_A)$. $\nu(\empty) = 0$ and, if $A = \Union_p A_n \in \mcF$, then $\sum_k 1_{A_k} = 1_A$ in $L_p$, so
    $$\sum_k \nu(A_k) = \sum_k \psi(1_{A_k}) = \psi(1_A)$$
    Hence $\nu$ is a complex measure. \\
    If $A \in \mcF, \mu(A) = 0$, then $1_A = 0$ ae in $L_p$, so $\nu(A) = \psi(1_A) = 0$. Hence $\nu \ll \mu$. \\
    By Theorem \ref{thm:rn}, find $g \in L_1$ such that $\for A \in \mcF, \nu(A) = \int_A g d\mu$. Hence
    \begin{align*}
      \psi(1_A) & = \int_\Omega 1_A g d\mu \text{ for all } A \in \mcF \\
      \psi(f) & = \int_\Omega fg d\mu \text{ for all simple function } f
    \end{align*}
    Given $f \in L_\infty$, find simple functions $f_n$ tending to $f$ in $L_\infty$. So $\psi(f_n) \to \psi(f)$ and $f_ng \to fg$ (by Hölder for $\infty, 1$), meaning that
    $$\psi(f) = \int_\Omega fg d\mu \text{ for all } f \in L_\infty$$
    For $n \in \N$, let $A = \{\abs g \le n\}$ and $f_n = \lambda 1_{A_n} \abs g^{q - 1}$ where $\abs\lambda = 1, \lambda g = \abs g$. As $f_n \in L_\infty$,
    $$\int_\Omega f_ng d\mu = \int_{A_n} \abs g^q d\mu = \psi(f_n)$$
    So $\left(\int_A \abs g^q d\mu\right)^{q^{-1}} \le \norm\psi$. By Monotone Convergence, $g \in L_q$. \\
    Given $f \in L_p$, find simple functions $f_n$ tending to $f$ in $L_p$. So $\psi(f_n) \to \psi(f)$ and $f_ng \to fg$ in $L_1$ (by Hölder for $p, q$). Hence $\psi(f) = \int_\Omega fg d\mu$, as wanted.

    Before going onto Case 2, for $A \in \mcF$, let $\mcF_A = \{B \in \mcF \mid B \subseteq A\}$ and $\mu_A = \mu\restriction_{\mcF_A}$ so that $(A, \mcF_A, \mu_A)$ is a measure space. Then $L_p(\mu_A) \subseteq L_p(\mu)$ (by extending $f \in L_p(\mu_A)$ by $0$ outside $A$). Let $\psi_A = \psi\restriction_{L_p(\mu_A)}$.

    \newlec

    \begin{claim}
      If $A, B \in \mcF$ are disjoint, then
      $$\norm{\psi_{A \cup B}} = \left(\norm{\psi_A}^q + \norm{\psi_B}^q\right)^{\frac 1q}$$
    \end{claim}
    \begin{proof}
      \begin{align*}
        \left(\norm{\psi_A}^q + \norm{\psi_B}^q\right)^{\frac 1q}
        & = \sup_{\substack{a, b \ge 0 \\ a^p + b^p \le 1}} a\norm{\psi_A} + b\norm{\psi_B} \\
        & = \sup_{\substack{a, b \ge 0 \\ a^p + b^p \le 1 \\ f \in B_{L_p(\mu_A)} \\ g \in B_{L_p(\mu_B)}}} a\abs{\psi_A(f)} + b\abs{\psi_B(g)} \\
        & = \sup_{\substack{\abs a^p + \abs b^p \le 1 \\ f \in B_{L_p(\mu_A)} \\ g \in B_{L_p(\mu_B)}}} \underbrace{\abs{a\psi_A(f) + b\psi_B(g)}}_{\psi_{A \cup B}(af + bg)} \\
        & = \sup_{h \in L_p(\mu_{A \cup B})} \abs{\psi_{A \cup B}(h)} \\
        & = \norm{\psi_{A \cup B}}
      \end{align*}
    \end{proof}

    {\bf Case 2: $\mu$ is $\sigma$-finite} \\
    Find a measurable partition $\Omega = \Union_n A_n$ such that $\mu(A_n) < \infty$. By Case 1, find, for each $n$, $g_n \in L_q(A_n)$ such that $\psi_{A_n} = \phi_{g_n}$, ie
    $$\psi(f) = \int_{A_n} fg_n d\mu \text{ for all } f \in L_q(\mu_{A_n})$$
    If we define $g$ on $\Omega$ by $g = g_n$ on $A_n$, then $g \in L_q$ and
    $$\psi(f) = \phi_g(f) \text{ for all } f \in L_p(\mu_{A_n})$$
    Hence $\psi = \phi_g$ on $\bar{\Span}\Union_n L_p(\mu_{A_n}) = L_p(\mu)$.

    {\bf Case 3: General $n$} \\
    First observe that, for $f \in L_p(\mu)$, $\{f \ne 0\}$ is $\sigma$-finite. Indeed,
    $$\{f \ne 0\} = \Union_n \curlybrack{\frac 1n < \abs f}$$
    and
    $$\mu\curlybrack{\frac 1n < \abs f} \le \abs{n ^ p} \norm f_p^p < \infty \text{ by Markov}$$
    Choose $f_n \in B_{L_p}$ such that $\psi(f_n) \to \norm\psi$. Then $A = \Union_n\{f_n \ne 0\}$ is $\sigma$-finite and $\norm{\psi_A} = \norm\psi$. By the claim,
    $$\norm\psi = \left(\norm{\psi_A}^q + \norm{\psi_{A^c}}^q\right)^{\frac 1q}$$
    So $\Psi_{A^c} = 0$. By Case 2, find $g \in L_q(\mu_A) \subseteq L_q(\mu)$ such that $\psi_A = \phi_g$, so that
    $$\psi(f) = \psi_A{f\restriction_A} + \psi{A^c}(f\restriction{A^c}) = \int_A fg d\mu + 0 = \int_\Omega fg d\mu$$
    \item {\bf $p = 1, \mu$ is $\sigma$-finite} \\
    {\bf $\phi$ is isometric} \\
    Let $g \in L_\infty$. We know $\norm{\phi_g} \le \norm g_\infty$ (by Hölder) Fix $s < \norm g_\infty$. Then $\mu\{s < \abs g\} > 0$. Since $\mu$ is $\sigma$-finite, find $A \subseteq \{s < \abs g\}$ such that $0 < \mu(A) < \infty$. Choose a measurable function $\lambda$ such that $\abs\lambda = 1, \lambda g = \abs g$. Then $\lambda 1_A \in L_1, \norm{\lambda 1_A}_1 = \mu(A)$. Now,
    $$\mu(A)\norm{\phi_g} \ge \abs{\phi_g(\lambda 1_A)} = \int_A \abs g d\mu \ge s\mu(A)$$
    So $\norm{\phi_g} \ge s$. Taking the sup, $\norm{\phi_g} \ge \norm g_\infty$.

    {\bf $\phi$ is onto} \\
    Fix $\psi \in L_q^*$. We seek $g \in L_\infty$ such that $\psi = \phi_g$.

    {\bf Case 1: $\mu$ is finite} \\
    Define $\nu(A) = \psi(1_A)$ for all $A \in \mcF$. Follow the same steps as for $1 < p < \infty$.

    {\bf Case 2: $\mu$ is $\sigma$-finite} \\
    This time, prove that
    $$\norm{\psi_{A \cup B}} = \max(\norm{\psi_A}, \norm{\psi_B})$$
    for all $A, B \in \mcF$ disjoint and proceed as before.
  \end{enumerate}
\end{proof}

\begin{cor}
  For $1 < p < \infty$, $L_p(\mu)$ is reflexive.
\end{cor}
\begin{proof}
  Let $\psi \in L_p^{**}$. Then $g \mapsto \inn{\phi_g}\psi : L_q \to \K$ is in $L_q^*$. By Theorem \ref{thm:lp-dual}.i, find $f \in L_p$ such that
  $$\inn{\phi_g}\psi = \int_\Omega fg d\mu = \inn f{\phi_g} = \inn{\phi_g}{\hat f}$$
  Since $L_p^* = \{\phi_g \mid g \in L_q\}$, this proves $\psi = \hat f$.
\end{proof}

\clearpage

\subsection{Dual space of \texorpdfstring{$C(K)$}{C(K)}}

Throughout, $K$ will be a compact Hausdorff topological space. Define
\begin{align*}
  C(K) & = \{f : K \to \C \mid f \text{ continuous}\} \\
  C^\R(K) & = \{f : K \to \R \mid f \text{ continuous}\} \\
  C^+(K) & = \{f : K \to \R^+ \mid f \text{ continuous}\} \\
  M(K) & = C(K)^* \\
  M^\R(K) & = \{\phi \in M(K) \mid \for f \in C^\R(K), \phi(f) \in \R\} \\
  M^+(K) & = \{\phi : C(K) \to \C \mid \phi \text{ is $\R$-linear}, \for f \in C^+(K), 0 \le \phi(f) \in \R\}
\end{align*}

$C(K), C^\R(K)$ are complex/real Banach spaces in the sup norm: $\norm f_\infty = \sup_K \abs f$. $M(K)$ is a complex Banach space in the operator norm. $M^\R(K)$ is a closed real-linear subspace of $M(k)$. Elements of $M^+(K)$ are called {\bf positive linear functionals}.

\begin{aim}
  Identify $M(K), M^\R(K)$.
\end{aim}

\newlec

The next lemma tells us that it's enough to understand $M^+(K)$.

\begin{lem}\label{lem:m-real-pos}~
  \begin{enumerate}
    \item For all $\phi \in M(K)$, there are unique $\phi_1, \phi_2 \in M^\R(K)$ such that $\phi = \phi_1 + i\phi_2$.
    \item $\phi \mapsto \phi\restriction_{C^\R(K)} : M^\R(K) \to C^\R(K)^*$ is an isometric isomorphism.
    \item $M^+(K) \subseteq M(K)$ and $M^+(K) = \{\phi \in M(K) \mid \norm\phi = \phi(1)\}$
    \item For all $\phi \in M^\R(K)$, there are unique $\phi^+, \phi^- \in M^+(K)$ such that $\phi = \phi^+ - \phi^-$ and $\norm\phi = \norm{\phi^+} + \norm{\phi^-}$.
  \end{enumerate}
\end{lem}
\begin{proof}~
  \begin{enumerate}
    \item Let $\phi \in M(K)$. Then $\bar\phi$ sending $f \mapsto \phi(\bar f)$ is in $M(K)$ as well and $\phi \in M^\R(K) \iff \bar\phi = \phi$.

    {\bf Uniqueness} \\
    Assume $\phi = \phi_1 + i\phi_2$ where $\phi_1, \phi_2 \in M^\R(K)$. Then $\bar\phi = \phi_1 - i\phi_2$, so
    $$\phi_1 = \frac{\phi + \bar\phi}2, \phi_2 = \frac{\phi - \bar\phi}{2i}$$

    {\bf Existence} \\
    Check that the above works
    \item Let $\phi \in M^\R(K)$. We show $\norm{\phi\restriction_{C^\R(K)}} = \norm\phi$. Clearly, $\norm{\phi\restriction_{C^\R(K)}} \le \norm\phi$. Let $f \in B_{C(K)}$. Choose $\lambda \in \C, \abs\lambda = 1, \lambda\phi(f) = \abs{\phi(f)}$, so that
    \begin{align*}
      \abs{\phi(f)}
      & = \lambda\phi(f) \\
      & = \phi(\lambda f) \\
      & = \phi(\Re(\lambda f)) + \cancelto 0{\phi(\Im(\lambda f))} \\
      & \le \norm{\phi\restriction_{C^\R(K)}} \norm{\Re(\lambda f)}_\infty \\
      & \le \norm{\phi\restriction_{C^\R(K)}}
    \end{align*}
    Hence $\norm\phi \le \norm{\phi\restriction_{C^\R(K)}}$. \\
    Finally, given $\psi \in C^\R(K)$, define $\phi(f) = \psi(\Re f) + i\psi(\Im f)$. Then $\phi \in M(K)$ and $\phi\restriction_{C^\R(K)} = \psi$.
    \item {\bf $M^+(K) \subseteq M(K)$} \\
    Let $\phi \in M^+(K)$. For $f \in B_{C^\R(K)}$, we have $1 \pm f \ge 0$, so $\phi(1 \pm f \ge 0$. Hence $\phi(f) \in \R$ and $\abs{\phi(f)} \le \phi(1)$. So $\phi\restriction_{C^\R(K)} \in C^\R(K)^*$ and $\norm{\phi\restriction_{C^\R(K)}} = \phi(1)$. By (ii), $\phi \in M(K), \norm\phi = \phi(1)$.

    {\bf $M^+(K) = \{\phi \in M(K) \mid \norm\phi = \phi(1)$} \\
    We have already checked one inclusion. Let $\phi \in M(K)$ with $\norm\phi = \phi(1)$. WLOG $\norm\phi = \phi(1) = 1$. Let $f \in B_{C^\R(K)}$ and write $\phi(f) = a + ib$ where $a, b \in \R$. We want $b = 0$. For $t \in \R$,
    \begin{align*}
      \abs{\phi(f + it)}^2
      & = a^2 + (b + t)^2 = a^2 + b^2 + t^2 + 2bt \\
      & \le \norm{f + it}_\infty \le 1 + t^2
    \end{align*}
    So $b = 0$. \\
    Given $f \in C^+(K)$ with $0 \le f \le 1$, we have $-1 \le 2f - 1 \le 1$, so $\abs{\phi(2f - 1)} \le \norm{2f - 1}_\infty \le 1$, ie $-1 \le 2\phi(f) - 1 \le 1$. So $\phi(f) \ge 0$.
    \item Let $\phi \in M^\R(K)$. Assume for a moment that $\phi = \psi_1 - \psi_2$ where $\psi_1, \psi_2 \in M^+(K)$. For $f, g \in C^+(K)$ with $0 \le g \le f$, we have $\psi_1(f) \ge \psi_1(g) = \phi(g) + \psi_2(g) \ge \phi(g)$. So
    $$\psi_1(f) \ge \sup_{0 \le g \le f} \phi(g)$$
    For $f \in C^+(K)$, define
    $$\phi^+(f) = \sup_{0 \le g \le f} \phi(g)$$
    Observe that $\phi^+ \ge 0, \phi^+(f) \le \norm\phi \norm f_\infty, \phi^+(f) \ge \phi(f), \phi^+$ is linear. \\
    Next, for $f \in C^\R(K)$, write $f = f_1 - f_2$ where $f_1, f_2 \in C^+(K)$ and define $\phi^+(f) = \phi^+(f_1) - \phi^+(f_2)$. This is well-defined and $\R$-linear. Then $\phi$ is $\C$-linear since $\phi^+(f) \ge 0$. For all $f \in C^+(K)$ and $\phi^+ \in M^+(K)$. \\
    Define $\phi^- = \phi^+ - \phi$. For $f \in C^+(K)$, $\phi^+(f) \ge \phi(f)$, so $\phi^-(f) \ge 0$, namely $\phi^- \in M^+(K)$. \\
    We now see that $\norm\phi \le \norm{\phi^+} + \norm{\phi^-}$.
    Given $f \in C^+(K), 0 \le f \le 1$, we have $-1 \le 2f - 1 \le 1$, so
    $$2\phi(f) - \phi(1) = \phi(2f - 1) \le \norm\phi$$
    Taking the sup over $f$, we thus check that
    $$\norm{\phi^+} + \norm{\phi^-} = \phi^+(1) + \phi^-(1) = 2\phi^+(1) - \phi(1) \le \norm\phi$$

    {\bf Uniqueness} \\
    Assume $\phi = \psi_1 - \psi_2, \psi_1, \psi_2 \in M^+(K), \norm\phi = \norm{\psi_1} + \norm{\psi_2}$. From the initial observation, $\psi_1 \ge \phi^+$, hence $\psi_2 = \psi_1 - \phi \ge \phi^+ - \phi = \phi^-$. Therefore $\psi_1 - \phi^+, \psi_2 - \phi^- \in M^+(K)$. By (iii),
    $$\norm{\psi_1 - \phi^+} + \norm{\psi_2 - \phi^-} = \psi_1(1) - \phi^+(1) + \psi_2(1) - \phi^-(1) = \norm\phi - \norm\phi = 0$$
    Hence $\psi_1 = \phi^+, \psi_2 = \phi^-$.
  \end{enumerate}
\end{proof}

{\bf Topological preliminaries}
\begin{enumerate}
  \item $K$ being compact Hausdorff, it is {\bf normal}: given disjoint closed sets $E, F$ in $K$, there are disjoint open sets $U, V$ such that $E \subseteq U, F \subseteq V$. Equivalently, given $E \subseteq U \subseteq K$, $E$, closed, $U$ open, there exists $V$ open such that $E \subseteq V \subseteq \bar V \subseteq U$.
  \item Urysohn says: given disjoint closed sets $E, F$, there is a continuous function $f : K \to [0, 1]$ such that $f = 0$ on $E$, $f = 1$ on $F$.
  \item Write $f \prec U$ to mean that $U$ is an open set, $f$ is continuous and $\supp f \subseteq U$. Write $E \prec f$ to mean that $E$ is closed, $f$ is continuous and $f = 1$ on $E$.
  \item Urysohn then becomes: Given $E \subseteq U$, there exists $f$ such that $E \prec f \prec U$.
\end{enumerate}

\begin{lem}\label{lem:partition-of-unity}
  Let $E$ closed, $U_1, \dots, U_n$ open such that $E \subseteq \Union_n U_n$. Then
  \begin{enumerate}
    \item There exist open sets $V_j$ such that $\bar{V_j} \subseteq U_j$ and $E \subseteq \Union_j V_j$.
    \item There exist $f_j \prec U_j$ such that $0 \le \sum_j f_j \le 1$ and $\sum_j f_j = 1$ on $E$.
  \end{enumerate}
\end{lem}
\begin{proof}~
  \begin{enumerate}
    \item Induction on $n$:
    {\bf $n = 0$} \\
    Obvious.

    {\bf $n > 0$} \\
    $E \setminus U_n \subseteq \Union_{j < n} U_j$ so, by induction, find open sets $V_j$ such that $\bar{V_j} \subseteq U_j$ for all $j < n$ and $E \setminus U_n \subseteq \Union_{j < n} U_j$. So $\underbrace{E \setminus \Union_{j < n} V_j}_{\text{closed}} \subseteq \underbrace{U_n}_{\text{open}}$. By Urysohn, find an open $V_n$ such that
    $$E \setminus \Union_{j < n} V_j \subseteq V_n \subseteq \bar{V_n} \subseteq U_n$$
    \item Find the $V_j$ as in $(i)$ for $1 \le j \le n$ and by Urysohn find $h_j$ such that $\bar{V_j} \prec h_j \prec U_j$. By Urysohn again, find $h_0$ such that $\left(\Union_j U_j\right)^c \prec h_0 \prec E^c$. Let $h = \sum_{j = 0}^n h_j \ge 1$ and $f_j = \frac{h_j}h$ for $1 \le j \le n$. Then $0 \le \sum_{j = 1}^n \le 1$, $f_j \prec U_j$ and $\sum_{j = 1}^n f_j = 1$ on $E$.
  \end{enumerate}
\end{proof}

\begin{dfn*}[Borel measures]
  Let $X$ be a Hausdorff space and $\mcG$ its family of open sets. The {\bf Borel $\sigma$-algebra} is $\mcB := \sigma(\mcG)$, the $\sigma$-algebra generated by open sets. Elements of $\mcB$ are called {\bf Borel sets}. A {\bf Borel measure} on $X$ is a measure $\mu$ on $\mcB$. We say $\mu$ is {\bf regular} if
  \begin{enumerate}
    \item $\mu(E) < \infty$ for all compact $E \subseteq X$
    \item $\mu(A) = \inf_{\substack{U \text{ open} \\ A \subseteq U}} \mu(A)$ for all Borel set $A$
    \item $\mu(U) = \sup_{\substack{E \text{ compact} \\ E \subseteq U}} \mu(E)$ for all open $U$
  \end{enumerate}
  A complex Borel measure $\nu$ is {\bf regular} if $\abs\nu$ is regular.
\end{dfn*}

If $X$ is compact and $\mu$ is a Borel measure on $X$, then
\begin{align*}
  \mu \text{ regular}
  & \iff \mu(X) < \infty \text{ and } \for A \in \mcB, \mu(A) = \inf_{\substack{U \text{ open} \\ A \subseteq U}} \mu(U) \\
  & \iff \mu(X) < \infty \text{ and } \for A \in \mcB, \mu(A) = \sup_{\substack{E \text{ closed} \\ E \subseteq A}} \mu(E)
\end{align*}

\begin{dfn*}[Integration with respect to a complex measure]
  Let $\Omega$ be a set, $\mcF$ a $\sigma$-algebra on $\Omega$, $\nu$ a complex measure on $\mcF$. Write $\nu = \nu_1 - \nu_2 + i\nu_3 - i\nu_4$ the Jordan decomposition. Say a measurable function is {\bf $\nu$-integrable} if $f$ is $\abs\nu$-integrable, or equivalently if $f$ is $\nu_1, \nu_2, \nu_3, \nu_4$-integrable. Define
  $$\int_\Omega f d\nu = \int_\Omega f d\nu_1 - \int_\Omega f d\nu_2 + i\int_\Omega f d\nu_3 - i\int_\Omega f d\nu_4$$
\end{dfn*}

\newlec

\begin{prop}~
  \begin{enumerate}
    \item $\int_\Omega d\nu = \nu(A)$ for all $A \in \mcF$.
    \item Linearity: If $f, g : \Omega \to \C$ are $\nu$-integrable and $\lambda \in \C$, then
    $$\int_\Omega f + g d\nu = \int_\Omega fd\nu + \int_\Omega gd\nu, \int_\Omega \lambda fd\nu = \lambda \int_\Omega fd\nu$$
  \item Dominated Convergence: Let $f_n, f, g$ be measurable functions $\Omega \to \C$ such that $f_n \to f$ ae (with respect to $\abs\nu$), $g \in L_1$ and $\for n, f_n \le g$ ae. Then $f$ is $\nu$-integrable and $\int_\Omega f_n d\nu \to \int_\Omega f d\nu$
  \item $\abs{\int_\Omega f d\nu} \le \int_\Omega \abs f d\abs\nu$ for all $\nu$-integrable $f$. This is true for simple functions by properties 1 and 2. For general $f$, use Dominated Convergence.
  \end{enumerate}
\end{prop}

Let $\nu$ be a complex Borel measure on $K$. Then for $f \in C(K)$ we have
$$\int_K \abs f d\abs\nu \le \norm f_\infty \abs\nu(K) = \norm f_\infty \norm\nu_1$$
So $f$ is $\nu$-integrable. Define $\phi : C(K) \to \C$ by $\phi(f) = \int_\Omega f d\nu$. Then $\phi \in M(K)$ and $\norm\phi \le \norm\nu_1$. If $\nu$ is a signed measure, then $\phi \in M^\R(K)$. If $\nu$ is a positive measure, then $\phi \in M^+(K)$.

\begin{thm}[Riesz Representation Theorem]\label{thm:riesz-rep}
  For every $\phi \in M^+(K)$, there exists a unique regular Borel measure $\mu$ on $K$ that represents $\phi$: $\phi(f) = \int_K f d\mu$ for all $f \in C(K)$. Moreover, $\norm\phi = \mu(K) = \norm\mu_1$.
\end{thm}
\begin{proof}~\\
  {\bf Uniqueness} \\
  Assume $\mu_1, \mu_2$ both represent $\phi$. Let $E \subseteq U \subseteq K$ where $E$ closed, $U$ open. By Urysohn, find $f$ such that $E \prec f \prec U$. Now,
  $$\mu_1(E) \le \int_K f d\mu_1 = \phi(f) = \int_K f d\mu_2 \le \mu_2(U)$$
  Taking the inf over $U$, we get $\mu_1(E) \le \mu_2(E)$. By symmetry, $\mu_1(E) = \mu_2(E)$. By regularity, $\mu_1 = \mu_2$.

  {\bf Existence} \\
  For $U$ open, define $\mu^*(U) = \sup_{f \prec U} \phi(f)$. Note that
  $$\mu^*(U) \ge 0, \mu \text{ monotone}, \mu^*(K) = \phi(1)$$
  It follows that, for $V$ open, $\mu^*(V) = \inf_{U \supseteq V} \mu^*(U)$. Hence extend the definition of $\mu^*$ to
  $$\mu^*(A) = \inf_{U \supseteq A} \mu^*(U)$$
  We will show that $\mu^*$ is an outer measure.
  \begin{itemize}
    \item $\mu(\emptyset) = 0$
    \item If $A \subseteq B$, then $\mu^*(A) \le \mu^*(B)$.
    \item Do we have $\mu^*\left(\Union_n A_n\right) = \sum_n \mu^*(A_n)$? \\
    First assume that the $A_n = U_n$ are open. Let $U = \Union_n U_n$. Assume $f \prec U$ and let $E = \supp f$. $E \subseteq \Union_n U_n$, so by compactness find $N$ such that $E \subseteq \Union_{n = 1}^N U_n$. By Lemma \ref{lem:partition-of-unity}, find $h_n \prec U_n$ with $\sum_{n = 1}^N h_n \le 1$ and $\sum_{n = 1}^N h_n = 1$ on $E$. So $f = \sum_{n = 1}^N fh_n$ and
    \begin{align*}
      \phi(f)
      & = \sum_{n = 1}^N \phi(fh_n) \\
      & \le \sum_{n = 1}^N \mu^*(U_j) \text{ as } fh_n \prec U_n \\
      & \le \sum_n \mu^*(U_n)
    \end{align*}
    Taking the sup over $f$, we get $\mu^*(U) \le \sum_n \mu^*(U_n)$. It follows that
    $$\mu^*(\Union_n A_n) \le \sum_n \mu^*(A_n)$$
    We now let $\mcM$ be the set of $\mu^*$-measurable sets. Then $\mcM$ is a $\sigma$-algebra and $\mu^*\restriction_{\mcM}$ is a measure on $\mcM$. \\
    To restrict it further to a Borel, we now show that $\mcB \subseteq \mcM$. It's enough to show that $\mcG \subseteq \mcM$. \\
    Let $U$ open. We need
    $$\mu^*(A) \ge \mu^*(A \inter U) + \mu^*(A \setminus U) \text{ for all } A$$
    First, let $A = V \in \mcG$. Fix $f \prec V \inter U$ and $g \prec V \setminus \supp f$. Then $f + g \prec V$, thus
    $$\mu^*(V) \ge \phi(f + g) = \phi(f) + \phi(g)$$
    Taking the sup over $g$,
    $$\mu^*(V) \ge \phi(f + g) = \phi(f) + \mu^*(V \setminus \supp f) \ge \phi(f) + \mu^*(V \setminus U)$$
    Taking the sup over $f$,
    $$\mu^*(V) \ge \mu^*(V \inter U) + \mu^*(V \setminus U)$$
    Now let $A$ be arbitrary. Fix $V$ open such that $A \subseteq V$. then
    $$\mu^*(V) \ge \mu^*(V \inter U) + \mu^*(V \setminus U) \ge \mu^*(A \inter U) + \mu^*(A \setminus U)$$
    Taking the inf over $V$,
    $$\mu^*(A) \ge \mu^*(A \inter U) + \mu^*(A \setminus U)$$
    Now, $\mu := \mu^*\restriction_{\mcB}$ is a Borel measure on $K$. We have
    $$\mu(K) = \phi(1) = \norm\phi < \infty$$
    and by definition $\mu$ is regular. It remains to show that $\phi(f) = \int_K f d\mu$ for all $f \in C(K)$. It is enough to check that for $f \in C^\R(K)$ and enough to check that $\phi(f) \le \int_K f d\mu$ (apply this to $-f$). \\
    Fix $0 < a < b$ in $\R$ such that $\phi(1) \in [a, b]$. Let $\eps > 0$. Choose $0 \le y_0 < a \le y_1 < \dots < y_n = b$ such that
    $$y_j < y_{j - 1} + \eps$$
    Let $A_j = f^{-1}\mathopen]y_{j - 1}, y_j\mathclose]$. Those sets form a measurable partition of $K$. Choose closed sets $E_j$ and open sets $U_j$ such that $E_j \subseteq A_j \subseteq U_j$ and $\mu(U_j \setminus E_j) < \frac\eps n$ (by regularity) and $f(U_j) \subseteq \mathopen]y_{j - 1}, y_j\mathclose]$. By Lemma \ref{lem:partition-of-unity}, find $h_j \prec U_j$ for each $j$ such that $\sum_j h_j = 1$. Now,
    \begin{align*}
      \phi(f)
      & = \sum_j \phi(f_j) \\
      & \le \sum_j (y_j + \eps)\phi(h_j) \\
      & \le \sum_j (y_{j - 1} + 2\eps)\left(\mu(E_j) + \frac\eps n\right) \\
      & = \sum_j y_{j - 1}\mu(E_j) + \underbrace{\sum_j (b + \eps) + 2\eps\mu(K) + 2\eps^2}_{o(1)} \\
      & = \int_K \sum_j y_{j - 1}1_{E_j} d\mu + o(1)
      & \le \int_K f d\mu + o(1)
    \end{align*}
    since $f \le y_j + \eps$ on $U_j$, $h_j \prec U_j$ and $\phi \in M^+(K)$. So $\phi(f) \le \int_K f d\mu$.
  \end{itemize}
\end{proof}

\newlec

\begin{cor}
  For every $\phi \in M(K)$, there exists a unique regular complex Borel measure $\nu$ on $K$ that represents $\phi$: $\phi(f) = \int_K f d\nu$ for all $f \in C(K)$. Moreover, $\norm\phi = \norm\nu_1$ and if $\phi \in M^\R(K)$ then $\nu$ is a signed measure.
\end{cor}
\begin{proof}~\\
  {\bf Existence} \\
  Apply Lemma \ref{lem:m-real-pos} and Theorem \ref{thm:riesz-rep} to obtain a regular complex Borel measure representing $\phi$. We now want $\norm\phi = \norm\nu_1$. \\
  We already know $\norm\phi \le \norm\nu_1$. Take a measurable partition $K = \Union_{j = 1}^n A_j$. Fix $\eps > 0$ and closed sets $E_j$, open sets $U_j$ such that $E_j \subseteq A_j \subseteq U_j$, $\abs\nu(U_j \setminus E_j) < \frac\eps n$ ($\nu$ is regular). We can also assume $U_i \subseteq \Inter_{j \ne i} E_j^c$. Fix $\lambda_j \in \C$ such that $\abs{\lambda_j} = 1, \lambda_j\nu(E_j) = \abs{\nu(E_j)}$. By Lemma \ref{lem:partition-of-unity}, find $h_j \prec U_j$ such that $\sum_{j = 1}^n h_j = 1$. Then $E_j \prec h_j$, hence
  \begin{align*}
    \abs{\int_K \left(\sum_{j = 1}^n \lambda_j 1_{E_j} - \sum_{j = 1}^n \lambda_jh_j\right) d\nu}
    & \le \sum_{j = 1}^n \int_K \abs{1_{E_j} - h_j} d\abs\nu \\
    & \le \sum_{j = 1}^n \abs\nu(U_j \setminus E_j) < \eps
  \end{align*}
  Now,
  \begin{align*}
    \sum_{j = 1}^n \abs{\nu(A_j)}
    & \le \sum_{j = 1}^n \abs{\nu(E_j)} + \eps \\
    & = \sum_{j = 1}^n \lambda_j\nu(E_j) + \eps \\
    & = \int_K \sum_{j = 1}^n \lambda_j1_{E_j} d\nu + \eps \\
    & \le \abs{\int_K \sum_{j = 1}^n \lambda_jh_j d\nu} + 2\eps \\
    & \le \abs{\phi\left(\sum_{j = 1}^n \lambda_jh_j\right)} + 2\eps \\
    & \le \norm\phi \norm{\sum_{j = 1}^n \lambda_jh_j}_\infty + 2\eps \\
    & \le \norm\phi + 2\eps
  \end{align*}
  It follows that $\norm\nu_1 \le \norm\phi$.
\end{proof}

\begin{cor}
  The space of regular real (resp. complex) Borel measures on $K$ is a real (resp. complex) Banach space in $\norm\cdot_1$ isomorphic to $M^\R(K)$ (resp. $M(K)$).
\end{cor}

\clearpage

\section{Weak topologies}
Let $X$ be a set and $\mcF$ a set of functions on $X$ such that each $f \in \mcF$ is a function $X \to Y_f$ where $Y_f$ is a topological space. The {\bf weak topology $\sigma(X, \mcF)$ on $X$ generated by $\mcF$} is the smallest topology on $X$ that makes each $f \in \mcF$ continuous.

\begin{rmks}~
  \begin{enumerate}
    \item $S = \{f^{-1}(U) \mid f \in \mcF, U \subseteq Y_f \text{ open}\}$ is a subbase of $\sigma(X, \mcF)$. So
    \begin{align*}
      V \subseteq X \text{ open}
      \iff {} & \for x \in V, \exists f_1, \dots, f_n \in \mcF, U_i \subseteq Y_{f_i}, x \in \Inter_i f_i^{-1}(U_i) \subseteq V \\
      \iff {} & \for x \in V, \exists f_1, \dots, f_n \in \mcF, \\
      & \text{ open neighborhoods } U_i \text{ of } f_i(x), \Inter_i U_i \subseteq V
    \end{align*}
    \item More generally, if $S_f$ is a subbase in $Y_f$, then $\{f^{-1}(U) \mid f \in \mcF, U \in S_f\}$ is a subbase of $\sigma(X, \mcF)$.
    \item If $Y_f$ is Hausdorff for all $f \in \mcF$ and {\bf $\mcF$ separates points of $X$} ($\for x \ne y, \exists f \in \mcF, f(x) \ne f(y)$), then $\sigma(X, \mcF)$ is Hausdorff.
    \item Let $Y \subseteq X, \mcF_Y = {f\restriction_Y \mid f \in \mcF}$. Then $\sigma(Y, \mcF_Y) = \sigma(X, \mcF)\restriction_Y$.
    \item {\bf Universal property}: Let $Z$ be a topological space and $g : Z \to X$. then
    $$g \text{ continuous} \iff \for f \in \mcF, f \comp g : Z \to Y_f \text{ continuous}$$
  \end{enumerate}
\end{rmks}

\begin{egs*}~
  \begin{enumerate}
    \item Let $X$ be a topological space, $Y \subseteq X$ and $\iota : Y \to X$ the inclusion map. Then $\sigma(Y, {\iota})$ is the subspace topology on $Y$.
    \item Let $\Gamma$ be a set, $X_\gamma$ a topological space for each $\gamma \in \Gamma$, $X = \prod_{\gamma \in \Gamma} X_\gamma$. For each $\gamma$, we have $\pi_\gamma : X \to X_\gamma$ sending $x \mapsto x_\gamma$, the {\bf evaluation map at $\gamma$}, or {\bf projection onto $X_\gamma$}. The weak topology $\sigma(X, \{\pi_\gamma \mid \gamma \in \Gamma\})$ is called the {\bf product topology} on $X$.
    $$V \subseteq X \text{ open} \iff \substack{\for x \in V, \exists s \subseteq \Gamma \text{ finite}, U_\gamma \text{ neighborhood of } x_\gamma, \\ \{y \mid \for \gamma \in s, y_\gamma \in U_\gamma\} \subseteq V}$$
  \end{enumerate}
\end{egs*}

\begin{prop}\label{prop:lcs-metrisable}
  Let $X$ be a set. For each $n$, let $(Y_n, d_n)$ be a metric space and $f_n : X \to Y_n$ be a separating family of functions. Then $\sigma(X, \{f_n \mid n \in \N\})$ is metrisable.
\end{prop}
\begin{proof}
  Call $\sigma := \sigma(X, \{f_n \mid n \in \N\})$. Define
  $$d(x, y) = \sum_{n = 0}^\infty 2^{-n}\min(d(f_n(x), f_n(y)), 1)$$
  This is a metric on $X$. Given $0 < \eps < 1$, if $d(x, y) < 2^{-n}\eps$, then $d(f_n(x), f_n(y)) < \eps$. So each $f_n$ is continuous with respect to the topology $\tau$ induced by that metric. Hence $\sigma \subseteq \tau$. \\
  Reciprocally, $y \mapsto d(x, y)$ is $\sigma$-continuous for each $x$ by the Weierstrass M-test since
  $$y \mapsto 2^{-n}\min(d(f_n(x), f_n(y)), 1)$$
  is $\sigma$-continuous for each $n$.
\end{proof}

\begin{thm}[Tychonoff]\label{thm:tychonoff}
  The product of compact topological spaces is compact.
\end{thm}
\begin{proof}
  Assume each $X_\gamma$ is compact. Let $\mcE$ be a family of closed subsets with the FIP (finite intersection property). We want $\Inter_{E \in \mcE} E \ne \emptyset$. \\
  By Zorn, find a maximal family $\mcA$ of sets in $X$ such that $\mcE \subseteq \mcA$ and $\mcA$ has the FIP. We will show that $\Inter_{A \in \mcA} \bar A \ne \emptyset$. Maximality of $\mcA$ means that
  \begin{itemize}
    \item $\mcA$ is closed under finite intersections.
    \item If $B$ intersects every $A \in \mcA$, then $B \in \mcA$.
  \end{itemize}
  For each $\gamma \in \Gamma$, $\{\pi_\gamma(A) \mid A \in \mcA\}$ has the FIP, hence find by compactness of $X_\gamma$ some $x_\gamma \in \Inter_{A \in \mcA} \overline{\pi_\gamma(A)}$. \\
  We show that all neighborhoods of $x$ are in $\mcA$. Then $\for A \in \mcA, x \in \bar A$. \\
  It's enough to show it for neighborhoods of the form $U = \Inter_{\gamma \in s} \pi_\gamma^{-1}(U_\gamma)$ for some $s \subseteq \Gamma$ finite where each $U_\gamma$ is a neighborhood of $x_\gamma$. For such $U$, we see that $\pi_\gamma^{-1}(U_\gamma)$ intersects every $A \in \mcA$, so $\pi_\gamma^{-1}(U_\gamma)$ by the second remark. Hence $U \in \mcA$ by the first remark.
\end{proof}

\subsection{Weak topologies on vector spaces}

\newlec

Let $E$ be a real or complex vector space. Let $F$ be a subspace of the space of all linear functionals on $E$ that separates points of $E$, ie $\for x \in E, x \ne 0 \imp \exists f \in F, f(x) \ne 0$. \\
Consider the weak topology $\sigma(E, F)$
$$U \text{ open} \iff \for x \in U, \exists f_1, \dots, f_n \in F, \eps > 0, \{y \mid \for i, \abs{f_i(x - y)} < \eps\} \subseteq U$$

For $f \in F$, $x \in E$, let $p_f(x) = \abs{f(x)}$. Let $\mcP = \{p_f \mid f \in \mcF\}$. Then $(E, \mcP)$ is a LCS whose topology is $\sigma(E, F)$. So $\sigma(E, F)$ is Hausdorff and vector addition and scalar multiplication are continuous.

\begin{lem}\label{lem:ker-inter}
  Let $E$ be as above, $f, g_1, \dots, g_n$ linear functionals on $E$ such that
  $$\Inter_i \ker g_i \subseteq \ker f$$
  Then $f \in \Span \{g_1, \dots, g_n\}$.
\end{lem}
\begin{proof}
  Reinterpret the $g_i$ as a single linear map $g : E \to \K^n$. Then
  $$\ker g = \Inter_i \ker g_i \subseteq f$$
  Hence we have a factorisation $f = h \comp g$. Find $a_1, \dots, a_n$ such that $h(y) = \sum_i a_i y_i$ for all $y \in \K^n$. Then
  $$f(x) = h(g(x)) = \sum_i a_ig_i(x)$$
  for all $x$, so $f = \sum_i a_i g_i \in \Span \{g_1, \dots, g_n\}$.
\end{proof}

\begin{prop}
  Let $E, F$ be as above and $f$ a linear functional on $E$. Then
  $$f \text{ is } \sigma(E, F)\text{-continuous} \iff f \in F$$
  Namely,
  $$(E, \sigma(E, F))^* = F$$
\end{prop}
\begin{proof}~\\
  $\impliedby$ \\
  True by definition.
  
  $\imp$ \\
  Find an open neighborhood $U$ of $0$ in $E$ such that $\for x \in U, \abs{f(x)} < 1$. WLOG $U = \{x \mid \for i, \abs{g_i(x)} < \eps\}$ for some $\eps > 0$, $g_1, \dots, g_n \in F$. \\
  If $x \in \Inter_i \ker g_i$, then $\lambda x \in U$ for all $\lambda$, hence
  $$\abs\lambda\abs{f(x)} = \abs{f(\lambda x)} < 1$$
  for all $\lambda$, so $f(x) = 0$. By Lemma \ref{lem:ker-inter}, $f \in \Span \{g_1, \dots, g_n\} \subseteq F$.
\end{proof}

\begin{egs*}~
  \begin{enumerate}
    \item Let $X$  be a normed space. The {\bf weak topology} on $X$ is the topology $\sigma(X, X^*)$ on $X$ ($X^*$ separates points of $X$ by Hahn-Banach). We sometimes write $(X, w)$ for $(X, \sigma(X, X^*))$. Open sets in $\sigma(X, X^*)$ are called {\bf weak open} or {\bf w-open}.
    \begin{equation*}
      \begin{gathered}
        U \subseteq X \text{ is w-open} \\
        \iff \\
        \for x \in U, \exists \eps > 0, f_1, \dots, f_n \in X^*, \{y \in X \mid \for i, \abs{f_i(y - x)} < \eps\} \subseteq U
      \end{gathered}
    \end{equation*}
    \item Let $X$ be a normed space. The {\bf weak star topology} or {\bf w*-topology} on $X^*$ is the topology $\sigma(X^*, X)$. Here we identify $X$ with its image $\hat X$ in $X^{**}$ under the canonical embedding. Open sets in $\sigma(X^*, X)$ are called {\bf w*-open}.
    \begin{equation*}
      \begin{gathered}
        U \subseteq X^* \text{ is w*-open} \\
        \iff \\
        \for f \in U, \exists \eps > 0, x_1, \dots, x_n \in X, \{g \in X^* \mid \for i, \abs{g(x_i) - f(x)} < \eps\} \subseteq U
    \end{gathered}
  \end{equation*}
  \end{enumerate}
\end{egs*}

\begin{properties}~
  \begin{enumerate}
    \item $(X, w)$ and $(X^*, w^*)$ are LCS, hence Hausdorff with continuous vector space operations.
    \item $\sigma(X, X^*)$ is a subtopology of the norm topology, with equality iff $X$ is finite dimensional.
    \item $\sigma(X^*, X)$ is a subtopology of $\sigma(X^*, X^{**})$, with equality iff $X$ is reflexive.
    \item Let $Y$ be a subspace of $X$. Then
    $$\sigma(X, X^*)\restriction_Y = \sigma(Y, \{f\restriction_Y \mid f \in X^*\}) \overset{\text{Hahn-Banach}} = \sigma(Y, Y^*)$$
    Similarly,
    $$\sigma(X^{**}, X^*)\restriction_X = \sigma(X, X^*) = \sigma(X, \{\hat f\restriction_X \mid f \in X^*\})$$
    So the canonical embedding is a homeomorphism $\sigma(X, X^*) \to \sigma(\hat X, X^*)$.
  \end{enumerate}
\end{properties}

\begin{prop}
  Let $X$ be a normed space.
  \begin{enumerate}
    \item A linear functional $f$ on $X$ is w-continuous iff $f \in X^*$. So $(X, w)^* = X^*$.
    \item A linear functional $\Lambda$ on $X^*$ is w*-continuous iff $\Lambda \in \hat X$. So $(X^*, w^*)^* = X$.
  \end{enumerate}
\end{prop}

It follows that $\sigma(X^*, X) = \sigma(X^*, X^{**})$ iff $X$ is reflexive.

\begin{dfn*}
  Let $X$ be a normed space.
  \begin{enumerate}
    \item A set $A$ in $X$ is {\bf weakly bounded} if $\{f(x) \mid x \in A\}$ is bounded for all $f \in X^*$, or equivalently if for all w-neighborhood $U$ there exists $\lambda$ such that $A \subseteq \lambda U$.
    \item A set $B$ in $X^*$ is {\bf w*-bounded} if $\{f(x) \mid f \in B\}$ is bounded for all $x \in X$, or equivalently if for all w*-neighborhood $U$ there exists $\lambda$ such that $B \subseteq \lambda U$.
  \end{enumerate}
\end{dfn*}

\begin{thm*}[Principle of Uniform Boundedness, PUB]
  Let $X$ be a Banach space, $Y$ a normed space $\mcT \subseteq \mcB(X, Y)$. If $\mcT$ is {\bf pointwise bounded} ($\for x \in X, \sup_{T \in \mcT} \norm{Tx} < \infty$), then $\mcT$ is {\bf uniformly bounded} ($\sup_{T \in \mcT} \norm T < \infty$).
\end{thm*}

\begin{prop}\label{prop:weak-norm-bounded}
  Let $X$ be a normed space.
  \begin{enumerate}
    \item If $A \subseteq X$ is weakly bounded, then $A$ is norm-bounded.
    \item If $X$ is complete and $B \subseteq X^*$ is w*-bounded, then $B$ is norm-bounded.
  \end{enumerate}
\end{prop}
\begin{proof}~
  \begin{enumerate}
    \item $A$ being weak bounded means that $\hat A = \{\hat x \mid x \in A\}$ is pointwise bounded. So we're done by PUB.
    \item $B$ being w*-bounded means that $B$ is pointwise bounded. So we're done by PUB.
  \end{enumerate}
\end{proof}

\begin{notation}
  We write $x_n \wto x$ if $x_n$ converges to $x$ in the weak topology. Note that
  $$x_n \wto x \iff \for f \in X^*, \inn{x_n}f \to \inn xf$$
  We write $f_n \wstarto f$ if $f_n$ converges to $f$ in the w*-topology. Note that
  $$f_n \wto f \iff \for x \in X, \inn x{f_n} \to \inn xf$$
\end{notation}

\begin{thm}[Consequence of PUB]
  Let $X$ be a Banach space, $Y$ a normed space, $T_n$ a sequence in $\mcB(X, Y)$. If $T_n$ converges pointwise to some function $T : X \to Y$, then $T \in \mcB(X, Y)$, $\sup_n \norm{T_n} < \infty$ and $\norm T \le \liminf_n \norm{T_n}$.
\end{thm}

\begin{prop}
  Let $X$ be a normed space.
  \begin{enumerate}
    \item If $x_n \wto x$ in $X$, then $\sup_n \norm{x_n} < \infty$ and $\norm x \le \liminf \norm{x_n}$.
    \item If $f_n \wstarto f$ in $X^*$, then $\sup_n \norm{f_n} < \infty$ and $\norm f \le \liminf \norm{f_n}$.
  \end{enumerate}
\end{prop}
\begin{proof}~
  \begin{enumerate}
    \item $\widehat{x_n} \to \hat x$ pointwise in $X^{**}$. Result follows by PUB.
    \item $f_n \to f$ pointwise in $X^*$. Result follows by PUB.
  \end{enumerate}
\end{proof}

\newlec

The weak topology is weaker than the norm topology as we see by the fact that $e_n \wto 0$ in $\ell_p$ ($1 \le p < \infty$) but $e_n \not\to 0$, where $e_n$ is the vector with a single $1$ in the $n$-th position.

\subsection{Hahn-Banach Separation Theorems}

Let $(X, \mcP)$ be a locally convex space. Let $C$ be a convex set such that $0 \in \interior C$. Then define
\begin{align*}
  \mu_C : X & \to \R \\
  x & \mapsto \inf \{t > 0 \mid x \in tC\}
\end{align*}
This is well-defined since $\frac 1n x \to 0$ and so $\frac 1n x \in C$ for some $n$. $\mu_C$ is the {\bf Minkowski functional} (aka {\bf gauge functional}) of $C$.

\begin{eg*}
  If $X$ is a normed space and $C = B_X$, then $\mu_C = \norm\cdot$.
\end{eg*}

\begin{lem}\label{lem:ball-gauge}
  $\mu_c$ is positive homogeneous and subadditive. Moreover,
  $$\{x \mid \mu_C(x) < 1\} \subseteq C \subseteq \{x \mid \mu_C(x) \le 1\}$$
  with the first equality holding iff $C$ is open and the second equality holding iff $C$ is closed.
\end{lem}
\begin{proof}~\\
  {\bf positive homogeneity} \\
  For $x \in X, s, t > 0$, we have $sx \in stC \iff x \in tC$. Hence
  $\mu_C(sx) = s\mu_C(x)$. It also holds for $s = 0$ since $\mu_C(0) = 0$.

  {\bf subadditivity} \\
  First observe that $\mu_C(x) < t$ implies $x \in tC$. Indeed, there is some $s < t$ such that $x \in sC$. Then
  $$\frac xt = \left(1 - \frac st\right) \cdot 0 + \frac st \cdot  \frac xs \in C$$
  by convexity. Now let $x, y \in X$. Fix $s > \mu_C(x), t > \mu_C(y)$. Then $x \in sC, y \in tC$, so
  $$x + y \in sC + tC = (s + t)C$$
  by convexity. So $\mu_C(x + y) < s + t$. Taking the infima over $s$ and $t$, $\mu_C(x + y) \le \mu_C(x) + \mu_C(y)$.

  {\bf $\{x \mid \mu_C(x) < 1\} \subseteq C$ with equality iff $C$ open} \\
  If $\mu_C(x) < 1$, then $x \in C$ by the observation. If $C$ is open and $x \in C$, find $n$ such that $\left(1 + \frac 1n\right)x \in C$. Then
  $$\mu_C(x) \le \frac 1{1 + \frac 1n} < 1$$

  {\bf $C \subseteq \{x \mid \mu_C(x) \le 1\}$ with equality iff $C$ closed} \\
  If $x \in C$, then $\mu_C(x) \le 1$ by definition. If $C$ is closed and $\mu_C(x) \le 1$, then by homogeneity $\mu_C\left(\left(1 - \frac 1n\right)x\right) < 1$ for all $n$, so $\left(1 - \frac 1n\right)x \in C$, and $x \in C$ since $C$ is closed.  
\end{proof}

\begin{rmk}
  If $C$ is balanced, then $\mu_C$ is a seminorm. If further $C$ is bounded, then $\mu_C$ is a norm.
\end{rmk}

\begin{thm}[Hahn-Banach Separation]\label{thm:hb-separation-point}
  Let $(X, \mcP)$ be a LCS and $C$ be an open convex set with $0 \in C$. Let $x_0 \nin C$. Then there exists $f \in X^*$ such that $f(x_0) > f(x)$ for all $x \in C$.
\end{thm}
TODO: Insert separation picture
\begin{rmk}
  From now on, we work with real scalars. The complex case follows from the fact that $\Re : X^* \to X_\R^*$ is a real-linear bijection.
\end{rmk}
\begin{proof}
  Consider $\mu_C$. By Lemma \ref{lem:ball-gauge}, $C = \{x \mid \mu_C(x) < 1\}$. So $\mu_C(x_0) \ge 1$. Let $Y = \Span(x_0)$ and $g : Y \to \R$ defined by $g(\lambda x_0) = \lambda$. $g$ is linear and $g(x_0) = 1 \le \mu_C(x_0)$. Hence $g \le \mu_C$ on $Y$. \\
  By Theorem \ref{thm:hb-positive}, find $ f : X \to \R$ linear such that $f\restriction_Y = g$ and $f \le \mu_C$. For all $x \in C$, $f(x) \le \mu_C(x) < 1 = f(x_0)$. further, $f$ is continuous since $C \inter (-C)$ is a neighborhood of $0$ on which $\abs{f(x)} \le 1$.
\end{proof}

\begin{thm}\label{thm:hb-separation-set}
  Let $(X, \mcP)$ be a LCS. Let $A, B$ be disjoint nonempty convex sets.
  \begin{itemize}
    \item If $A$ is open, then there exists $f \in X^*$ such that $f(x) < \inf_B f$ for all $x \in A$.
    \item If $A$ is compact and $B$ is closed, then there exists $f \in X^*$ such that $\sup_A f < \inf_B f$.
  \end{itemize}
\end{thm}
\begin{proof}~
  \begin{itemize}
    \item Fix $a \in A, b \in B$. Let $C = (A - a) - (B - b)$ and $x_0 = b - a$. Then $C$ is open, convex, $0 \in C$ and $x_0 \nin C$ ($A, B$ are disjoint). By Theorem \ref{thm:hb-separation-point}, find $f \in X^*$ such that $f(z) < f(x_0)$ for all $z \in C$. So for all $x \in A, y \in B$, $f(x - y + x_0) < f(x_0)$, namely $f(x) < f(y)$. In particular, $f \ne 0$. So find $u$ such that $f(u) > 0$. Given $x \in A$, as $A$ is open and $x + \frac 1n u \to x$, find $n$ such that $x + \frac 1n u \in A$. Then
    $$f(x) < f\left(x + \frac 1n u\right) \le \inf_B f$$
  \item
  \begin{claim}
    There exists a convex open neighborhood $U$ of $0$ such that $A + U$ is disjoint from $B$.
  \end{claim}
  \begin{proof}
    For $x \in A$, find $U_x$ an open neighborhood of $0$ such that $x + U_x$ is disjoint from $B$ (since $B$ is closed). By continuity of addition, find $V_x$ an open neighborhood of $0$ such that $V_x + V_x \subseteq U_x$. WLOG $V_x$ is convex and symmetric. By compactness, find $x_1, \dots, x_n \in A$ such that $A \subseteq \Union_{i = 1}^n x_i + V_{x_i}$. We claim $U = \Inter_{i = 1}^n V_{x_i}$ works. Given $x \in A$, find $i$ such that $x \in x_i + V_{x_i}$, so that
    $$x + U \subseteq x_i + V_{x_i} + U \subseteq x_i + V_{x_i} + V_{x_i} \subseteq x_i + U_{x_i}$$
    is disjoint from $B$. Hence $A + U$ is disjoint from $B$.
  \end{proof}
  Apply the first part with $A + U$ and $B$ to get $f \in X^*$ such that $f(x + u) < f(y)$ for all $x \in A, y \in B, u \in U$. In particular, $f \ne 0$, so find $z$ such that $f(z) > 0$. As $\frac 1n z \to 0$, find $n$ such that $\frac 1n z \in U$. Then $f(x) + \frac 1n f(z) < f(y)$ for all $x \in A, y \in B$. So
  $$\sup_A f < \sup_A f + \frac 1n f(z) \le \inf_B f$$
  \end{itemize}
\end{proof}

\begin{thm}[Mazur]
  Let $C$ be a convex set in a normed space. Then $\bar C^{\norm\cdot} = \bar C^{\text w}$. In particular,
  $$C \text{ norm-closed} \iff C \text{ w-closed}$$
\end{thm}
\begin{proof}
  WLOG $C$ is nonempty. We already know $\bar C^{\norm\cdot} \subseteq \bar C^{\text w}$ as the weak topology is weaker than the norm-topology. \\
  If $x \nin \bar C^{\norm\cdot}$, then apply Theorem \ref{thm:hb-separation-set} to $A = \{x\}$ and $B = \bar C^{\norm\cdot}$ to obtain $f \in X^*$ such that $f(x) < \inf_B f$. Then $\{z \mid f(z) < \inf_B f\}$ is a w-open neighborhood of $x$ disjoint from $B$. So $x \nin \bar C^{\text w}$.
\end{proof}

\begin{cor}
  If $x_n \wto 0$ in a normed space, then for $\eps > 0$ there is some $x$ in the convex hull of the $x_0$ such that $\norm x < \eps$.
\end{cor}
\begin{proof}
  $$0 \in \overline{\conv\{x_n \mid n \in \N\}}^{\text w} = \overline{\conv\{x_n \mid n \in \N\}}^{\norm\cdot}$$
\end{proof}

\begin{rmk}
  It follows from this that there exist $p_1 < q_1 < p_2 < q_2 < \dots $ and convex combinations $z_n = \sum_{i = p_n}^{q_n} t_i x_i$ such that $z_n \to 0$.
\end{rmk}

\newlec

\begin{thm}[Banach-Alaoglu]\label{thm:banach-alaoglu}
  For any normed space $X$, $(B_{X^*}, w^*)$ is compact.
\end{thm}
\begin{proof}
  For $x \in X$, let $K_x = \{\lambda \in \K \mid \abs\lambda \le \norm x\}$. Equip $K$ with its product topology. Let $\pi_x : K \to K_x$ be the projection. Note
  $$K = \{\lambda : X \to \K \mid \for x \in X, \abs{\lambda(x)} \le \norm x\}$$
  So $B_{X^*} \subseteq K$. By Tychonoff (Theorem \ref{thm:tychonoff}), $K$ is compact. So all we need to show is that $B_{X^*}$ is closed in $K$.
  \begin{align*}
    B_{X^*}
    & = \{\lambda \in K \mid \for a, b, x, y, \lambda(ax + by) = a\lambda(x) + b\lambda(y)\} \\
    & = \Inter_{a, b, x, y} \{\lambda \in \K \mid \pi_{ax + by}(\lambda) = a\pi_x(\lambda) + b\pi_y(\lambda)\} \\
    & = \Inter_{a, b, x, y} (\pi_{ax + by} - a\pi_x - b\pi_y)^{-1}\{0\}
  \end{align*}
  is closed in $K$ since each $\pi_x$ is continuous.
\end{proof}

\begin{prop}\label{prop:ck-separable}
  Let $X$ be a normed space and $K$ be a compact Hausdorff space.
  \begin{enumerate}
    \item $X \text{ separable} \iff (B_{X^*}, w^*) \text{ metrisable}$
    \item $C(K) \text{ separable} \iff K \text{ metrisable}$
  \end{enumerate}
\end{prop}
\begin{proof}~
  \begin{itemize}
    \item[1. $\Rightarrow$] Fix a dense sequence $x_n$ in $X$. Let $\mcF = \{\hat x_n \mid n \in \N\}$. Then $\mcF$ separates points of $X^*$, so $\sigma(B_{X^*}, \mcF)$ is Hausdorff and contained in the w*-topology of $B_{X^*}$. So
    $$\id : (B_{X^*}, w^*) \to (B_{X^*}, \sigma(B_{X^*}, \mcF))$$
    is a continuous bijection from compact to Hausdorff, hence a homeomorphism. So $\sigma(B_{X^*}, \mcF)$ is the w*-topology on $B_{X^*}$. This is metrisable by Proposition \ref{prop:lcs-metrisable}.
  \item[2. $\Rightarrow$] By the above, $(B_{C(K)^*}, w^*)$ is metrisable. For $k \in K$, define
  \begin{align*}
    \delta_k : C(K) & \to \K \\
    F \mapsto f(k)
  \end{align*}
  Then $\delta_k \in B_{C(K)^*}$. We thus have $\delta : K \to (B_{C(K)^*}, w^*)$.

  {\bf $\delta$ continuous} \\
  By the universal property, it's enough to check that $\hat f \comp \delta$ is continuous for all $f \in C(K)$. But
  $$(\hat f \comp \delta)(k) = \delta_k(f) = f(k)$$
  So $\hat f \comp \delta = f$ is continuous.

  {\bf $\delta$ injective} \\
  $C(K)$ separates points of $K$.

  Now, $\delta : K \to (\delta(K), w^*)$ is a continuous bijection from compact to Hausdorff, hence a homeomorphism. Hence $K$ is metrisable.
  \item[2. $\Leftarrow$] As $K$ is compact metrisable, it is separable. Fix a sequence $x_n$ dense in $K$. Let $f_n(x) = d(x, x_n)$. $d$ is a metric inducing the topology of $K$. Let $A$ be the subalgebra of $C(K)$ generated by $1$ and the $f_n$. Then $A$ is separable, closed under complex conjugation, separates points of $K$ and $1 \in A$. By Stone-Weierstrass, $\bar A = C(K)$. So $C(K)$ is separable.
  \item[1. $\Leftarrow$] Let $K = (B_{X^*}, w^*)$. This is compact by Theorem \ref{thm:banach-alaoglu}. Since $K$ is metrisable, $C(K)$ is separable. It's enough to show that $X$ embeds isometrically into $C(K)$. Let
  \begin{align*}
    T : X & \to C(K) \\
    x & \mapsto \hat x\restriction_{B_{X^*}}
  \end{align*}
  Then $T$ is linear and $\norm{Tx}_\infty = \norm{\hat x} = \norm x$.
  \end{itemize}
\end{proof}

\begin{rmks}~
  \begin{enumerate}
    \item If $X$ is separable, then $(B_{X^*}, w^*)$ is compact metrisable, hence w*-sequentially compact.
    \item $X \text{ separable} \imp X^* \text{ w*-separable}$ ($(B_{X^*}, w^*)$ compact metrisable, hence separable). Recall that, for any topological vector space $Y$,
    $$Y \text{ separable} \iff \exists A\text{ countable}, \overline\Span A = Y$$
    Hence Mazur tells us
    $$X \text{ separable} \iff X \text{ w-separable}$$
    So $X \text{ w-separable} \imp X^* \text{ w*-separable}$. The converse is false, eg $\ell_\infty$.
    \item The proof shows that $(B_{C(K)^*}, w^*)$ contains a homeomorphic copy of $K$.
    \item The proof shows that every normed space $X$ embeds isometrically into $C(K)$ for some compact Hausdorff space $K$, eg $K = (B_{X^*}, w^*)$.
  \end{enumerate}
\end{rmks}

\begin{prop}
  Let $X$ be a normed space. Then
  $$X^* \text{ separable} \iff (B_X, w) \text{ metrisable}$$
\end{prop}
\begin{proof}~
  \begin{itemize}
    \item[$\Rightarrow$] By Proposition \ref{prop:ck-separable}, $(B_{X^{**}}, w^*)$ is metrisable. Hence
    $$(B_X, w) = (B_{X^{**}}, w^*)\restriction_{B_X}$$
    is metrisable.
    \item[$\Leftarrow$] Let $d$ metrise $(B_X, w)$. For all $n$, find $F_n \subseteq X^*$ finite and $\eps_n > 0$ such that
    $$U_n = \{x \in B_X \mid \for f \in F_n, \abs{f(x)} < \eps_n\} \subseteq \curlybrack{x \mid d(x, 0) < \frac 1n}$$
    We claim $Z = \Span_n F_n$ is dense. Then we're done. \\
    Let $g \in X^*, \eps > 0$. Then $\{x \in B_X \mid \abs{g(x)} < \eps\}$ is a w-neighborhood of $0$ in $B_X$, hence contains $U_n$ for some $n$. Let $Y = \Inter_{f \in F_n} \ker f$. For $x \in B_Y$, $x \in U_n$, so $\abs{g(x)} < \eps$. So $\norm{g\restriction_Y} \le \eps$. By Hahn-Banach, find $h \in X^*$ such that $h\restriction_Y = g\restriction_Y$ and $\norm h \le \eps$. Now
    $$Y = \Inter_{f \in F_n} \ker f \subseteq \ker(g - h)$$
    hence $g - h \subseteq \Span F_n \subseteq Z$ by Lemma \ref{lem:ker-inter}. So $d(g, Z) < \eps$. Hence $g \in \bar Z$.
  \end{itemize}
\end{proof}

\begin{thm}[Goldstine]\label{thm:goldstine}
  For any normed space $X$,
  $$\overline{B_X}^{w^*} = B_{X^{**}}$$
  where $B_X$ is thought of as a subspace of $X^{**}$.
\end{thm}
\begin{proof}
  $B_{X^{**}}$ is w*-closed (by Theorem \ref{thm:banach-alaoglu}) and $B_X \subseteq B_{X^{**}}$, so $\overline{B_X}^{w^*} \subseteq B_{X^{**}}$. Now, let $\phi \nin \overline{B_X}^{w^*}$. Apply Theorem \ref{thm:hb-separation-set}.ii to $(X^{**}, w^*), A = \{\phi\}, B = \overline{B_X}^{w^*}$ and find $f \in X^*$? such that $\phi(f) > \sup_B \hat f$ (or $\Re \phi(f) > \sup_B \Re \hat f$ in the complex case).
  $$\norm\phi\norm f \ge \abs{\phi(f)} > \sup_B \norm{\hat f} = \sup_B \norm f \ge 1$$
  So $\phi \nin B_{X^{**}}$.
\end{proof}

\newlec

\begin{thm}
  Let $X$ be a Banach space. TFAE
  \begin{enumerate}
    \item $X$ is reflexive.
    \item $(B_X, w)$ is compact.
    \item $X^*$ is reflexive.
  \end{enumerate}
\end{thm}
\begin{proof}~
  \begin{itemize}
    \item[$1 \Rightarrow 2$] $(B_X, w) \cong (B_{X^{**}}, w^*)$ is compact by Banach-Alaoglu (Theorem \ref{thm:banach-alaoglu}).
    \item[$2 \Rightarrow 1$] $(B_X, w) = (B_{X^{**}}, w^*)\restriction_{B_X}$, so $B_X$ is compact in the w*-topology of $X^{**}$. Hence it is w*-closed in $X^{**}$. By Goldstine (Theorem \ref{thm:goldstine}), $B_{X^{**}} = \overline{B_X}^{w^*} = B_X$.
    \item[$1 \Rightarrow 3$] $(B_{X^*}, w) = (B_{X^*}, w^*)$ is compact by Theorem \ref{thm:banach-alaoglu}. By $2 \imp 1$, $X^*$ is reflexive.
    \item[$3 \Rightarrow 1$] By $1 \Rightarrow 3$, $X^{**}$ is reflexive. So by $1 \Rightarrow 2$, $(B_{X^{**}}, w)$ is compact. Since $X$ is complete, $X$ is closed in $X^{**}$, hence w-closed in $X^{**}$ (by Mazur). Hence $B_X = X \inter B_{X^{**}}$ is a w-closed subset of $B_{X^{**}}$ and thus is w-compact. By $2 \Rightarrow 1$, $X$ is reflexive.
  \end{itemize}
\end{proof}

\begin{rmk}
  If $X$ is separable and reflexive, then $(B_X, w)$ is compact metrisable. Hence $B_X$ is sequentially compact.
\end{rmk}

\begin{lem}\label{lem:pi-surjection}
  Let $(K, d)$ be a nonempty compact metric space. There exists a continuous surjection $\phi : \{0, 1\}^\N \to K$ where $\{0, 1\}^\N$ is given the product topology.
\end{lem}
\begin{proof}
  Since $K$ is totally bounded, if $A$ is nonempty closed and $\eps > 0$ there exist nonempty closed sets $B_1, \dots, B_n$ such that $A = \Union_i B_i$ and $\diam B_i < \eps$. Applying this repeatedly, find nonempty closed sets $K_\eps$ for all $\eps \in \Sigma := \Union_{n = 0}^\infty \{0, 1\}^n$ such that
  $$K_\emptyset = K, K_\eps = K_{\eps, 0} \union K_{\eps, 1}, \max_{\eps \in \{0, 1\}^n} \diam K_\eps \to 0$$
  Define
  \begin{align*}
    \phi : \{0, 1\}^\N & \to K \\
    \eps & \mapsto \text{unique point in } \Inter_n K_{\eps_1, \dots, \eps_n}
  \end{align*}

  {\bf $\phi$ onto} \\
  Given $x \in K$, find $\eps$ such that $x \in K_{\eps_1, \dots, \eps_n}$ for all $n$. Then $\phi(\eps) = x$.

  {\bf $\phi$ continuous} \\
  Let $\eps, \delta \in \{0, 1\}^\N, n \in \N$. If $\eps_i = \delta_i$ for all $i \le n$, then
  $$d(\phi(\eps), \phi(\delta)) \le \diam K_{\eps_1, \dots, \eps_n} \to 0$$
\end{proof}

\begin{rmk}
  $\{0, 1\}^\N$ is homeomorphic to the middle third Cantor set $\Delta$ via
  $$\eps \mapsto \sum_{i = 1}^\infty 2\eps_i 3^{-i}$$
\end{rmk}

\begin{thm}
  Every separable Banach space $X$ embeds isometrically into $C[0, 1]$, namely $C[0, 1]$ is isometrically universal for $\mcS\mcB$.
\end{thm}
\begin{proof}
  From the proof of Proposition \ref{prop:ck-separable}, $X$ embeds isometrically into $C(K)$ where $K = (B_{X^*}, w^*)$. Since $X$ is separable, $K$ is metrisable. By Lemma \ref{lem:pi-surjection}, find $\phi : \Delta \to K$ a continuous surjection. Hence $C(K)$ embeds isometrically into $C(\Delta)$ via $f \mapsto f \comp \phi$. $C(\Delta)$ embeds isometrically into $C[0, 1]$ via $f \mapsto \tilde f$ where $\tilde f$ linearly interpolates $f$ between elements of the Cantor set.
\end{proof}

\clearpage

\section{Convexity}

Let $X$ be a real or complex vector space and $K$ a convex set. A point $x \in K$ is an {\bf extreme point} of $K$ if, whenever $x = ay + bz, a, b > 0, a + b = 1$, we have $x = y = z$. Denote $\Ext K$ the set of extreme points of $K$.

\begin{egs*}~
  TODO: Pictures
  \begin{itemize}
    \item $\Ext(B_{\ell_1^2}) = \{\pm e_1, \pm e_2\}$
    \item $\Ext(B_{\ell_2^2}) = S_{\ell_2^2}$
    \item $\Ext(B_{c_0}) = \emptyset$. Indeed, if $x \in B_{c_0}$, we can find $n$ such that $\abs{x_n} < \frac 12$ and define $y = x + \frac 12 e_n, z = x - \frac 12 e_n$ so that $y, z \in B_{c_0}$ and $x + \frac 12 y + \frac 12 z, y \ne x, z \ne x$.
  \end{itemize}
\end{egs*}

\begin{thm}[Krein-Milman]\label{thm:krein-milman}
  Let $K$ be a nonempty compact convex set in a LCS $(X, \mcP)$. Then
  $$K = \overline\conv(\Ext K)$$
  In particular, $\Ext K$ is nonempty if $K$ is nonempty.
\end{thm}

\begin{cor}
  If $X$ is a normed space, then $B_{X^*} = \overline\conv(\Ext B_{X^*})$ and $\Ext B_{X^*}$ is nonempty.
\end{cor}

\begin{rmk}
  $c_0$ is not a dual space since $\Ext B_{c_0}$ is empty.
\end{rmk}

\begin{dfn*}
  Let $K$ be a nonempty compact convex set in a LCS $(X, \mcP)$. A {\bf face} of $K$ is a nonempty compact convex set $E \subseteq K$ such that, for all $y, z \in K, a, b > 0, a + b = 1$, if $ay + bz \in E$ then $y, z \in E$.
\end{dfn*}

\begin{egs*}~
  \begin{itemize}
    \item $K$ is a face of $K$.
    \item For $x \in K$, $\{x\}$ is a face of $K$ iff $x \in \Ext K$.
    \item Let $f \in X^*, \alpha = \sup_K f$. Then $E = \{x \in K \mid f(x) = \alpha\}$ is a face of $K$.
    \item Let $E$ be a face of $K$. If $F$ is a face of $E$, then $F$ is a face of $K$. In particular, $\Ext E \subseteq \Ext K$.
  \end{itemize}
\end{egs*}

\begin{proof}[Proof of Theorem \ref{thm:krein-milman}]
  First we show that any nonempty compact convex set $K$ has an extreme point.

  By Zorn, find a minimal face $E$ of $K$. \\
  If $\abs E > 1$, then pick $x \ne y$ in $E$ such that $f(x) > f(y)$. Then $F = \{z \in K \mid f(z) = \sup_E f\}$ is a face of $E$ which does not contain $y$. Hence it is a strictly smaller face of $K$. Contradiction. \\
  So $F$ is a singleton and $\Ext E \ne \emptyset$.
  
  Now WLOG $K$ is nonempty and let $L = \overline\conv(\Ext K)$. Then $L$ is a nonempty face of $K$. Assume $x_0 \ne K \setminus L$. By Theorem \ref{thm:hb-separation-point}, find $f \in X^*$ such that $f(x_0) > \sup_L f$. Let $\alpha = \sup_K f$. Then $E = \{x \in K \mid f(x) = \alpha\}$ is a face of $K$. Find $z$ an extreme point of $E$. Then $z \nin L$ is an extreme point of $K$. Contradiction.
\end{proof}

\newlec

\begin{lem}\label{lem:nbhd-compact}
  Let $(X, \mcP)$ be a LCS. Let $K \subseteq X$ be compact and $x_0 \in K$. Then for a neighborhood $V$ of $x_0$ in $X$, there exist $f_1, \dots, f_n \in X^*$ and $\alpha_1, \dots, \alpha_n \in \R$ such that
  $$x \in \{x \in X \mid \for i, f_i(x) < \alpha_i\} \inter K \subseteq v$$
\end{lem}
\begin{proof}
  Let $\tau$ be the topology of $X$ induced by $\mcP$. Let $\sigma = \sigma(X, X^*)$. Then $\id : (K, \tau) \to (K, \sigma)$ is a continuous bijection from compact to Hausdorff. Hence it is a homeomorphism and $\sigma = \tau$ on $K$.
\end{proof}

\begin{lem}\label{lem:nbhd-compact-extreme}
  Let $(X, \mcP)$ be a LCS, $K \subseteq X$ be compact convex. Let $x_0 \in \Ext K$. Then for a neighborhood $V$ of $x_0$ in $X$ there exists $f \in X^*$ and $\alpha \in \R$ such that
  $$x_0 \in \{x \in X \mid f(x) < \alpha\}$$
\end{lem}
\begin{proof}
  Find $f_i$ and $\alpha_i$ from Lemma 3. Let $K_i = \{x \in K \mid \alpha_i \le f_i(x)\}$. This is compact and convex. $x_0 \nin V^c \subseteq \Union_i K_i$ and
  $$\conv \Union_i K_i = \curlybrack{\sum_i t_i x_i \mid x_i \in K_i, t_i \ge 0, \sum_i t_i = 1}$$
  Since $x_0$ is an extreme point of $K$, $x_0 \nin \conv \Union_i K_i$. Also,
  $$K_1 \times \dots \times K_n \times \curlybrack{t \ge 0 \mid \sum_i t_i = 1}$$
  is compact and
  $$(x_1, \dots, x_n, t) \mapsto \sum_i t_i x_i$$
  is continuous. So the image $B = \conv \Union_i K_i$ is compact. By Hahn-Banach Separation (Theorem \ref{thm:hb-separation-point}), find $f \in X^*$ such that $f(x_0) < \inf_B f$. Choose $\alpha \in \R$ such that $f(x_0) < \alpha < \inf_B f$. Then
  $$x \in {\underbrace{\{x \in X \mid f(x) < \alpha\}}_{\text{disjoint from $B$}}} \inter K \subseteq V$$
\end{proof}

\begin{thm}[Partial converse to Krein-Milman]\label{thm:krein-milman-converse}
  Let $(X, \mcP)$ be a LCS, $K \subseteq X$ compact convex, $S \subseteq K$. If $K = \overline\conv S$, then $\Ext K \subseteq \bar S$.
\end{thm}
\begin{rmk}
  The closure is necessary, eg let $S$ be a dense subset of $B_{\ell_2^2}$. Then $\overline\conv S = B_{\ell_2^2}$ and $\Ext B_{\ell_2^2} = S_{\ell_2^2}$. Also $\Ext K$ need not be closed.

  TODO: Insert picture
\end{rmk}
\begin{proof}
  Assume $x_0 \in \Ext K \setminus \bar S$. Apply Lemma \ref{lem:nbhd-compact-extreme} with $V = \bar S^c$. So there exists $f \in X^*, \alpha \in \R$ such that
  $$x_0 \in \{x \in X \mid f(x) < \alpha\} \inter K \subseteq V$$
  Then $L = \{x \in K \mid \alpha \le f(x)\}$ is compact convex with $S \subseteq L$. Hence $K = \overline\conv S \subseteq L$, contradicting $x_0 \nin L$.
\end{proof}

\begin{ex}
  $\Ext B_{C(K)^*} = \{\lambda \delta_k \mid \abs\lambda = 1, k \in K\}$ where $K$ is compact Hausdorff. Use Theorem \ref{thm:krein-milman-converse} for the inclusion.
\end{ex}

\begin{thm}[Banach-Stone]
  Let $K, L$ be compact Hausdorff spaces. Then
  $$C(K) \cong C(L) \iff K \cong L$$
\end{thm}
\begin{proof}~
  \begin{itemize}
    \item[$\Leftarrow$] If $\phi : K \to L$ is a homeomorphism, then $\phi^* : C(L) \to C(K)$ is an isometric isomorphism.
    \item[$\Rightarrow$] Let $T : C(L) \to C(K)$ be an isometric isomorphism. Then so is $T^* : C(K)^* \to C(L)^*$. So $T^*(B_{C(K)^*}) = B_{C(L)^*}$ and $T^*(\Ext B_{C(K)^*}) = \Ext B_{C(L)^*}$. Thus, for each $k \in K$,
    $$T^*(\delta_k) = \lambda(k) \delta_{\phi(k)}$$
    for some scalar $\lambda(k)$ with $\abs{\lambda(k)} = 1$ and some $\phi(k) \in L$. So we have functions
    \begin{align*}
      \lambda : K & \to \K \\
      \phi : K & \to L
    \end{align*}
    For all $k$, $\lambda(k) = T^*(\delta_k)(1) = (T 1)(k)$. So $\lambda = T 1 \in C(K)$ is continuous.

    Recall $\delta : K \to (C(K), w^*)$ is continuous (in fact a homeomorphism from $K$ to $\delta(K)$). Also, $T^* : C(K)^* \to C(L)^*$ is w*-to-w*-continuous. Hence $k \mapsto \overline{\lambda(k)}T^*(\delta_k) = \delta_{\phi(k)}$ is continuous and so is $\phi$.

    {\bf $\phi$ injective} \\
    Assume $\phi(k_1) = \phi(k_2)$. Then $\overline{\lambda(k_1)}T^*(\delta_{k_1}) = \overline{\lambda(k_2)}T^*(\delta_{k_2})$. Evaluate at $1$ to get $\lambda(k_1) = \lambda(k_2)$. As $T^*$ is injective, we get $\delta_{k_1} = \delta_{k_2}$ and hence $k_1 = k_2$.

    {\bf $\phi$ onto} \\
    Given $\ell \in L$, find by surjectivity of $T^*$ a scalar $\mu$ and $k \in K$ such that $\abs\mu = 1, T^*(\mu\delta_k) = \delta_\ell$. So $\mu \lambda(k)\delta_{\phi(k)} = \delta_\ell$. Evaluate at $1$ to get $\mu \lambda(k) = 1$. So $\phi(k) = \ell$.
  \end{itemize}
\end{proof}

\clearpage

\section{Banach Algebras}

A real/complex {\bf algebra} is a real/complex vector space $A$ with multiplication $A \times A \to A$ such that
\begin{enumerate}
  \item $a(bc) = (ab)c$
  \item $a(b + c) = ab + ac$
  \item $\lambda(ab) = (\lambda a)b = a(\lambda b)$
\end{enumerate}

$A$ is {\bf unital} if there exists an element $1 \in A$ such that $1 \ne 0$ and $\for a \in A, 1a = a1 = a$. This element is unique and is called the {\bf unit} of $A$.

An {\bf algebra norm} on $A$ is a norm on $A$ such that $\for a, b, \norm{ab} \le \norm a\norm b$. A {\bf normed algebra} is an algebra equipped with an algebra norm. A {\bf Banach algebra} is a complete normed algebra. A {\bf unital normed algebra} is a normed algebra which is unital and such that $\norm 1 = 1$ ($1 \le \norm 1$ always since $\norm 1 = \norm{1 \cdot 1} \le \norm 1 \cdot \norm 1$). A {\bf unital Banach algebra} is a complete unital normed algebra.

If $A$ is a normed algebra which is also a unital algebra (but not assuming $\norm 1 = 1$), then $\abs a = \sup_{\norm b \le 1} \norm{ab}$ defines an equivalent norm that makes $A$ into a unital normed algebra.

In the category of normed algebras, an isomorphism will mean a continuous homeomorphism with continuous inverse. But {\bf a morphism need not be continuous}.

\begin{note}
  From now on, the scalar field is $\C$.
\end{note}

\newlec

\begin{egs*}~
  \begin{enumerate}
    \item $C(K)$ with $K$ compact is a commutative unital normed algebra with pointwise multiplication and uniform norm.
    \item Let $K$ be compact Hausdorff. A {\bf uniform normed algebra} on $K$ is a closed subalgebra of $C(K)$ that separates points of $K$ and contains the constant functions (if it's further closed under complex conjugation, Stone-Weierstrass says it's everything).
    Eg, the {\bf disk algebra}
    $$A(\Delta) = \{f \in C(\Delta) \mid f \text{ holomorphic on } \interior \Delta\}$$
    where
    $$\Delta = \{z \in \C \mid \abs z \le 1\}$$
    More generally, let $K \subseteq \C$ be nonempty compact. We have the following uniform algebras on $K$:
    $$\mcP(K) \subseteq \mcR(K) \subseteq \mcO(K) \subseteq A(K) \subseteq C(K)$$
    where $\mcP(K), \mcR(K), \mcO(K)$ are the closures in $C(K)$ of polynomials, rational functions without poles in $K$, functions holomorphic on some open neighborhood of $K$ respectively, and
    $$A(K) = \{f \in C(K) \mid f \text{ holomorphic on } \interior K\}$$
    Later we will show that $\mcR(K) = \mcO(K)$ always, and
    $$\mcP(K) = \mcR(K) \iff K^c \text{ connected}$$
    In general, $A(K) \ne \mcO(K)$, and
    $$A(K) = C(K) \iff \interior K = \emptyset$$
    \item $L_1(\R)$ with $L_1$ norm and convolution
    $$(f * g)(x) = \int_\R f(y)g(x - y) dy$$
    is a commutative Banach without a unit (Riemann-Lebesgue lemma).
    \item If $X$ is a Banach algebra, then $\mcB(X)$ with composition and operator norm is a unital Banach algebra (it's not commutative if $\dim X > 1$). If $X$ is a Hilbert space, then $\mcB(X)$ is a C*-algebra.
  \end{enumerate}
\end{egs*}

\begin{dfn*}[Elementary constructions]~
  \begin{enumerate}
    \item If $A$ is a unital algebra, then a {\bf unital subalgebra} is a subalgebra $B$ of $A$ that contains $1$.
    
    If $A$ is a normed algebra, then the closure of a subalgebra of $A$ is a subalgebra.
    \item {\bf Unitisation} \\
    The unitisation of an algebra $A$ is the vector space direct sum $A_+ = A \oplus \C$ with multiplication $(a, \lambda)(b, \mu) = (ab + \lambda b + \mu a, \lambda\mu)$. Then $A_+$ is a unital algebra with unit $(0, 1)$. The ideal $\{(a, 0) \mid a \in A\}$ is isomorphic to $A$ and will always be identified with $A$. We can write
    $$A_+ = \{a + \lambda 1 \mid a \in A, \lambda \in \C\}$$
    If $A$ is a normed algebra, then $A_+$ becomes a unital normed algebra with
    $$\norm{a + \lambda 1} = \norm a + \abs\lambda$$
    Then $A$ is a closed ideal of $A_+$.

    If $A$ is a Banach algebra, then $A_+$ is a unital Banach algebra.
    \item The closure of an ideal of a normed algebra is an ideal. If $J$ is a closed ideal of the normed algebra $A$, then $A / J$ is a normed algebra in the quotient norm.
    
    If $A$ is a unital normed algebra and $J$ is a proper ($J \ne A$) closed ideal of $A$, then $A / J$ is a unital normed algebra with unit $1 + J$ ($\norm{1 + J} \le \norm 1 = 1$).
    \item Let $\tilde A$ be the Banach space completion of a normed algebra $A$. Then $\tilde A$ is a Banach algebra with the following multiplication. Given $a, b \in \tilde A$, choose $a_n, b_n$ in $A$ such that $a_n \to a, b_n \to b$. Then define $ab = \lim_n a_n b_n$. One can check this is well-defined and respects the algebra axioms.
    \item Let $A$ be a unital Banach algebra. Let $X = A$ thought of as a Banach space. FOr $a \in A$, define $L_a : X \to X, L_a(x) = ax$. Then $L_a \in \mcB(X)$ and $\norm{L_a} = \norm a$. The map $L : X \to \mcB(X)$ is an isometric unital homomorphism.
  \end{enumerate}
\end{dfn*}

\begin{lem}\label{lem:inverse-near-one}
  Let $A$ be a unital Banach algebra and $a \in A$. If $\norm{1 - a} < 1$, then $a$ is invertible and
  $$\norm{a^{-1}} \le \frac 1{1 - \norm{1 - a}}$$
\end{lem}
\begin{proof}
  For all $n$, $\norm{(1 - a)^n} \le \norm{1 - a}^n$. So $\sum_n \norm{(1 - a)^n} < \infty$. Hence $b := \sum_n (1 - a)^n$ converges absolutely. Then
  $$(1 - a)b = b(1 - a) = b - 1$$
  So $ab = ba = 1$. We see that
  $$\norm b \le \sum_n \norm{(1 - a)^n} \le \sum_n \norm{1 - a}^n = \frac 1{1 - \norm{1 - a}}$$
\end{proof}

\begin{notation}
  Let $G(A)$ denote the group of invertibles of a unital algebra $A$.
\end{notation}

\begin{cor}\label{cor:invertibles}
  Let $A$ be a unital Banach algebra.
  \begin{enumerate}
    \item $G(A)$ is open in $A$.
    \item $x \mapsto x^{-1}$ is continuous on $G(A)$.
    \item If $X_n$ in $G(A)$ converges to $x \nin G(A)$, then $\norm{x_n^{-1}} \to \infty$.
    \item If $x \in \partial G(A)$, then there exist $z_n$ in $A$ such that $\norm{z_n} = 1$ and $z_nx, xz_n \to 0$. It follows that $x$ has no left or right inverse in $A$, nor even in any unital Banach algebra $B$ containing $A$ as a (not necessarily unital) subalgebra.
  \end{enumerate}
\end{cor}
\begin{proof}~
  \begin{enumerate}
    \item Let $x \in G(A)$. If $y \in A$ and $\norm{y - x} < \norm{x^{-1}}^{-1}$, then
    $$\norm{1 - x^{-1}y} = \norm{x^{-1}(x - y)} \le \norm{x^{-1}}\norm{x - y} < 1$$
    Hence $x^{-1}y \in G(A)$ by Lemma \ref{lem:inverse-near-one}. So $y \in G(A)$.
    \item Let's fix $x \in G(A)$. For $y \in G(A)$;
    $$y^{-1} - x^{-1} = y^{-1}(x - y)x^{-1}$$
    So
    $$\norm{y^{-1} - x^{-1}} \le \norm{y^{-1}}\norm{x - y}\norm{x^{-1}}$$
    If $\norm{x - y} < \frac 1{2\norm{x^{-1}}}$, then
    $$\norm{y^{-1} - x^{-1}} \le 2\norm{x^{-1}}^2\norm{x - y} \to 0$$
    as $y \to x$.
    \item From the proof of 1, if $\norm{x - x_n} < \norm{x_n^{-1}}^{-1}$, then $x \in G(A)$. Contradiction.
    \item Given $x \in \partial G(A)$, find $x_n$ in $G(A)$ such that $x_n \to x$. By 3, $\norm{x_n^{-1}} \to \infty$. Set $z_n = \frac{x_n^{-1}}{\norm{x_n^{-1}}}$. Then
    $$z_nx = z_n x_n + z_n(x - x_n) = \norm{x_n^{-1}}^{-1} + z_n(x - x_n) \to 0$$
    as $\norm{z_n(x - x_n)} \le \norm{z_n}\norm{x - x_n} \to 0$. Similarly, $xz_n \to 0$.

    Assume $B$ is a unital Banach algebra and $A$ is a subalgebra of $B$. If $y \in B$ and $yx = 1$, then $yxz_n = z_n$. So
    $$1 = \norm{z_n} = \norm{yxz_n} \le \norm y\norm{xz_n} \to 0$$
    Similarly, we can't have $y \in B$ and $xy = 1$.
  \end{enumerate}
\end{proof}

\newlec

\begin{dfn*}
  Let $A$ be an algebra and $x \in A$. The {\bf spectrum} $\sigma_A$ of $x$ in $A$ is
  \begin{itemize}
    \item $\sigma_A(x) = \{\lambda \in \C \mid \lambda 1 - x \nin G(A)\}$ if $A$ is unital.
    \item $\sigma_A(x) = \sigma_{A_+}(x)$ if $A$ is non-unital.
  \end{itemize}
\end{dfn*}

\begin{egs*}~
  \begin{enumerate}
    \item If $A = M_n(\C)$, then $\sigma_A(x)$ is the set of eigenvalues of $x$.
    \item If $A = C(K)$ where $K$ is compact Hausdorff, then $\sigma_A(f) = f(K)$.
    \item If $A = \mcB(X)$ where $X$ is a Banach space, then
    $$\sigma_A(T) = \{\lambda \in \C \mid \lambda I - T \text{ not an isomorphism}\}$$
  \end{enumerate}
\end{egs*}

\begin{thm}\label{thm:spectrum-nonempty}
  Let $A$ be a Banach algebra, $x \in A$. Then $\sigma_A(x)$ is a nonempty compact subset of $\{\lambda \in \C \mid \abs\lambda \le \norm x\}$.
\end{thm}
\begin{proof}
  WLOG $A$ is a unital Banach algebra.

  If $\abs\lambda > \norm x$, then $\norm{\frac x\lambda} < 1$. So by Lemma \ref{lem:inverse-near-one}, $1 - \frac x\lambda \in G(A)$ and $\lambda1 - x = \lambda\left(1 - \frac x\lambda\right) \in G(A)$. Hence $\sigma_A(x) \subseteq \{\lambda \in \C \mid \abs\lambda \le \norm x\}$. Also, $\sigma_A(x)$ is the preimage of the closed set $G(A)^c$ (Corollary \ref{cor:invertibles}.i) under the continuous function $\lambda \mapsto \lambda1 - x$. Hence it is closed. It follows that $\sigma_A(x)$ is compact.

  Consider
  \begin{align*}
    f : \sigma_A(x)^c & \to A \\
    \lambda & \mapsto (\lambda 1 - x)^{-1}
  \end{align*}
  By Corollary \ref{cor:invertibles}.ii, $f$ is continuous. For $\lambda \ne \mu$,
  $$f(\lambda) - f(\mu) = f(\lambda)((\mu 1 - x) - (\lambda 1 - x))f(\mu) = (\mu - \lambda)f(\lambda)f(\mu)$$
  So
  $$\frac{f(\lambda) - f(\mu)}{\lambda - \mu} = -f(\lambda)f(\mu) \underset{\lambda \to \mu}\to -f(\mu)^2$$
  Thus $f$ is holomorphic.
  
  If $\abs\lambda > \norm x$, then $\lambda 1 - x \in G(A)$ and
  $$\norm{(\lambda 1 - x)^{-1}} = \frac 1{\abs\lambda}\norm{\left(1 - \frac x\lambda\right)^{-1}} \le \frac 1{\abs\lambda}\frac 1{1 - \norm{\frac x\lambda}} = \frac 1{\abs\lambda - \norm x} \to 0$$
  as $\abs\lambda \to \infty$. \\
  If $\sigma_A(x) = \emptyset$, then $f$ is an entire function. By vector-valued Liouville (Theorem \ref{thm:vector-liouville}), $f$ is constant. But then $f(x) = x^{-1} \ne 0$ contradicts $f(\lambda) \to 0$ as $\lambda \to 0$.
\end{proof}

\begin{cor}[Gelfand-Mazur]
  A complex unital normed division algebra $A$ ($G(A) = \{0\}^c$) is isometrically isomorphic to $\C$.
\end{cor}
\begin{proof}
  Let's define
  \begin{align*}
    \theta : \C & \to A \\
    \lambda & \mapsto \lambda 1
  \end{align*}
  Then $\theta$ is an isometric homomorphism. Is it onto?

  Fix $a \in A$ and consider $B$ the completion of $A$. $B$ is a unital Banach algebra. Hence Theorem \ref{thm:spectrum-nonempty} tells us that $\sigma_B(x)$ is nonempty, namely there exists $\lambda \in \C$ such that $\lambda 1 - x$ is not invertible in $B$. But then $\lambda 1 - x$ is not invertible in $A$, so $\lambda 1 - x = 0$ and $\theta(\lambda) = x$.
\end{proof}

\begin{dfn*}
  Let $A$ be a Banach algebra and $x \in A$. The {\bf spectral radius} $r_A(x)$ of $x$ in $A$ is
  $$r_A(x) = \sup_{\lambda \in \sigma_A(x)} \abs\lambda$$
  From Theorem \ref{thm:spectrum-nonempty}, $r_A(x)$ is well-defined and $r_A(x) \le \norm x$.
\end{dfn*}

\begin{note}
  Let $x, y$ be commuting elements of a unital algebra $A$. Then
  $$xy \in G(A) \iff x, y \in G(A)$$
  Indeed, if $z(xy) = (xy)z = 1$, then
  $$x(yz) = 1, (yz)x = zxyyzx = zyxyzx = zyx = zxy = 1$$
  So $x \in G(A)$. Similarly, $y \in G(A)$.
\end{note}

\begin{lem}[Spectral Mapping Theorem for polynomials]\label{lem:poly-smt}
  Let $A$ be a unital Banach algebra and $x \in A$. Then for a complex polynomial $p = \sum_{k = 0}^n a_k z^k$ we have
  $$\sigma_A(p(x)) = \{p(\lambda) \mid \lambda \in \sigma_A(x)\} = p(\sigma_A(x))$$
\end{lem}
\begin{proof}
  WLOG $n \ge 1$ and $a_n \ne 0$ ($\sigma_A(\lambda 1) = \{\lambda\}$). Fix $\mu \in \C$. Write
  $$\mu - p(z) = c\prod_{k = 1}^n (\lambda_k - z)$$
  for some $c, \lambda_1, \dots, \lambda_n \in \C, c \ne 0$. Note that
  $$p^{-1}(\mu) = \{\lambda_1, \dots, \lambda_n\}$$
  Now,
  \begin{align*}
    \mu \nin \sigma_A(p(x))
    & \iff \mu 1 - p(x) = c \prod_{k = 1}^n (\lambda_k 1 - x) \text{ invertible} \\
    & \iff \for k, \lambda_k 1 - x \text{ invertible (the factors commute)} \\
    & \iff \for \lambda \in \sigma_A(x), p(\lambda) \ne \mu
  \end{align*}
  The result follows.
\end{proof}

\begin{thm}[Beurling-Gelfand Spectral Radius Formula]
  Let $A$ be a Banach algebra, $x \in A$. Then
  $$r_A(x) = \lim_n \norm{x^n}^{\frac 1n} = \inf_n \norm{x^n}^{\frac 1n}$$
\end{thm}
\begin{proof}
  WLOG $A$ is unital.

  By Lemma \ref{lem:poly-smt}, if $\lambda \in \sigma_A(x)$ and $n \in \N$, then $\lambda^n \in \sigma_A(x^n)$. By Theorem \ref{thm:spectrum-nonempty}, $\abs{\lambda^n} \le \norm{x^n}$. So $\abs\lambda \le \norm{x^n}^{1/n}$. It follows that
  $$r_A(x) \le \inf_n \norm{x^n}^{1/n}$$
  Consider
  \begin{align*}
    f : \sigma_A(x)^c & \to \C \\
    \lambda & \mapsto (\lambda 1 - x)^{-1}
  \end{align*}
  By the proof of Theorem \ref{thm:spectrum-nonempty}, $f$ is holomorphic. Note
  $$\sigma_A(x)^c \supseteq \{\lambda \mid r_A(x) < \abs\lambda\} \supseteq \{\lambda \mid \norm x < \abs\lambda\}$$
  If $\abs\lambda > \norm x$, then
  $$f(\lambda) = \frac 1\lambda \left(1 - \frac x\lambda\right)^{-1} = \frac 1\lambda \sum_{n = 0}^\infty \frac{x^n}{\lambda^n}$$
  by the proof of Theorem \ref{lem:inverse-near-one}.

  Fix $\varphi \in A^*$. Then $\varphi \comp f$ is holomorphic on $\sigma_A(x)^c$ and if $\abs\lambda > \norm x$ then
  $$\varphi(f(\lambda)) = \sum_{n = 0}^\infty \varphi\left(\frac{x^n}{\lambda^n}\right)$$
  So, for $\abs\lambda > r_A(x)$ and $\varphi \in A^*$, we have $\varphi(\frac{x^n}{\lambda^n}) \to 0$. Hence $\frac{x^n}{\lambda^n} \wto 0$. By Proposition \ref{prop:weak-norm-bounded}, there exists $M > 0$ such that $\norm{\frac{x^n}{\lambda^n}} \le M$ for all $n$. So $\norm{x^n} \le M^{1/n}\abs\lambda$. Hence $\limsup \norm{x^n}^{1/n} \le \abs\lambda$.

  Thus we proved
  $$r_A(x) \le \inf_n \norm{x^n}^{1/n} \le \liminf_n \norm{x^n}^{1/n} \le \limsup \norm{x^n}^{1/n} \le r_A(x)$$
  The result follows.
\end{proof}

\begin{thm}
  Let $A$ be a unital Banach algebra and $B$ be a closed unital subalgebra of $A$. Let $x \in B$. Then
  $$\sigma_A(x) \subseteq \sigma_B(x), \quad \partial \sigma_B(x) \subseteq \partial \sigma_A(x)$$
  It follows that $\sigma_B(x)$ is the union of $\sigma_A(x)$ and some of the bounded components of $\sigma_A(x)^c$.

  INSERT PICTURE
\end{thm}
\begin{proof}
  Let $\lambda \nin \sigma_B(x)$. Then $\lambda 1 - x$ is invertible in $B$, hence is invertible in $A$ and $\lambda \nin \sigma_A(x)$.

  Assume $\lambda \in \partial \sigma_B(x) = \sigma_B(x) \setminus \interior \sigma_B(x)$. On one hand, $\lambda \nin \interior \sigma_B(x) \supseteq \interior \sigma_A(x)$. On the other hand, $\lambda 1 - x \in \partial G(B) = \overline{G(B)} \setminus G(B)$ (pick a sequence $\lambda_n \to \lambda$ with $\lambda_n \nin \sigma_B(x)$, then $\lambda_n 1 - x \to \lambda 1 - x$ and $\lambda_n 1 - x \in G(B)$), so Corollary \ref{cor:invertibles} tells us that $\lambda 1 - x$ is not invertible in any Banach superalgebra, in particular in $A$, meaning that $\lambda \in G(A)$. Hence $\lambda \in \sigma_A(x) \setminus \interior \sigma_A(x) = \partial \sigma_A(x)$.
\end{proof}

\printindex
\end{document}