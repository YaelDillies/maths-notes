\documentclass{article}

% preamble
\def\npart{III}
\def\nyear{2023}
\def\nterm{Michaelmas}
\def\nlecturer{Dr Andr\'as Zs\'ak}
\def\ncourse{Functional Analysis}
\def\draft{Incomplete}

\ifx \nauthor\undefined
  \def\nauthor{Ya\"el Dillies}
\else
\fi

\author{Based on lectures by \nlecturer \\\small Notes taken by \nauthor}
\date{\nterm\ \nyear}
\ifdefined\draft
\title{Part \npart\ -- \ncourse\ (\draft)}
\else
\title{Part \npart\ -- \ncourse}
\fi

\usepackage[utf8]{inputenc}
\usepackage{amsmath}
\usepackage{amsthm}
\usepackage{amssymb}
\usepackage{cancel}
\usepackage{enumerate}
\usepackage{mathtools}
\usepackage{fancyhdr}
\usepackage{graphicx}
\usepackage[dvipsnames]{xcolor}
\usepackage{tikz}
\usepackage{wrapfig}
\usepackage{centernot}
\usepackage{float}
\usepackage{braket}
\usepackage{marginnote}
\usepackage{mathdots}
\usepackage{mathrsfs}
\usepackage{ifthen}
\usepackage{imakeidx}
\usepackage{parskip}
\usepackage{relsize}
\usepackage{tabularx}
\usepackage[hypcap=true]{caption}
\usepackage[shortlabels]{enumitem}
\usepackage[pdftex,
  colorlinks=true,
  linkcolor=lblue,
  pdfauthor={\nauthor},
  pdfsubject={Cambridge Maths Notes: Part \npart\ - \ncourse},
  pdftitle={\ncourse - Part \npart},
pdfkeywords={Cambridge Mathematics Maths Math \npart\ \nterm\ \nyear\ \ncourse}]{hyperref}

\usepackage[capitalise,nameinlink,noabbrev]{cleveref}
\usepackage{nameref}
\usepackage[margin=1.5in,a4paper]{geometry}

\reversemarginpar
\newcommand{\lecnum}[1]{\leavevmode\marginnote{\emph{Lecture #1}}\ignorespaces}
\newcounter{lecturenumber}
\newcommand{\newlec}{\stepcounter{lecturenumber}\lecnum{\arabic{lecturenumber}}}

% Theorems
\theoremstyle{definition}
\newtheorem*{aim}{Aim}
\newtheorem*{assumption}{Assumption}
\newtheorem*{axiom}{Axiom}
\newtheorem*{claim}{Claim}
\newtheorem*{cor}{Corollary}
\newtheorem*{conjecture}{Conjecture}
\newtheorem*{defi}{Definition}
\newtheorem*{eg}{Example}
\newtheorem*{egs}{Examples}
\newtheorem*{ex}{Exercise}
\newtheorem*{fact}{Fact}
\newtheorem*{goal}{Goal}
\newtheorem*{idea}{Idea}
\newtheorem*{law}{Law}
\newtheorem*{lemma}{Lemma}
\newtheorem*{notation}{Notation}
\newtheorem*{note}{Note}
\newtheorem*{obs}{Observation}
\newtheorem*{prop}{Proposition}
\newtheorem*{properties}{Properties}
\newtheorem*{question}{Question}
\newtheorem*{rrule}{Rule}
\newtheorem*{steps}{Steps}
\newtheorem*{thm}{Theorem}

\newtheorem*{rmk}{Remark}
\newtheorem*{rmks}{Remarks}
\newtheorem*{warning}{Warning}
\newtheorem*{exercise}{Exercise}

\newtheorem{nthm}{Theorem}[section]
\newtheorem{ndef}[nthm]{Definition}
\newtheorem{nprop}[nthm]{Proposition}
\newtheorem{nconjecture}{Conjecture}
\newtheorem{ncor}[nthm]{Corollary}
\newtheorem{nex}[nthm]{Example}
\newtheorem{nlemma}[nthm]{Lemma}
\newtheorem{problem}[nthm]{Problem}

% Command redirections
\let\P\oldP
\let\oldemptyset\emptyset
\let\emptyset\varnothing

% Letter shorthands
\newcommand{\C}{\mathbb C}
\newcommand{\bbE}{\mathbb E}
\newcommand{\F}{\mathbb F}
\newcommand{\K}{\mathbb K}
\newcommand{\N}{\mathbb N}
\newcommand{\P}{\mathbb P}
\newcommand{\Q}{\mathbb Q}
\newcommand{\R}{\mathbb R}
\newcommand{\Z}{\mathbb Z}
\newcommand{\mcA}{\mathcal A}
\newcommand{\mcB}{\mathcal B}
\newcommand{\mcC}{\mathcal C}
\newcommand{\mcD}{\mathcal D}
\newcommand{\mcE}{\mathcal E}
\newcommand{\mcF}{\mathcal F}
\newcommand{\mcG}{\mathcal G}
\newcommand{\mcH}{\mathcal H}
\newcommand{\mcM}{\mathcal M}
\newcommand{\mcN}{\mathcal N}
\newcommand{\mcO}{\mathcal O}
\newcommand{\mcP}{\mathcal P}
\newcommand{\mcQ}{\mathcal Q}
\newcommand{\mcR}{\mathcal R}
\newcommand{\mcS}{\mathcal S}
\newcommand{\mcT}{\mathcal T}
\newcommand{\mcU}{\mathcal U}
\newcommand{\mcV}{\mathcal V}
\newcommand{\eps}{\varepsilon}
\newcommand{\Eps}{\mathcal E}

\newcommand{\curlybrack}[1]{\left\{ #1\right\}}
\newcommand{\abs}[1]{\left\lvert #1\right\rvert}
\newcommand{\norm}[1]{\left\lVert #1\right\rVert}
\newcommand{\inn}[2]{\left\langle #1, #2\right\rangle}
\newcommand{\floor}[1]{\left\lfloor #1\right\rfloor}
\newcommand{\ceil}[1]{\left\lceil #1\right\rceil}
\newcommand{\doublesqbrack}[1]{[\![#1]\!]}

\newcommand{\imp}{\implies}
\newcommand{\for}{\forall}
\newcommand{\mor}{\rightarrow}
\newcommand{\nin}{\notin}
\newcommand{\comp}{\circ}
\newcommand{\union}{\cup}
\newcommand{\inter}{\cap}
\newcommand{\Union}{\bigcup}
\newcommand{\Inter}{\bigcap}
\newcommand{\hatplus}{\mathbin{\widehat{+}}}
\newcommand{\symdif}{\mathbin\varbigtriangleup}
\newcommand{\aeeq}{\overset{\text{ae}}=}
\newcommand{\lexlt}{\overset{\text{lex}}<}
\newcommand{\colexlt}{\overset{\text{colex}}<}
\newcommand{\wtendsto}{\overset w\mor}
\newcommand{\wstartendsto}{\overset{w*}\mor}
\renewcommand{\vec}[1]{\boldsymbol{\mathbf{#1}}}
\renewcommand{\bar}[1]{\overline{#1}}

\newcommand*{\E}{
  \mathop{
    \mathchoice{\vcenter{\hbox{\larger[4]$\mathbb{E}$}}}
               {\kern0pt\mathbb{E}}
               {\kern0pt\mathbb{E}}
               {\kern0pt\mathbb{E}}
  }\displaylimits
}

\newcommand{\named}[1]{\textbf{#1}\index{#1}}
\newcommand{\bonusnamed}[1]{\textbf{#1}\index{#1@*#1}}

\let\Im\relax
\let\Re\relax

\DeclareMathOperator{\Ber}{Ber}
\DeclareMathOperator{\conv}{conv}
\DeclareMathOperator{\diam}{diam}
\DeclareMathOperator{\codim}{codim}
\DeclareMathOperator{\esssup}{ess sup}
\DeclareMathOperator{\Ext}{Ext}
\DeclareMathOperator{\id}{id}
\DeclareMathOperator{\im}{im}
\DeclareMathOperator{\Im}{Im}
\DeclareMathOperator{\interior}{int}
\DeclareMathOperator{\lhs}{LHS}
\DeclareMathOperator{\rank}{rank}
\DeclareMathOperator{\Re}{Re}
\DeclareMathOperator{\rhs}{RHS}
\DeclareMathOperator{\Span}{Span}
\DeclareMathOperator{\Spec}{Spec}
\DeclareMathOperator{\supp}{supp}
\DeclareMathOperator{\Var}{Var}

\definecolor{lblue}{rgb}{0., 0.05, 0.6}
\definecolor{mblue}{rgb}{0.2, 0.3, 0.8}
\definecolor{morange}{rgb}{1, 0.5, 0}
\definecolor{mgreen}{rgb}{0.1, 0.4, 0.2}
\definecolor{mred}{rgb}{0.5, 0, 0}

\colorlet{bred}{red}
\colorlet{bblue}{Cyan!50!blue}
\colorlet{byellow}{yellow}
\colorlet{bgreen}{YellowGreen!50!Green}
\colorlet{borange}{red!20!yellow}
\colorlet{bpurple}{violet}

\newcommand{\red}[1]{\textcolor{bred}{#1}}
\newcommand{\green}[1]{\textcolor{bgreen}{#1}}
\newcommand{\blue}[1]{\textcolor{bblue}{#1}}
\newcommand{\yellow}[1]{\textcolor{byellow}{#1}}
\newcommand{\orange}[1]{\textcolor{borange}{#1}}
\newcommand{\purple}[1]{\textcolor{bpurple}{#1}}

\pagestyle{fancy}
\fancyhf{}
\fancyfoot[R]{\href{yaeldillies.github.io/maths-notes}{\color{lblue}{Updated online}}}
\fancyfoot[C]{\thepage}
\ifdefined\draft
\fancyfoot[L]{\emph{\draft}}
\else
\fi
\renewcommand{\headrulewidth}{0pt}
\renewcommand{\footrulewidth}{0.2pt}

% Counters and table of content

\swapnumbers
\reversemarginpar

\usetikzlibrary{positioning, decorations.pathmorphing, decorations.text, calc, backgrounds, fadings}
\tikzset{node/.style = {circle,draw,inner sep=0.8mm}}

\makeindex[intoc]
\swapnumbers
\reversemarginpar

\usetikzlibrary{positioning, decorations.pathmorphing, decorations.text, calc, backgrounds, fadings}
\tikzset{node/.style = {circle,draw,inner sep=0.8mm}}

\makeindex[intoc]

\setcounter{section}{-1}

% and here we go!
\begin{document}
\maketitle

\tableofcontents

\clearpage

\section{Introduction}

\subsection*{Prerequisites}

\begin{itemize}
  \item some basic functional analysis
  \item a bit of measure theory
  \item a bit of complex analysis
\end{itemize}

\subsection*{Books}

Books relevant to the course are:
\begin{itemize}
  \item Bollob\'as, \textit{Linear Analysis}
  \item Murphy, \textit{$C^*$-algebras}
  \item Rudin
  \item Graham-Allan
\end{itemize}

\subsection*{Notation}

We will use $\K$ to mean "either $\R$ or $\C$".

For $X$ a normed space, we define
$$B_X = \{x \in X | \norm x \le 1\}$$
$$S_X = \{x \in X | \norm x = 1\}$$

For $X, Y$ normed spaces, we write $X \sim Y$ if $X, Y$ are isomorphic, ie there
exists a linear bijection $T : X \mor Y$ such that $T$ and $T^{-1}$ are continuous. We write $X \cong Y$ if $X, Y$ are isometrically isomorphic, ie there exists a surjective linear map $T : X \mor Y$ such that $\norm{T x} = \norm x$ for all $x$.

\clearpage

\section{Hahn-Banach extension theorems}

\newlec

Let $X$ be a normed space. The {\bf dual space} of $X$ is the space $X^*$ of
bounded linear functionals on $X$.
$X^*$ is always a Banach space in the operator norm: for $f \in X^*$,
$$\norm f = \sup_{x \in B_X} |f(x)|$$

\begin{eg}
  For $1 < p, q < \infty$, $p^{-1} + q^{-1} = 1$, $\ell_p^* \cong \ell_q$. \\
  We also have $\ell_1^* \cong \ell_\infty$, $c_0^* \cong \ell_1$. \\
  If $H$ is a Hilbert space, then $H^* \cong H$ (the isomorphism is conjugate-linear in the complex case).
\end{eg}

For $x \in X, f \in X^*$, we write $\langle x, f\rangle = f(x)$. Note that
$$\langle x, f\rangle = |f(x)| \le \norm f\norm x$$

\begin{defi}
  Let $X$ be a {\it real} vector space. A functional $p : X \mor \R$ is
  \begin{itemize}
    \item {\bf positive homogeneous} if $p(tx) = tp(x)$ for all $x \in X$, $t \ge 0$
    \item {\bf subadditive} if $p(x + y) \le p(x) + p(y)$ for all $x, y \in X$
  \end{itemize}
\end{defi}

\begin{defi}
  Let $P$ be a preorder, $A \subseteq P, x \in P$. We say
  \begin{itemize}
    \item $x$ is an {\bf upper bound} for $A$ if $\for a \in A, a \le x$.
    \item $A$ is a {\bf chain} if $\for a, b \in A, a \le b \lor b \le a$.
    \item $x$ is a {\bf maximal element} if $\for y \in P, x \not < y$
  \end{itemize}
\end{defi}

\begin{fact}[Zorn's lemma]
  A nonempty preorder in which all nonempty chains have an upper bound has a maximal element.
\end{fact}

\begin{thm}[Hahn-Banach, positive homogeneous version]\label{thm:hb-positive}
  Let $X$ be a real vector space and $p : X \mor \R$ be positive homogeneous and subadditive. Let $Y$ be a subspace of $X$ and $g : Y \mor \R$ be linear such that $\for y \in Y, g(y) \le p(y)$. Then there exists $f : X \mor \R$ linear such that $f\restriction_Y = g$ and $\for x \in X, f(x) \le p(x)$.
\end{thm}
\begin{proof}
  Let $P$ be the set of pairs $(Z, h)$ where $Z$ is a subspace of $X$ with $Y \subseteq Z$ and $h : Z \mor \R$ linear, $h\restriction_Y = g$ and $\for z \in Z, h(z) \le p(z)$. $P$ is nonempty since $(Y, g) \in P$, and is partially ordered by
  $$(Z_1, h_1) \le (Z_2, h_2) \iff Z_1 \subseteq Z_2 \land h_2\restriction_{Z_1} = h_1$$
  If $\{(Z_i, h_i) | i \in I\}$ is a chain with $I$ nonempty, then we can define
  $$Z := \bigcup_{i \in I} Z_i, h\restriction_{Z_i} = h_i$$
  The definition of $h$ makes sense thanks to the chain assumption. $(Z, h) \in P$ is therefore an upper bound for the chain. \\
  Hence find by Zorn a maximal element $(Z, h)$ of $P$. If $Z = X$, we won. So assume there is some $x \in X \ Z$. Let $W = \Span(Z \cup \{x\})$ and define $f : W \mor \R$ by
  $$f(z + \lambda x) = h(z) + \lambda\alpha$$
  for some $\alpha \in \R$. Then $f$ is linear and $f\restriction_Z = h$. We now look for $\alpha$ such that $\for w \in W, f(w) \le p(w)$. We would then have $(W, f) \in P$ and $(Z, h) < (W, f)$, contradicting maximality of $(Z, h)$. \\
  We need
  $$h(z) + \lambda\alpha \le p(z + \lambda x) \for z \in Z, \lambda \in \R$$
  Since $p$ is positive homogeneous, this becomes
  \begin{align}
    h(z) + \alpha \le p(z + x)
    h(z) - \alpha \le p(z - x)
  \end{align}
  ie
  $$h(z) - p(z - x) \le \alpha \le p(z + x) - h(z) \for z \in Z$$
  The existence of $\alpha$ now amounts to
  $$h(z_1) - p(z_1 - x) \le \alpha \le p(z_2 + x) - h(z_2) \for z_1, z_2 \in Z$$
  But indeed
  $$h(z_1) + h(z_2) = h(z_1 + z_2) \le p(z_1 + z_2) \le p(z_1 - x) + p(z_2 + x)$$
\end{proof}

\begin{defi}
  Let $X$ be a $\K$-vector space. A {\bf seminorm} on $X$ is a functional $p : X \mor \R$ such that
  \begin{itemize}
    \item $\for x \in X, p(x) \ge 0$
    \item $\for x \in X, \lambda \in \K, p(\lambda x) = |\lambda| p(x)$
    \item $\for x, y \in X, p(x + y) \le p(x) + p(y)$
  \end{itemize}
\end{defi}

\begin{rmk}
  $$\text{norm} \implies \text{seminorm} \implies \text{positive homogeneous}$$
\end{rmk}

\newlec

\begin{thm}[Hahn-Banach, absolute homogeneous version]\label{thm:hb-absolute}
  Let $X$ be a real of complex vector space and $p$ a seminorm on $X$. Let $Y$ be a subspace of $X$, $g$ a linear functional on $Y$ such that $\for y \in Y, |g(y)| \le p(y)$. Then there exists a linear functional $f$ on $X$ such that $f\restriction_Y = g$ and $\for x \in X, |f(x)| \le p(x)$.
\end{thm}
\begin{proof}~\\
  {\bf Real case}
  $$\for y \in Y, g(y) \le |g(y)| \le p(y)$$
  By Theorem \ref{thm:hb-positive}, there exists $f : X \mor \R$ such that $f\restriction_Y = g$ and $\for x \in X, f(x) \le p(x)$. We also have
  $$\for x \in X, -f(x) = f(-x) \le p(-x) = p(x)$$
  Hence $|f(x)| \le p(x)$ \\
  {\bf Complex case} \\
  $\Re g : Y \mor \R$ is real-linear.
  $$\for y \in Y, |\Re g(y)| \le |g(y)| \le p(y)$$
  By the real case, find $h : X \mor \R$ real-linear such that $h\restriction_Y = \Re g$
  \begin{claim}
    There exists a unique complex-linear $f : X \mor \C$ such that $h = \Re f$.
  \end{claim}
  \begin{proof}~\\
    {\bf Uniqueness} \\
    If we have such $f$, then
    \begin{eqnarray*}
      f(x)
      & = & \Re f(x) + i\Im f(x) \\
      & = & \Re f(x) - i\Re f(ix) \\
      & = & h(x) - ih(ix)
    \end{eqnarray*}
    {\bf Existence} \\
    Define $f(x) = h(x) - ih(ix)$. Then $f$ is real-linear and $f(ix) = if(x)$, so $f$ is complex-linear with $\Re f = h$.
  \end{proof}
  We now have $f : X \mor \C$ such that $\Re f = h$.
  $$\Re f\restriction_Y = h\restriction_Y = \Re g$$
  So, by uniqueness, $f\restriction_Y = g$. \\
  Given $x \in X$, find $\lambda$ with $|\lambda| = 1$ such that
  \begin{eqnarray*}
    |f(x)|
    & = & \lambda f(x) \\
    & = & f(\lambda x) \\
    & = & \Re f(\lambda x) \\
    & = & h(\lambda x) \\
    & \le & p(\lambda x) \\
    & = & p(x)
  \end{eqnarray*}
\end{proof}

\begin{rmk}
  For a complex vector space $X$, if we write $X_\R$ for $X$ considered as a real vector space, the above proof shows that
  $$\Re : (X^*)_\R \mor X_\R^*$$
  is an isometric isomorphism.
\end{rmk}

\begin{cor}\label{cor:hb-point}
  Let $X$ be a $\K$-vector space, $p$ a seminorm on $X$, $x_0 \in X$. Then there exists a linear functional $f$ on $X$ such that $f(x_0) = p(x_0)$ and $\for x \in X, |f(x)| \le p(x)$.
\end{cor}
\begin{proof}
  Let $Y = \Span(x_0)$,
  \begin{eqnarray*}
    g : Y & \mor & \K \\
    \lambda x_0 & \mapsto & \lambda p(x_0)
  \end{eqnarray*}
  We see that $\for y \in Y, g(y) \le p(y)$. Hence find by Theorem \ref{thm:hb-absolute} a linear functional $f$ on $X$ such that $f\restriction_Y = g$ and $\for x \in X, |f(x)| \le p(x)$. We check that $f(x_0) = g(x_0) = p(x_0)$.
\end{proof}

\begin{thm}[Hahn-Banach, existence of support functionals]
  Let $X$ be a real or complex normed space. Then
  \begin{enumerate}
    \item If $Y$ is a subspace of $X$ and $g \in Y^*$, then there exists $f \in X^*$ such that $f\restriction_Y = g$ and $\norm f = \norm g$.
    \item Given $x_0 \ne 0$, there exists $f \in S_{X^*}$ such that $f(x_0) = \norm{x_0}$.
  \end{enumerate}
\end{thm}
\begin{proof}~
  \begin{enumerate}
    \item Let $p(x) = \norm g \norm x$. Then $p$ is a seminorm on $X$ and
      $$\for y \in Y, |g(y)| \le \norm g \norm y = p(y)$$
      Find by Theorem \ref{thm:hb-positive} a linear functional $f$ on $X$ such that $f\restriction_Y = g$ and $\for x \in X, |f(x)| \le p(x) = \norm g \norm x$. So $\norm f \le \norm g$. Since $f\restriction_Y = g$, we also have $\norm g \le \norm f$. Hence $\norm f = \norm g$.
    \item Apply Corollary \ref{cor:hb-point} with $p(x) = \norm x$ to get $f \in X^*$ such that
      $$\for x \in X, |f(x)| \le \norm x \text{ and } f(x_0) = \norm{x_0}$$
      It follows that $\norm f = 1$.
  \end{enumerate}
\end{proof}

\begin{rmks}~
  \begin{itemize}
    \item Part 1 is a sort of linear version of Tietze's extension theorem: Given $K$ compact Hausdorff, $L \subseteq K$ closed, $g : L \mor \K$ continuous, there exists $f : K \mor \K$ such that $f\restriction_L = g$ and $\norm f_\infty = \norm g_\infty$.
    \item Part 2 shows that for all $x \ne y$ in $X$ there exists $f \in X^*$ such that $f(x) \ne f(y)$, namely $X^*$ {\bf separates points} of $X$. This is a sort of linear version of Urysohn: $C(K)$ separates points of $K$.
    \item The $f$ in part 2 is called a {\bf norming functional}, aka {\bf support functional}, for $x_0$. The existence of support functionals shows that
      $$x_0 = \max_{g \in B_{X^*}} \langle x_0, g\rangle$$
      Assuming $X$ is a real normed space and $\norm{x_0} = 1$, we have $B_X \subseteq \{x \in X| f(x) \le 1\}$. Visually, TODO: insert tangency diagram
  \end{itemize}
\end{rmks}

\subsection{Bidual}

Let $X$ be a normed space. Then $X^{**}$ is called the {\bf bidual} or {\bf second dual} of $X$.

For $x \in X$, define $\hat x : X^* \mor \K$, the {\bf evaluation at $x$}, by $\hat x(f) = f(x)$. $\hat x$ is linear and $|\hat x(f)| = |f(x)| \le \norm f \norm x$, so $\hat x \in X^{**}$ and $\norm{\hat x} \le \norm x$.

The map $x \mapsto \hat x : X \mor X^{**}$ is called the {\bf canonical embedding} of $X$ into $X^{**}$.

\begin{thm}\label{thm:can-emb}
  The canonical embedding is an isometric embedding.
\end{thm}
\begin{proof}~\\
  {\bf Linearity}
  \begin{eqnarray*}
    \widehat{x + y}(f) & = & f(x + y) = f(x) + f(y) = \hat x(f) + \hat y(f) \\
    \widehat{\lambda x}(f) & = & f(\lambda x) = \lambda f(x) = \lambda \hat x(f)
  \end{eqnarray*}
  {\bf Isometry} \\
  If $x \ne 0$, there exists a support functional $f$ for $x$. Then
  $$ \norm{\hat x} \ge |\hat x(f)| = |f(x)| = \norm x$$
\end{proof}

\begin{rmks}~
  \begin{itemize}
    \item In bracket notation, $\langle f, \hat x\rangle = \langle x, f\rangle$
    \item Let $\hat X$ be the image of $X$ in $X^{**}$. Theorem \ref{thm:can-emb} says
      $$X \cong \hat X \subseteq X^{**}$$
      We often identify $\hat X$ with $X$ and think of $X$ as living isometrically inside $X^{**}$. Note that
      $$X \text{ complete } \iff \hat X \text{ closed in } X^{**}$$
    \item More generally, $\bar{\hat X}$ is a Banach space containing an isometric copy of $X$ as a dense subspace. We proved that normed spaces have completions!
  \end{itemize}
\end{rmks}

\begin{defi}
  A normed space $X$ is {\bf reflexive} if the canonical embedding $X \mor X^{**}$ is surjective.
\end{defi}

\begin{egs}~
  \begin{itemize}
    \item Some reflexive spaces are Hilbert spaces, finite-dimensional spaces, $\ell_p$ and $L_p(\mu)$ for $1 < p < \infty$.
    \item Some non-reflexive spaces are $c_0, \ell_1, \ell_\infty, L_1[0, 1]$.
  \end{itemize}
\end{egs}

\begin{rmks}~
  \begin{itemize}
  \item If $X$ is reflexive, then $X \cong X^{**}$, so $X$ is complete.
  \item There are Banach spaces $X$ such that $X \cong X^{**}$ but $X$ is not reflexive, eg {\bf James' space}. Any isomorphism to the bidual is then necessarily not the canonical embedding.
  \end{itemize}
\end{rmks}

\newlec

\printindex
\end{document}