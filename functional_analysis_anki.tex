% To use these notes, you must copy anki_header.tex
% into the header of your card type in Anki

% layout in Anki:
\documentclass[10pt]{article}
\usepackage[a4paper]{geometry}
\geometry{paperwidth=.5\paperwidth,paperheight=100in,left=2em,right=2em,bottom=1em,top=2em}
\pagestyle{empty}
\setlength{\parindent}{0in}

% encoding:
\usepackage[T1]{fontenc}
\usepackage[utf8]{inputenc}
\usepackage{lmodern}

% packages:
\usepackage{amsmath}
\usepackage{amsfonts}
\usepackage{amsthm}
\usepackage{amssymb}
\usepackage{centernot}
\usepackage{parskip}

% Theorem-like environments
\theoremstyle{definition}
\newtheorem*{claim}{Claim}
\newtheorem*{conjecture}{Conjecture}

% Command redirections
\let\P\oldP
\let\oldemptyset\emptyset
\let\emptyset\varnothing

% Letter shorthands
\newcommand{\C}{\mathbb C}
\newcommand{\E}{\mathbb E}
\newcommand{\F}{\mathbb F}
\newcommand{\K}{\mathbb K}
\newcommand{\N}{\mathbb N}
\newcommand{\P}{\mathbb P}
\newcommand{\Q}{\mathbb Q}
\newcommand{\R}{\mathbb R}
\newcommand{\Z}{\mathbb Z}
\newcommand{\mcA}{\mathcal A}
\newcommand{\mcB}{\mathcal B}
\newcommand{\mcC}{\mathcal C}
\newcommand{\mcD}{\mathcal D}
\newcommand{\mcE}{\mathcal E}
\newcommand{\mcF}{\mathcal F}
\newcommand{\mcG}{\mathcal G}
\newcommand{\mcH}{\mathcal H}
\newcommand{\mcM}{\mathcal M}
\newcommand{\mcN}{\mathcal N}
\newcommand{\mcO}{\mathcal O}
\newcommand{\mcP}{\mathcal P}
\newcommand{\mcQ}{\mathcal Q}
\newcommand{\mcR}{\mathcal R}
\newcommand{\mcS}{\mathcal S}
\newcommand{\mcT}{\mathcal T}
\newcommand{\mcU}{\mathcal U}
\newcommand{\mcV}{\mathcal V}
\newcommand{\eps}{\varepsilon}
\newcommand{\Eps}{\mathcal E}

\newcommand{\curlybrack}[1]{\left\{ #1\right\}}
\newcommand{\abs}[1]{\left\lvert #1\right\rvert}
\newcommand{\norm}[1]{\left\lVert #1\right\rVert}
\newcommand{\inn}[2]{\left\langle #1, #2\right\rangle}
\newcommand{\floor}[1]{\left\lfloor #1\right\rfloor}
\newcommand{\ceil}[1]{\left\lceil #1\right\rceil}
\newcommand{\doublesqbrack}[1]{[\![#1]\!]}

\newcommand{\imp}{\implies}
\newcommand{\for}{\forall}
\newcommand{\nin}{\notin}
\newcommand{\comp}{\circ}
\newcommand{\union}{\cup}
\newcommand{\inter}{\cap}
\newcommand{\Union}{\bigcup}
\newcommand{\Inter}{\bigcap}
\newcommand{\hatplus}{\mathbin{\widehat{+}}}
\newcommand{\symdif}{\mathbin\varbigtriangleup}
\newcommand{\aeeq}{\overset{\text{ae}}=}
\newcommand{\lexlt}{\overset{\text{lex}}<}
\newcommand{\colexlt}{\overset{\text{colex}}<}
\newcommand{\wto}{\overset w\to}
\newcommand{\wstarto}{\overset{w*}\to}
\renewcommand{\vec}[1]{\boldsymbol{\mathbf{#1}}}
\renewcommand{\bar}[1]{\overline{#1}}

\let\Im\relax
\let\Re\relax

\DeclareMathOperator{\Ber}{Ber}
\DeclareMathOperator{\conv}{conv}
\DeclareMathOperator{\diam}{diam}
\DeclareMathOperator{\codim}{codim}
\DeclareMathOperator{\esssup}{ess sup}
\DeclareMathOperator{\Ext}{Ext}
\DeclareMathOperator{\id}{id}
\DeclareMathOperator{\Im}{Im}
\DeclareMathOperator{\interior}{int}
\DeclareMathOperator{\lhs}{LHS}
\DeclareMathOperator{\rank}{rank}
\DeclareMathOperator{\Re}{Re}
\DeclareMathOperator{\rhs}{RHS}
\DeclareMathOperator{\Span}{Span}
\DeclareMathOperator{\Spec}{Spec}
\DeclareMathOperator{\supp}{supp}
\DeclareMathOperator{\Var}{Var}

%  pdf layout:
\geometry{paperheight=74.25mm}
\usepackage{pgfpages}
\pagestyle{empty}
\pgfpagesuselayout{8 on 1}[a4paper,border shrink=0cm]
\makeatletter
\@tempcnta=1\relax
\loop\ifnum\@tempcnta<9\relax
\pgf@pset{\the\@tempcnta}{bordercode}{\pgfusepath{stroke}}
\advance\@tempcnta by 1\relax
\repeat
\makeatother

%  notes, fields, tags:
\def \ifempty#1{\def\temp{#1} \ifx\temp\empty }
\newcommand{\xfield}[1]{
        #1\par
        \vfill
        {\tiny\texttt{\parbox[t]{\textwidth}{\localtag\hfill\\\globaltag\hfill\uuid}}}
        \newpage}
\newenvironment{field}{}{\newpage}
\newif\ifnote
\newenvironment{note}{\notetrue}{\notefalse}
\newcommand{\localtag}{}
\newcommand{\globaltag}{}
\newcommand{\uuid}{}
\newcommand{\tags}[1]{
    \ifnote
        \renewcommand{\localtag}{#1}
    \else
        \renewcommand{\globaltag}{#1}
    \fi
    }
\newcommand{\xplain}[1]{
  \label{#1} % make sure there's no duplicate label
  \renewcommand{\uuid}{#1} % update the UUID for display and Anki disambiguation
  }

\begin{document}

% Lecture 1
% Lecture 2
% Lecture 3
% Lecture 4
% Lecture 5
% Lecture 6
% Lecture 7
% Lecture 8
% Lecture 9
% Lecture 10
% Lecture 11

\begin{note}
  \tags{boundedness, norm-topology}
  \xplain{principle-uniform-Boundedness}
  \xfield{Principle of Uniform Boundedness}
  \begin{field}
    If $\mathcal T \subseteq X^*$ is pointwise bounded ($\for x, \sup_{T \in \mathcal T} \norm{Tx} < \infty$), then it is uniformly bounded ($\sup_{T \in \mathcal T} \norm T < \infty$).
  \end{field}
\end{note}

\begin{note}
  \tags{boundedness, weak-topology, norm-topology}
  \xplain{weak-bounded-implies-norm-bounded}
  \xfield{If $A \subseteq X$ is weak-bounded, then it is norm-bounded.}
  \begin{field}
    This is exactly PUB applied to $\hat A = \{\hat x \mid x \in A\}$.
  \end{field}
\end{note}

\begin{note}
  \tags{boundedness, weak-star-topology, norm-topology}
  \xplain{weak-star-bounded-implies-norm-bounded}
  \xfield{If $B \subseteq X^*$ is w*-bounded, then it is norm-bounded.}
  \begin{field}
    This is exactly PUB applied to $B$.
  \end{field}
\end{note}

% Lecture 12

\begin{note}
  \tags{convexity, norm-topology, weak-topology}
  \xplain{mazur}
  \xfield{Mazur's theorem}
  \begin{field}
    Let $C$ be a convex set in a normed space. Then $\bar C^{\norm\cdot} = \bar C^{\text w}$. In particular,
    $$C \text{ norm-closed} \iff C \text{ w-closed}$$
    \begin{proof}
      WLOG $C$ is nonempty. We already know $\bar C^{\norm\cdot} \subseteq \bar C^{\text w}$ as the weak topology is weaker than the norm-topology. \\
      If $x \nin \bar C^{\norm\cdot}$, then Hahn-Banach with $A = \{x\}$ and $B = \bar C^{\norm\cdot}$ gives us $f \in X^*$ such that $f(x) < \inf_B f$. Then $\{z \mid f(z) < \inf_B f\}$ is a w-open neighborhood of $x$ disjoint from $B$. So $x \nin \bar C^{\text w}$.
    \end{proof}
  \end{field}
\end{note}

% Lecture 13

% Lecture 14

% Lecture 15

% Lecture 16

\begin{note}
  \tags{}
  \xplain{p-r-o-a-def}
  \xfield{Definitions of $\mathcal P(K), \mathcal R(K), \mathcal O(K), A(K)$}
  \begin{field}
    \begin{align*}
      \mathcal P(K) & = \overline{\{f \in C(K) \mid f \text{ polynomial}\}} \\
      \mathcal R(K) & = \overline{\{f \in C(K) \mid f \text{ rational function without poles}\}} \\
      \mathcal O(K) & = \overline{\{f \in C(K) \mid f \text{ holomorphic on a nhbd of } K\}} \\
      A(K) & = \{f \in C(K) \mid f \text{ is holomorphic on } \interior K\}
    \end{align*}
  \end{field}
\end{note}

\begin{note}
  \tags{}
  \xplain{p-r-o-a-inclusions}
  \xfield{Inclusions between $\mathcal P(K), \mathcal R(K), \mathcal O(K), A(K)$}
  \begin{field}
    $$\mathcal P(K) \subseteq \mathcal R(K) \subseteq \mathcal O(K) \subseteq A(K) \subseteq C(K)$$
    \begin{align*}
      \mathcal P(K) = \mathcal R(K) & \iff K^c \text{ connected} \\
      \mathcal R(K) = \mathcal O(K) & \text{ always} \\
      \mathcal O(K) \ne A(K) & \text{ in general} \\
      \mathcal A(K) = C(K) & \iff \interior K = \emptyset
    \end{align*}
  \end{field}
\end{note}

\begin{note}
  \tags{}
  \xplain{closed-subalgebra-b}
  \xfield{Any Banach algebra $A$ is a closed subalgebra of $\mathcal B(X)$ for some $X$.}
  \begin{field}
    WLOG $A$ is unital. For $a \in A$, consider the map
    \begin{align*}
      L_a : A & \to A \\
      b & \mapsto ab
    \end{align*}
    $L_a \in \mathcal B(A)$ and $\norm{L_a} = \norm a$. Hence
    $$L : A \to \mathcal B(A)$$
    is a unital isometric homomorphism.
  \end{field}
\end{note}

% Lecture 17

\begin{note}
  \tags{spectrum}
  \xplain{spectrum-compact}
  \xfield{Let $A$ be a Banach algebra and let $x \in A$. Then $\sigma_A(x)$ is a compact subset of
  $$\{\lambda \in \C \mid \abs\lambda \le \norm x\}$$}
  \begin{field}
    First, if $\abs\lambda > \norm x$, then $\norm{\frac x\lambda} < 1$ and $1 - \frac x\lambda$ is invertible. So $\lambda 1 - x$ is invertible and $\lambda \nin \sigma_A(x)$.

    $\sigma_A(x)$ is the preimage of the closed set $G(A)^c$ under the continuous map $\lambda \mapsto \lambda 1 - x$, hence is closed. Since it is bounded, it is also compact.
  \end{field}
\end{note}

\begin{note}
  \tags{spectrum}
  \xplain{spectrum-nonempty}
  \xfield{Let $A$ be a normed algebra and let $x \in A$. Then $\sigma_A(x)$ is nonempty.}
  \begin{field}
    WLOG $A$ is a Banach algebra. If $\sigma_A(x)$ is empty, then
    \begin{align*}
      f : \C & \to A \\
      \lambda & \mapsto (\lambda 1 - x)^{-1}
    \end{align*}
    is holomorphic since it is continuous and $f(\lambda) - f(\mu) = (\mu - \lambda)f(\lambda)f(\mu)$, namely
    $$\frac{f(\lambda) - f(\mu)}{\lambda - \mu} \underset{\lambda \to \mu}\to -f(\mu)^2$$
    Also, as $\abs\lambda \to \infty$,
    $$\norm{f(\lambda)} \le \frac 1{\abs\lambda - \norm x} \to 0$$
    meaning that $f$ is bounded. By vector-valued Liouville, $f$ is constant, which is clearly nonsense.
  \end{field}
\end{note}

\begin{note}
  \tags{division-algebra}
  \xplain{gelfand-Mazur}
  \xfield{Gelfand-Mazur theorem}
  \begin{field}
    Any complex unital normed division algebra is isomorphic to $\C$.
    \begin{proof}
      Consider
      \begin{align*}
        f : \C & \to A \\
        \lambda & \mapsto \lambda 1
      \end{align*}
      $f$ is an isometric homomorphism. Since $\sigma_A(x)$ is nonempty, there is some $\lambda$ such that $\lambda 1 - x$ is not invertible, namely $\lambda 1 = x$ and $f(\lambda) = x$. So $f$ is surjective.
    \end{proof}
  \end{field}
\end{note}

\begin{note}
  \tags{spectrum}
  \xplain{spectral-mapping-polynomial}
  \xfield{Spectral Mapping Theorem for polynomials}
  \begin{field}
    Let $A$ be a unital Banach algebra, $x \in A$, $p$ a polynomial. Then
    $$\sigma_A(p(x)) = p(\sigma_A(x))$$
    \begin{proof}
      For a fixed $\mu \in \C$, write $\mu - p(z) = c\prod_{i = 1}^n (\lambda_i - z)$ for some $c \ne 0$ and some $\lambda_1, \dots, \lambda_n$. Then
      \begin{align*}
        \mu \nin \sigma_A(p(x))
        & \iff \mu 1 - p(x) = c\prod_{i = 1}^n (\lambda_i 1 - x) \text{ invertible} \\
        & \iff \for i,\lambda_i 1 - x \text{ invertible} \\
        & \iff \for \lambda \in \sigma_A(x), \for i, \lambda_i \ne \lambda \\
        & \iff \for \lambda \in \sigma_A(x), \mu - p(\lambda) \ne 0
      \end{align*}
    \end{proof}
  \end{field}
\end{note}

\begin{note}
  \tags{spectrum:spectral-radius}
  \xplain{spectral-radius-formula}
  \xfield{Beurling-Gelfand Spectral Radius Formula}
  \begin{field}
    $$r_A(x) = \lim_n \norm{x^n}^{1/n} = \inf_n \norm{x^n}^{1/n}$$
    \begin{proof}
      If $\lambda \in \sigma_A(x)$, then $\lambda^n \in \sigma_A(x^n)$. So $\abs\lambda \le \norm{x^n}^{1/n}$.

      Let's show $\frac{x^n}{\lambda^n} \wto 0$ if $\abs\lambda > r_A(x)$. This implies $\norm{x^n}^{1/n} \le C^{1/n}\abs\lambda$ for some $C$. Let $\varphi \in A^*$. Define $f : \C \to \C, \lambda \mapsto \varphi((\lambda 1 - x)^{-1})$. Observe that for all $\abs\lambda > \norm x$ we have the Laurent series
      $$f(\lambda) = \frac 1\lambda \sum_{n = 0}^\infty \varphi\left(\frac{x^n}{\lambda^n}\right)$$
      By unicity of Laurent series, this also holds for all $\abs\lambda > r_A(x)$, meaning that $\varphi\left(\frac{x^n}{\lambda^n}\right) \to 0$, as wanted.
    \end{proof}
  \end{field}
\end{note}

% Lecture 18

% Lecture 19

% Lecture 20

% Lecture 21

% Lecture 22

% Lecture 23

% Lecture 24

\end{document}

