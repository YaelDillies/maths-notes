\documentclass{article}

% preamble
\def\npart{III}
\def\nyear{2023}
\def\nterm{Michaelmas}
\def\nlecturer{Prof B\'ela Bollob\'as}
\def\ncourse{Combinatorics}
\def\draft{Incomplete}

\ifx \nauthor\undefined
  \def\nauthor{Ya\"el Dillies}
\else
\fi

\author{Based on lectures by \nlecturer \\\small Notes taken by \nauthor}
\date{\nterm\ \nyear}
\ifdefined\draft
\title{Part \npart\ -- \ncourse\ (\draft)}
\else
\title{Part \npart\ -- \ncourse}
\fi

\usepackage[utf8]{inputenc}
\usepackage{amsmath}
\usepackage{amsthm}
\usepackage{amssymb}
\usepackage{cancel}
\usepackage{enumerate}
\usepackage{mathtools}
\usepackage{fancyhdr}
\usepackage{graphicx}
\usepackage[dvipsnames]{xcolor}
\usepackage{tikz}
\usepackage{wrapfig}
\usepackage{centernot}
\usepackage{float}
\usepackage{braket}
\usepackage{marginnote}
\usepackage{mathdots}
\usepackage{mathrsfs}
\usepackage{ifthen}
\usepackage{imakeidx}
\usepackage{parskip}
\usepackage{relsize}
\usepackage{tabularx}
\usepackage[hypcap=true]{caption}
\usepackage[shortlabels]{enumitem}
\usepackage[pdftex,
  colorlinks=true,
  linkcolor=lblue,
  pdfauthor={\nauthor},
  pdfsubject={Cambridge Maths Notes: Part \npart\ - \ncourse},
  pdftitle={\ncourse - Part \npart},
pdfkeywords={Cambridge Mathematics Maths Math \npart\ \nterm\ \nyear\ \ncourse}]{hyperref}

\usepackage[capitalise,nameinlink,noabbrev]{cleveref}
\usepackage{nameref}
\usepackage[margin=1.5in,a4paper]{geometry}

\reversemarginpar
\newcommand{\lecnum}[1]{\leavevmode\marginnote{\emph{Lecture #1}}\ignorespaces}
\newcounter{lecturenumber}
\newcommand{\newlec}{\stepcounter{lecturenumber}\lecnum{\arabic{lecturenumber}}}

% Theorems
\theoremstyle{definition}
\newtheorem*{aim}{Aim}
\newtheorem*{assumption}{Assumption}
\newtheorem*{axiom}{Axiom}
\newtheorem*{claim}{Claim}
\newtheorem*{cor}{Corollary}
\newtheorem*{conjecture}{Conjecture}
\newtheorem*{defi}{Definition}
\newtheorem*{eg}{Example}
\newtheorem*{egs}{Examples}
\newtheorem*{ex}{Exercise}
\newtheorem*{fact}{Fact}
\newtheorem*{goal}{Goal}
\newtheorem*{idea}{Idea}
\newtheorem*{law}{Law}
\newtheorem*{lemma}{Lemma}
\newtheorem*{notation}{Notation}
\newtheorem*{note}{Note}
\newtheorem*{obs}{Observation}
\newtheorem*{prop}{Proposition}
\newtheorem*{properties}{Properties}
\newtheorem*{question}{Question}
\newtheorem*{rrule}{Rule}
\newtheorem*{steps}{Steps}
\newtheorem*{thm}{Theorem}

\newtheorem*{rmk}{Remark}
\newtheorem*{rmks}{Remarks}
\newtheorem*{warning}{Warning}
\newtheorem*{exercise}{Exercise}

\newtheorem{nthm}{Theorem}[section]
\newtheorem{ndef}[nthm]{Definition}
\newtheorem{nprop}[nthm]{Proposition}
\newtheorem{nconjecture}{Conjecture}
\newtheorem{ncor}[nthm]{Corollary}
\newtheorem{nex}[nthm]{Example}
\newtheorem{nlemma}[nthm]{Lemma}
\newtheorem{problem}[nthm]{Problem}

% Command redirections
\let\P\oldP
\let\oldemptyset\emptyset
\let\emptyset\varnothing

% Letter shorthands
\newcommand{\C}{\mathbb C}
\newcommand{\bbE}{\mathbb E}
\newcommand{\F}{\mathbb F}
\newcommand{\K}{\mathbb K}
\newcommand{\N}{\mathbb N}
\newcommand{\P}{\mathbb P}
\newcommand{\Q}{\mathbb Q}
\newcommand{\R}{\mathbb R}
\newcommand{\Z}{\mathbb Z}
\newcommand{\mcA}{\mathcal A}
\newcommand{\mcB}{\mathcal B}
\newcommand{\mcC}{\mathcal C}
\newcommand{\mcD}{\mathcal D}
\newcommand{\mcE}{\mathcal E}
\newcommand{\mcF}{\mathcal F}
\newcommand{\mcG}{\mathcal G}
\newcommand{\mcH}{\mathcal H}
\newcommand{\mcM}{\mathcal M}
\newcommand{\mcN}{\mathcal N}
\newcommand{\mcO}{\mathcal O}
\newcommand{\mcP}{\mathcal P}
\newcommand{\mcQ}{\mathcal Q}
\newcommand{\mcR}{\mathcal R}
\newcommand{\mcS}{\mathcal S}
\newcommand{\mcT}{\mathcal T}
\newcommand{\mcU}{\mathcal U}
\newcommand{\mcV}{\mathcal V}
\newcommand{\eps}{\varepsilon}
\newcommand{\Eps}{\mathcal E}

\newcommand{\curlybrack}[1]{\left\{ #1\right\}}
\newcommand{\abs}[1]{\left\lvert #1\right\rvert}
\newcommand{\norm}[1]{\left\lVert #1\right\rVert}
\newcommand{\inn}[2]{\left\langle #1, #2\right\rangle}
\newcommand{\floor}[1]{\left\lfloor #1\right\rfloor}
\newcommand{\ceil}[1]{\left\lceil #1\right\rceil}
\newcommand{\doublesqbrack}[1]{[\![#1]\!]}

\newcommand{\imp}{\implies}
\newcommand{\for}{\forall}
\newcommand{\mor}{\rightarrow}
\newcommand{\nin}{\notin}
\newcommand{\comp}{\circ}
\newcommand{\union}{\cup}
\newcommand{\inter}{\cap}
\newcommand{\Union}{\bigcup}
\newcommand{\Inter}{\bigcap}
\newcommand{\hatplus}{\mathbin{\widehat{+}}}
\newcommand{\symdif}{\mathbin\varbigtriangleup}
\newcommand{\aeeq}{\overset{\text{ae}}=}
\newcommand{\lexlt}{\overset{\text{lex}}<}
\newcommand{\colexlt}{\overset{\text{colex}}<}
\newcommand{\wtendsto}{\overset w\mor}
\newcommand{\wstartendsto}{\overset{w*}\mor}
\renewcommand{\vec}[1]{\boldsymbol{\mathbf{#1}}}
\renewcommand{\bar}[1]{\overline{#1}}

\newcommand*{\E}{
  \mathop{
    \mathchoice{\vcenter{\hbox{\larger[4]$\mathbb{E}$}}}
               {\kern0pt\mathbb{E}}
               {\kern0pt\mathbb{E}}
               {\kern0pt\mathbb{E}}
  }\displaylimits
}

\newcommand{\named}[1]{\textbf{#1}\index{#1}}
\newcommand{\bonusnamed}[1]{\textbf{#1}\index{#1@*#1}}

\let\Im\relax
\let\Re\relax

\DeclareMathOperator{\Ber}{Ber}
\DeclareMathOperator{\conv}{conv}
\DeclareMathOperator{\diam}{diam}
\DeclareMathOperator{\codim}{codim}
\DeclareMathOperator{\esssup}{ess sup}
\DeclareMathOperator{\Ext}{Ext}
\DeclareMathOperator{\id}{id}
\DeclareMathOperator{\im}{im}
\DeclareMathOperator{\Im}{Im}
\DeclareMathOperator{\interior}{int}
\DeclareMathOperator{\lhs}{LHS}
\DeclareMathOperator{\rank}{rank}
\DeclareMathOperator{\Re}{Re}
\DeclareMathOperator{\rhs}{RHS}
\DeclareMathOperator{\Span}{Span}
\DeclareMathOperator{\Spec}{Spec}
\DeclareMathOperator{\supp}{supp}
\DeclareMathOperator{\Var}{Var}

\definecolor{lblue}{rgb}{0., 0.05, 0.6}
\definecolor{mblue}{rgb}{0.2, 0.3, 0.8}
\definecolor{morange}{rgb}{1, 0.5, 0}
\definecolor{mgreen}{rgb}{0.1, 0.4, 0.2}
\definecolor{mred}{rgb}{0.5, 0, 0}

\colorlet{bred}{red}
\colorlet{bblue}{Cyan!50!blue}
\colorlet{byellow}{yellow}
\colorlet{bgreen}{YellowGreen!50!Green}
\colorlet{borange}{red!20!yellow}
\colorlet{bpurple}{violet}

\newcommand{\red}[1]{\textcolor{bred}{#1}}
\newcommand{\green}[1]{\textcolor{bgreen}{#1}}
\newcommand{\blue}[1]{\textcolor{bblue}{#1}}
\newcommand{\yellow}[1]{\textcolor{byellow}{#1}}
\newcommand{\orange}[1]{\textcolor{borange}{#1}}
\newcommand{\purple}[1]{\textcolor{bpurple}{#1}}

\pagestyle{fancy}
\fancyhf{}
\fancyfoot[R]{\href{yaeldillies.github.io/maths-notes}{\color{lblue}{Updated online}}}
\fancyfoot[C]{\thepage}
\ifdefined\draft
\fancyfoot[L]{\emph{\draft}}
\else
\fi
\renewcommand{\headrulewidth}{0pt}
\renewcommand{\footrulewidth}{0.2pt}

% Counters and table of content

\swapnumbers
\reversemarginpar

\usetikzlibrary{positioning, decorations.pathmorphing, decorations.text, calc, backgrounds, fadings}
\tikzset{node/.style = {circle,draw,inner sep=0.8mm}}

\makeindex[intoc]
\swapnumbers
\reversemarginpar

\usetikzlibrary{positioning, decorations.pathmorphing, decorations.text, calc, backgrounds, fadings}
\tikzset{node/.style = {circle,draw,inner sep=0.8mm}}

\makeindex[intoc]

\setcounter{section}{-1}

% and here we go!
\begin{document}
\maketitle

\tableofcontents

\clearpage

\section{Introduction}

For a finite set $A$, we write its cardinality $\abs A$.

For a graph $G = (V, E)$ and $A, B \subseteq V$, we denote $\Gamma(A) = \{b | \exists a \in A, a \sim b\}$ the set of neighbors of $A$ and $e(A, B)$ the number of edges between $A$ and $B$.

\clearpage

\section{Basic Results}

\subsection{Chains, Antichains and Scattered Sets of Vectors}

\newlec

During WW2, Littlewood and Offord were interested in roots of polynomials with random coefficients. They came up with the following neat theorem.

\begin{thm}[Littlewood-Offord, 1943]
  If $z_1, \dots, z_n \in \C$ with $\abs{z_i} \ge 1$, then, for any disk $D$ of radius $r$,
  $$\#\{\eps \in \{-1, 1\}^n | \sum_i \eps_i z_i \in D\} \le c \log n \frac{2^n}{\sqrt n}$$
  for some constant $c$ depending only on $r$.
\end{thm}

Upon seeing this theorem, Erd\H os immediately knew he could drastically improve the bound if the $z_i$ were real.

\begin{thm}[Erd\H os, 1945]
  If $x_1, \dots, x_n \in \R$, $\abs{x_i} \ge 1$, then, for any interval $I$ of length $2$,
  $$\#\{\eps \in \{-1, 1\}^n | \sum_i \eps_i z_i \in I\} \le \binom n{\frac n 2}$$
\end{thm}

This is best possible, as we see by taking $x_1 = \dots = x_n = 1$.

Let $G$ be a bipartite graph with parts $U$ and $W$. A {\bf complete matching} from $U$ to $W$ is an injective function $f : U \mor W$ such that $\for u \in U, u \sim f(u)$. \\
If $G$ has a complete matching, then certainly $\abs A \le \abs{\Gamma(A)}$. Surprisingly, this is enough.

\begin{thm}[K\H onig-Egerv\'ary-Hall Theorem, Hall's Marriage Theorem]\label{thm:hall}
  $$G \text{ has a complete matching } \iff \for A \subseteq U, \abs A \le \abs{\Gamma(A)}$$
\end{thm}
\begin{proof}
  Exercise
\end{proof}

Let $\mathcal F = (F_1, \dots, F_m)$ where the $F_i$ are finite sets. We say $a_1, \dots, a_m$ is a {\bf set of distinct representatives}, aka {\bf SDR} if they are distinct and $\for i, a_i \in F_i$. Certainly, if $\mathcal F$ has SDR, then $\abs I \le \abs{\bigcup_{i \in I} F_i}$ for all $I \subseteq [m]$.

\begin{thm}
  $$\mathcal F \text{ is a SDR } \iff \for I \subseteq [m], \abs I \le \abs{\bigcup_{i \in I} F_i}$$
\end{thm}
\begin{proof}
  Define a bipartite graph $G$ with parts $[m]$ and $\bigcup_i F_i$ by $i \sim a \iff a \in F_i$. For all $I \subseteq [m]$, $\abs I \le \abs{\bigcup_{i \in I} F_i} = \abs{\Gamma(I)}$, so Theorem \ref{thm:hall} applies.
\end{proof}

\begin{thm}
  If $G$ is a bipartite graph with parts $U$, $W$ such that $\deg(u) \ge \deg(w)$ for all $u \in U, w \in W$, then there is a complete matching from $U$ to $W$.
\end{thm}
\begin{proof}
  Find $d$ such that $\deg(u) \ge d \ge \deg(w)$ for all $u \in U, w \in W$. For all $A \subseteq U$, we have
  $$d\abs A \le e(A, \Gamma(A)) \le d\abs{\Gamma(A)}$$
  Hence $\abs A \le \abs{\Gamma(A)}$. We're done by Theorem \ref{thm:hall}.
\end{proof}

For $A \subseteq U, B \subseteq W$, define $w(A) = \frac{\abs A}{\abs U}, w(B) = \frac{\abs B}{\abs W}$.

Say a bipartite graph $G$ with parts $U, W$ is {\bf $(k, \ell)$-biregular} if
$\deg(u) = k, \deg(w) = \ell$ for all $u \in U, w \in W$.

\begin{lemma}
  If $G$ is biregular with parts $U, W$ and $A \subseteq U$, then $w(A) \le w(\Gamma(A))$.
\end{lemma}
\begin{proof}
  First, $k\abs U = e(G) = \ell\abs W$. Second,
  $$k\abs A = e(A, \Gamma(A)) \le \ell\abs{\Gamma(A)}$$
  Dividing the inequality by the equality gives the result.
\end{proof}

\newlec

\printindex
\end{document}