% To use these notes, you must copy anki_header.tex
% into the header of your card type in Anki

% layout in Anki:
\documentclass[10pt]{article}
\usepackage[a4paper]{geometry}
\geometry{paperwidth=.5\paperwidth,paperheight=100in,left=2em,right=2em,bottom=1em,top=2em}
\pagestyle{empty}
\setlength{\parindent}{0in}

% encoding:
\usepackage[T1]{fontenc}
\usepackage[utf8]{inputenc}
\usepackage{lmodern}

% packages:
\usepackage{amsmath}
\usepackage{amsfonts}
\usepackage{amsthm}
\usepackage{amssymb}
\usepackage{centernot}
\usepackage{parskip}

% Theorem-like environments
\theoremstyle{definition}
\newtheorem*{claim}{Claim}
\newtheorem*{conjecture}{Conjecture}

% Command redirections
\let\P\oldP
\let\oldemptyset\emptyset
\let\emptyset\varnothing

% Letter shorthands
\newcommand{\C}{\mathbb C}
\newcommand{\E}{\mathbb E}
\newcommand{\F}{\mathbb F}
\newcommand{\K}{\mathbb K}
\newcommand{\N}{\mathbb N}
\newcommand{\P}{\mathbb P}
\newcommand{\Q}{\mathbb Q}
\newcommand{\R}{\mathbb R}
\newcommand{\Z}{\mathbb Z}
\newcommand{\mcA}{\mathcal A}
\newcommand{\mcB}{\mathcal B}
\newcommand{\mcC}{\mathcal C}
\newcommand{\mcD}{\mathcal D}
\newcommand{\mcE}{\mathcal E}
\newcommand{\mcF}{\mathcal F}
\newcommand{\mcG}{\mathcal G}
\newcommand{\mcH}{\mathcal H}
\newcommand{\mcM}{\mathcal M}
\newcommand{\mcN}{\mathcal N}
\newcommand{\mcO}{\mathcal O}
\newcommand{\mcP}{\mathcal P}
\newcommand{\mcQ}{\mathcal Q}
\newcommand{\mcR}{\mathcal R}
\newcommand{\mcS}{\mathcal S}
\newcommand{\mcT}{\mathcal T}
\newcommand{\mcU}{\mathcal U}
\newcommand{\mcV}{\mathcal V}
\newcommand{\eps}{\varepsilon}
\newcommand{\Eps}{\mathcal E}

\newcommand{\curlybrack}[1]{\left\{ #1\right\}}
\newcommand{\abs}[1]{\left\lvert #1\right\rvert}
\newcommand{\norm}[1]{\left\lVert #1\right\rVert}
\newcommand{\inn}[2]{\left\langle #1, #2\right\rangle}
\newcommand{\floor}[1]{\left\lfloor #1\right\rfloor}
\newcommand{\ceil}[1]{\left\lceil #1\right\rceil}
\newcommand{\doublesqbrack}[1]{[\![#1]\!]}

\newcommand{\imp}{\implies}
\newcommand{\for}{\forall}
\newcommand{\nin}{\notin}
\newcommand{\comp}{\circ}
\newcommand{\union}{\cup}
\newcommand{\inter}{\cap}
\newcommand{\Union}{\bigcup}
\newcommand{\Inter}{\bigcap}
\newcommand{\hatplus}{\mathbin{\widehat{+}}}
\newcommand{\symdif}{\mathbin\varbigtriangleup}
\newcommand{\aeeq}{\overset{\text{ae}}=}
\newcommand{\lexlt}{\overset{\text{lex}}<}
\newcommand{\colexlt}{\overset{\text{colex}}<}
\newcommand{\wto}{\overset w\to}
\newcommand{\wstarto}{\overset{w*}\to}
\renewcommand{\vec}[1]{\boldsymbol{\mathbf{#1}}}
\renewcommand{\bar}[1]{\overline{#1}}

\let\Im\relax
\let\Re\relax

\DeclareMathOperator{\Ber}{Ber}
\DeclareMathOperator{\conv}{conv}
\DeclareMathOperator{\diam}{diam}
\DeclareMathOperator{\codim}{codim}
\DeclareMathOperator{\esssup}{ess sup}
\DeclareMathOperator{\Ext}{Ext}
\DeclareMathOperator{\id}{id}
\DeclareMathOperator{\Im}{Im}
\DeclareMathOperator{\interior}{int}
\DeclareMathOperator{\lhs}{LHS}
\DeclareMathOperator{\rank}{rank}
\DeclareMathOperator{\Re}{Re}
\DeclareMathOperator{\rhs}{RHS}
\DeclareMathOperator{\Span}{Span}
\DeclareMathOperator{\Spec}{Spec}
\DeclareMathOperator{\supp}{supp}
\DeclareMathOperator{\Var}{Var}

%  pdf layout:
\geometry{paperheight=74.25mm}
\usepackage{pgfpages}
\pagestyle{empty}
\pgfpagesuselayout{8 on 1}[a4paper,border shrink=0cm]
\makeatletter
\@tempcnta=1\relax
\loop\ifnum\@tempcnta<9\relax
\pgf@pset{\the\@tempcnta}{bordercode}{\pgfusepath{stroke}}
\advance\@tempcnta by 1\relax
\repeat
\makeatother

%  notes, fields, tags:
\def \ifempty#1{\def\temp{#1} \ifx\temp\empty }
\newcommand{\xfield}[1]{
        #1\par
        \vfill
        {\tiny\texttt{\parbox[t]{\textwidth}{\localtag\hfill\\\globaltag\hfill\uuid}}}
        \newpage}
\newenvironment{field}{}{\newpage}
\newif\ifnote
\newenvironment{note}{\notetrue}{\notefalse}
\newcommand{\localtag}{}
\newcommand{\globaltag}{}
\newcommand{\uuid}{}
\newcommand{\tags}[1]{
    \ifnote
        \renewcommand{\localtag}{#1}
    \else
        \renewcommand{\globaltag}{#1}
    \fi
    }
\newcommand{\xplain}[1]{
  \label{#1} % make sure there's no duplicate label
  \renewcommand{\uuid}{#1} % update the UUID for display and Anki disambiguation
  }

\begin{document}

% Lecture 1

\begin{note}
  \xplain{weak-law-large-numbers}
  \xfield{Weak Law of Large Numbers}
  \begin{field}
    Let $X_i$ be iid random variables with finite expectation and second moment. Then, for any $\eps > 0$,
    $$\lim_{n \to \infty} \P\left(\abs{\frac{\sum_{i = 1}^n X_i}n - \frac 12} > \eps\right) = 0$$
    \begin{proof}
      By Chebyshev,
      $$\P\left(\abs{\frac{\sum_{i = 1}^n (X_i - \mu)}n} \ge t\right) \le \frac{n\sigma^2}{n^2t^2} = \frac{\sigma^2}{nt^2} \to 0$$
      assuming we have finite variance.
    \end{proof}
  \end{field}
\end{note}

\begin{note}
  \xplain{central-limit-theorem}
  \xfield{Central Limit Theorem}
  \begin{field}
    Let $X_i$ be iid random variables with mean $\mu$ and variance $\sigma^2$. Then
    $$\frac{\sum_{i = 1}^n(X_i - \mu)}{\sigma\sqrt n} \overset d\to \mathcal N(0, 1)$$
  \end{field}
\end{note}

\begin{note}
  \xplain{chebyshev-inequality}
  \xfield{Chebyshev's inequality}
  \begin{field}
    For a random variable with mean $\mu$ and variance $\sigma^2$,
    $$\P(\abs{X - \mu} \ge t) \le \frac{\sigma^2}{t^2}$$
    \begin{proof}
      By Markov,
      $$\P(\abs{X - \mu} \ge t) = \P((X - \mu)^2 \ge t ^ 2) \le \frac{\sigma^2}{t^2}$$
    \end{proof}
  \end{field}
\end{note}

\begin{note}
  \xplain{talagrand-principle}
  \xfield{Talagrand's principle}
  \begin{field}
    A {\it smooth} function of many {\it independent} random variables concentrates around its mean.
  \end{field}
\end{note}

% Lecture 2

\begin{note}
  \xplain{markov-inequality}
  \xfield{Markov's inequality}
  \begin{field}
    Let $Y$ be a nonnegative random variable. Then for all $t > 0$ we have
    $$\P(T \ge t) \le \frac{\E Y}t$$
    \begin{proof}
      Observe that
      $$Y \ge Y1_{Y \ge t} \ge t1_{Y \ge t}$$
      and take expectations.
    \end{proof}
  \end{field}
\end{note}

\begin{note}
  \tags{log-mgf}
  \xplain{log-mgf-def}
  \xfield{log-MGF of a random variable $Z$}
  \begin{field}
    $$\psi_Z(\lambda) = \log \E e^{\lambda Z}$$
  \end{field}
\end{note}

\begin{note}
  \tags{log-mgf cramer-transform}
  \xplain{cramer-transform-def}
  \xfield{Cramer transform}
  \begin{field}
    $$\psi_Z^*(t) = \sup_{\lambda \ge 0} \lambda t - \phi_Z(\lambda)$$
  \end{field}
\end{note}

\begin{note}
  \tags{cramer-transform}
  \xplain{chernoff-bound}
  \xfield{Chernoff bound}
  \begin{field}
    $$\P(Z \ge t) \le \exp(-\psi_Z^*(t))$$
  \end{field}
\end{note}

\begin{note}
  \tags{log-mgf cramer-transform}
  \xplain{log-mgf-cramer-transform-properties}
  \xfield{Basic properties of $\psi_Z$ and $\psi_Z^*$}
  \begin{field}
    \begin{itemize}
      \item $\psi_Z$ is infinitely differentiable on $]0, \sup \{\lambda \mid \phi_Z(\lambda) < \infty\}[$ because the MGF is.
      \item $\psi_Z$ is convex: If $a, b \ge 0, a + b = 1$, then
      $$\E e^{(a\lambda_1 + b\lambda_2)Z} = \E (e^{\lambda_1X})^a (e^{\lambda_2 Z})^b \le (\E e^{\lambda_1 Z})^a (\E e^{\lambda_2 Z})^b$$
      by Hölder.
      \item $\psi_Z^*$ is nonnegative because $\lambda t - \psi_Z(\lambda) = 0$ when $\lambda = 0$.
      \item $\psi_Z^*$ is convex because it is the supremum of linear functions.
    \end{itemize}
  \end{field}
\end{note}

\begin{note}
  \tags{log-mgf cramer-transform}
  \xplain{cramer-transform-unconstrained}
  \xfield{How to unconstrain $\psi_Z^*$}
  \begin{field}
    If $t > \E Z$ (namely we're looking for a right tail bound), then
    $$\psi_Z* = \sup_\lambda \lambda t - \psi_Z(\lambda)$$
    because in general $\E e^{\lambda Z} \ge e^{\lambda \E Z}$ by Jensen, meaning that $\psi_Z(\lambda) \ge \lambda \E Z$ and that, if $\lambda < 0$ then $$\lambda t - \psi_Z(\lambda) \le \lambda(t - \E Z) < 0 \le \psi_Z^*(t)$$
  \end{field}
\end{note}

% Lecture 3

\begin{note}
  \tags{log-mgf}
  \xplain{log-mgf-gaussian}
  \xfield{MGF and log-MGF of the gaussian distribution}
  \begin{field}
    Complete the square inside the exponent to get
    \begin{align*}
      \E e^{\lambda Z}
      & = \int \frac 1{\sqrt{2\pi\sigma^2}} e^{-\frac{t^2}{2\sigma^2}} e^{\lambda t}\ dt \\
      & = e^{\frac{\lambda^2 \sigma^2}2} \int \frac 1{\sqrt{2\pi\sigma^2}} e^{-\frac{(t - \lambda \sigma^2)^2}{2\sigma^2}}\ dt \\
      & = e^{\frac{\lambda^2 \sigma^2}2}
    \end{align*}
    So the log-MGF is
    $$\psi_Z(\lambda) = \frac{\lambda^2 \sigma^2}2$$
  \end{field}
\end{note}

\begin{note}
  \tags{cramer-transform}
  \xplain{cramer-transform-gaussian}
  \xfield{Cramer transform and Chernoff bound for the gaussian distribution}
  \begin{field}
    The log-MGF of the gaussian distribution is
    $$\psi_Z(\lambda) = \frac{\lambda^2 \sigma^2}2$$
    So $\lambda t - \psi_Z(\lambda) = \lambda t - \frac{\lambda^2 \sigma^2}2$ is maximised at $\lambda = \frac t{\sigma^2}$ and, for all $t \ge 0$,
    $$\psi_Z^*(t) = \sup_{\lambda \ge 0} \lambda t - \frac{\lambda^2 \sigma^2}2 = \frac{t^2}{2\sigma^2}$$
    Hence the Chernoff bound is
    $$\P(Z \ge t) \le \exp\left(-\frac{t^2}{2\sigma^2}\right)$$
  \end{field}
\end{note}

\begin{note}
  \tags{subgaussian}
  \xplain{subgaussian-def}
  \xfield{Subgaussian random variables}
  \begin{field}
    A random variable $X$ with mean $0$ is subgaussian with variance parameter $\nu$ if
    $$\psi_X(\lambda) \le \frac{\lambda^2\nu}2$$
    for all $\lambda$. The set of all subgaussian random variables with variance parameter $\nu$ is denoted $\mathcal G(\nu)$.
  \end{field}
\end{note}

\begin{note}
  \tags{subgaussian}
  \xplain{subgaussian-basic}
  \xfield{Basic properties of subgaussian random variables}
  \begin{field}
    \begin{itemize}
      \item If $X \in \mathcal G(\nu)$, then $\P(X \ge t), \P(X \le -t) \le e^{-\frac{t^2}[2\nu]}$.
      \item If $X \in \mathcal G(\nu)$, then $\P(X \ge t), \P(X \le -t) \le e^{-\frac{t^2}[2\nu]}$.
      \item If $X_i \in \mathcal G(\nu_i)$ are independent, then $\sum_i X_i \in \mathcal G(\sum_i \nu_i)$.
    \end{itemize}
  \end{field}
\end{note}

\begin{note}
  \tags{subgaussian variance}
  \xplain{subgaussian-variance}
  \xfield{If $X \in \mathcal G(\nu)$, then $\Var X \le \nu$.}
  \begin{field}
    We know
    $$\E e^{\lambda X} \le e^{\frac{\lambda^2\nu}2}$$
    Taylor-expanding and using the fact that $\E X = 0$,
    $$1 + \frac{\lambda^2}2 \E X^2 + O(\lambda^3) \le 1 + \frac{\lambda^2}2 \nu + O(\lambda^3)$$
    Taking $\lambda \to 0$,
    $$\Var X = \E X^2 \le \nu$$
  \end{field}
\end{note}

\begin{note}
  \tags{subgaussian}
  \xplain{subgaussian-alt}
  \xfield{Equivalent definitions of subgaussian random variables}
  \begin{field}
    The following are equivalent up to choices of $\nu, b, c, d$:
    \begin{itemize}
      \item $X \in \mathcal G(\nu)$
      \item $\for t > 0, P(X \ge t), \P(X \le -t) \le e^{-\frac{t^2}{2b}}$
      \item $\for q, \E X^{2q} \le q! c^q$
      \item $\E e^{dX^2} \le 2$
    \end{itemize}
  \end{field}
\end{note}

\end{document}

